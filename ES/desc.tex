%-*- ES -*-
%----------------------------------------------------------------------
% Capitulo 7: Descripción del problema

En éste capítulo se verá por qué los protocolos http y smtp son inseguros, introducción al snnifin, spoofing, arp attack
%----------------------------------------------------------------------

(Agregar una intro)


\section{Riesgos de intermediarios}

Un intermediario es alguien que puede ver el contenido de lo que circula por 
la red. A esta actividad o ataque se lo conoce como “man-in-the-middle” u 
hombre en el medio. Los intermediarios pueden tener acceso a información 
relacionada con la seguridad, información personal sobre usuarios individuales 
y organizaciones, o contenido propietario de usuarios y proveedores de 
contenido. HTTP en sí mismo no puede resolver este problema.

En este proyecto, explotaremos esta vulnerabilidad para demostrar la facilidad
 con la que un agente puede hacerse de nuestro tráfico.


\section{(posible a irse) Ataques a traves de la longitud del elemento del protocolo}

Debido a que HTTP utiliza principalmente campos textiales delimitados por caracteres,
los analizadores a menudo son vulnerables a ataques basados en el envío de 
flujos de datos muy largos (o muy lentos), particularmente cuando una 
implementación espera un elemento de protocolo sin una longitud predefinida.


\section{División de respuesta}


La división de respuestas aprovecha una vulnerabilidad en los servidores 
(generalmente dentro de un servidor de aplicaciones) donde un atacante 
puede enviar datos codificados dentro de algún parámetro de la solicitud 
que luego se decodifica y se repite dentro de cualquiera de los campos de 
encabezado en la respuesta. Si los datos decodificados están diseñados para 
que parezca que la respuesta ha finalizado y ha comenzado una respuesta 
posterior, la respuesta se ha dividido y el atacante controla el contenido 
de la segunda respuesta aparente. Luego, el atacante puede realizar 
cualquier otra solicitud en la misma conexión persistente y engañar a los 
destinatarios (incluidos los intermediarios) para que crean que la segunda 
mitad de la división es una respuesta autorizada a la segunda solicitud.



\section{Integridad de los mensajes}

HTTP no define un mecanismo específico para garantizar la integridad de los 
mensajes, sino que se basa en la capacidad de detección de errores de los 
protocolos de transporte subyacentes y en el uso de tramas delimitadas por 
longitud. Históricamente, la falta de un mecanismo de integridad único se 
ha justificado por la naturaleza informal de la mayoría de las comunicaciones 
HTTP. Sin embargo, el predominio de HTTP como mecanismo de acceso a la 
información ha dado como resultado la necesidad de veririficar la integridad
en ciertos entornos.

\section{Confidencialidad del mensaje}

De la misma manera que en la integridad del mensaje, HTTP se basa en 
protocolos de transporte subyacentes para brindar confidencialidad a los 
mensajes cuando se desee. Por sí solo, el protocolo no cifra los mensajes, 
sin embargo, dado que HTTP se ha diseñado para ser independiente del protocolo de 
transporte, de modo que se puede utilizar en muchas formas diferentes de 
conexión cifrada.


\section{Tipos de ataques}
Existen principalmente dos tipos de ataques a la red: ataques pasivos y 
ataques activos. La motivación detrás de los atacantes pasivos y los atacantes 
activos son totalmente diferentes. Mientras que la motivación de los atacantes 
pasivos es simplemente robar información sensible y analizar el tráfico para 
robar mensajes futuros, la motivación de los atacantes activos es detener la 
comunicación normal entre dos entidades legítimas.

\subsection{Passive Attacks}
Los ataques pasivos ocurren cuando se monitorea y analiza 
información sensible, posiblemente comprometiendo la seguridad de las 
empresas y sus clientes. 

Los atacantes pasivos están principalmente interesados en robar información 
confidencial. Esto sucede sin el conocimiento de la víctima. Como tales, 
los ataques pasivos son difíciles de detectar y, por lo tanto, es dificil de
proteger la red de los mismos. Entre los ataques mas comunes estan los 
ataques de análisis y monitoreo de tráfico, como así también las 
escuchas de comunicaciones telefónicas.

\subsection{Active Attacks}
Los ataques activos ocurren cuando la información se modifica, se altera o 
se destruye por completo. Aquí el intruso inicia instrucciones para 
perturbar la comunicación regular de la red. Algunos de los ataques 
activos son los  siguientes:
\begin{itemize}
    \setlength\itemsep{-0.6em}
    \item Modificación: el nodo malicioso realiza algunas alteraciones en la 
    enrutación. Esto da como resultado que el remitente envíe mensajes a 
    través de la ruta larga, lo que provoca un retraso en la comunicación. 
    Este es un ataque a la integridad.
    \item Wormhole(Agujero de gusano): este ataque también se denomina ataque 
    de túnel. Un atacante recibe un paquete en un punto. Luego lo envia 
    a otro nodo malicioso en la red. Esto hace que se piense que se encontró 
    un camino más corto en la red.
    \item Fabricación: un nodo malicioso genera un mensaje de enrutamiento 
    falso que provoca la generación de información incorrecta sobre la ruta 
    entre dispositivos. Este es un ataque a la autenticidad.
    \item Spoofing: un nodo malicioso presenta incorrectamente su identidad 
    para que el remitente cambie su topología, y por consiguiente el 
    destino de sus mensajes.
    \item Denegación de servicios: un nodo malicioso envía un mensaje al 
    nodo y consume el ancho de banda de la red en cómputo desperdiciado.
    \item Ataque Man-in-the-middle: también llamado ataque de secuestro, 
    es un ataque en el que el atacante altera y transmite en secreto 
    las comunicaciones entre dos partes legítimas sin su conocimiento. 
    Estas partes, a su vez, desconocen lo que sucede, pues no perciben 
    un cambio en la comunicación.
\end{itemize}


\section{Caso de estudio: Interceptando la red para obtener credenciales}

\subsection{Herramienta Utilizadas} 
%    \label{capTools}

    \section{Kali Linux}
    BackTrack is one of the most famous Linux distribution systems, as can be proven by
the number of downloads that reached more than four million as of BackTrack Linux
4.0 pre final.

Kali Linux Version 1.0 was released on March 12, 2013. Five days later, Version 1.0.1
was released, which fixed the USB keyboard issue. In those five days, Kali has been
downloaded more than 90,000 times.

    Kali Linux is security-focused Linux distribution based on Debian. It's a rebranded
    version of the famous Linux distribution known as Backtrack, which came with
    a huge repository of open source hacking tools for network, wireless, and web
    application penetration testing. Although Kali Linux contains most of the tools
    from Backtrack, the main aim of Kali Linux is to make it portable so that it could
    be installed on devices based on the ARM architectures such as tablets and
    Chromebook, which makes the tools available at your disposal with much ease.

    Using open source hacking tools comes with a major drawback: they contain a
whole lot of dependencies when installed on Linux and they need to be installed in
a predefined sequence. Moreover, authors of some tools have not released accurate
documentation, which makes our life difficult.

Kali Linux simplifies this process; it contains many tools preinstalled with all the
dependencies and is in ready to use condition so that you can pay more attention
for the actual attack and not on installing the tool. Updates for tools installed in Kali
Linux are more frequently released, which helps you to keep the tools up to date. A
non-commercial toolkit that has all the major hacking tools preinstalled to test realworld networks and applications is a dream of every ethical hacker and the authors
of Kali Linux make every effort to make our life easy, which enables us to spend
more time on finding the actual flaws rather than building a toolkit.

(EXPLICAR COMO Y DONDE LO INSTALE)
(DECIR QUE FUE EL SISTEMA OPERATIVO UTILIZADO QUE CONTIENEN LAS SIGUIENTES HERRAMIENTAS)
\section{Wireshark}
Wireshark is one of the most popular, free, and open source network protocol
analyzers. Wireshark is preinstalled in Kali and ideal for network troubleshooting,
analysis, and for this chapter, a perfect tool to monitor traffic from potential targets
with the goal of capturing session tokens. Wireshark uses a GTK+ widget toolkit to
implement its user interface and pcap to capture packets. It operates very similarly to
a tcpdump command; however, acting as a graphical frontend with integrated sorting
and filtering options.
(HAY MAS, CON GRAFICOS EN)
Joseph Muniz, Aamir Lakhani - Web Penetration Testing with Kali Linux-Packt Publishing (2013)

    \section{Ettrcap}
    Ettercap is a free and open source comprehensive suite for man-in-the-middle-based
attacks.
Ettercap can be used for computer network protocol analysis and security auditing,
featuring sniffing live connections, content filtering, and support for active and passive
dissection of multiple protocols. Ettercap works by putting the attacker's network
interface into promiscuous mode and ARP for poisoning the victim machines.



(HAY MAS, CON GRAFICOS EN)
Joseph Muniz, Aamir Lakhani - Web Penetration Testing with Kali Linux-Packt Publishing (2013)
Aca yo segui un tutorial, buscarlo

La idea principal de esta sección es demostrar que, encontrandose en una red interna
y con con herramientas ya desarrolladas y libres es posible realizar un ataque 
sin necesidad de conocer a fondo la implementacion de la misma ni de tener mayores
privilegios

Recordar que esto fue realizado en una red interna donde son todos equipos de nuestra propiedad

\subsection{Diagrama de explicacion}
IMAGEN de le red

Tiene que estar:
-Router

-Origen de la pagina

-Consumidor de la pagina

-El atacante

\subsection{Preparando Ettercap para el ataque ARP Poisoning}

Lo primero que debemos hacer, en la lista de aplicaciones, es buscar el apartado 
«9. Sniffing y Spoofing«, ya que es allí donde encontraremos las herramientas necesarias
 para llevar a cabo este ataque.

IMAGEN Kali Linux Spoofing

A continuación, abriremos «Ettercap» y veremos una ventana similar a la siguiente.

IMAGEN  Kali Linux Ettercap

El siguiente paso es seleccionar la tarjeta de red con la que vamos a trabajar. Para ello, en el menú superior de Ettercap seleccionaremos «Sniff > Unified Sniffing» y, cuando nos lo pregunte, seleccionaremos nuestra tarjeta de red (por ejemplo, en nuestro caso, eth0).

IMAGEN Kali Linux - Ettercap - Tarjeta de red

El siguiente paso es buscar todos los hosts conectados a nuestra red local. Para ello, seleccionaremos «Hosts» del menú de la parte superior y seleccionaremos la primera opción, «Hosts List«.

IMAGEN Kali Linux - Ettercap - Lista de hosts

Aquí deberían salirnos todos los hosts o dispositivos conectados a nuestra red. Sin embargo, en caso de que no salgan todos, podemos realizar una exploración completa de la red simplemente abriendo el menú «Hosts» y seleccionando la opción «Scan for hosts«. Tras unos segundos, la lista de antes se debería actualizar mostrando todos los dispositivos, con sus respectivas IPs y MACs, conectados a nuestra red.

IMAGEN Kali Linux - Ettercap - Lista de hosts 2

\subsection{Nuestro Ettercap ya está listo. Ya podemos empezar con el ataque ARP Poisoning}

En caso de querer realizar un ataque dirigido contra un solo host, por ejemplo, suplantar la identidad de la puerta de enlace para monitorizar las conexiones del iPad que nos aparece en la lista de dispositivos, antes de empezar con el ataque debemos establecer los dos objetivos.

Para ello, debajo de la lista de hosts podemos ver tres botones, aunque nosotros prestaremos atención a los dos últimos:

    Target 1 – Seleccionamos la IP del dispositivo a monitorizar, en este caso, el iPad, y pulsamos sobre dicho botón.
    Target 2 – Pulsamos la IP que queremos suplantar, en este caso, la de la puerta de enlace.

IMAGEN Objetivos Ettercap

Todo listo. Ahora solo debemos elegir el menú «MITM» de la parte superior y, en él, escoger la opción «ARP Poisoning«.

IMAGEN Kali Linux - Ettercap - Ataques MITM

Nos aparecerá una pequeña ventana de configuración, en la cual debemos asegurarnos de marcar «Sniff Remote Connections«.

IMAGEN Comenzar MITM ARP Poisoning

Pulsamos sobre «Ok» y el ataque dará lugar. Ahora ya podemos tener el control sobre el host que hayamos establecido como «Target 1«. Lo siguiente que debemos hacer es, por ejemplo, ejecutar Wireshark para capturar todos los paquetes de red y analizarlos en busca de información interesante o recurrir a los diferentes plugins que nos ofrece Ettercap, como, por ejemplo, el navegador web remoto, donde nos cargará todas las webs que visite el objetivo.

Plugins Ettercap


