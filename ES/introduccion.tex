%-*- ES -*-
%----------------------------------------------------------------------
% capitulo 1: Introduccion
%----------------------------------------------------------------------
% Estructura del capitulo:
%
%----------------------------------------------------------------------


% se pone una introducci�n al tema en general
A lo largo de la carrera, y, particularmente, en una de mis materias preferidas, 
Seguridad en Sistemas, nos han explicado la importancia la protección de los 
datos, como así también la de los canales de comunicación. Mientras se desarrolla 
una red segura, se debe considerar la confidencialidad, integridad y disponibilidad 
(CIA). 

También, en nuestra corta experiencia hemos visto cómo las organizaciones 
abordan los temas de seguridad en sus sistemas informáticos. Mayormente, se 
concentran en los equipos que están expuestos a la red pública, dejando de 
lado los que se encuentran aislados de la misma. Erróneamente, muchas veces 
se piensa que es suficiente, sin embargo, puede traer graves inconvenientes. 
Es por eso que realizaremos un estudio teórico/práctico sobre las consecuencias 
de la navegación en redes internas sin ningún tipo de cifrado de datos ni 
certificaciones.

\section{Objetivos}

\begin{itemize}
    \setlength\itemsep{-0.6em}
    \item Aplicar los conocimientos en seguridad en redes para desplegar aplicaciones de red.
    \item Aprender el uso de nuevas herramientas de administración, automatización y seguridad en sistemas.
    \item Montar un escenario virtual que sirva de pruebas frente a la gestión de certificados, administración de la infraestructura y a la detección de debilidades dentro una red privada.
    \item Concientizar la implementación de medidas de seguridad en redes internas
    \item Implementar una mejora en la navegación en una red interna utilizando Docker
\end{itemize}


\section{Metodología de trabajo}
El trabajo se dividirá en tres etapas:

-	Primera: Se investigarán y plantearán escenarios donde la navegación insegura pueda traer problemas asociados dentro de la organización. 

-	Segunda: Se plantearán escenarios de trabajo buscando adquirir experiencia en la utilización de la herramienta Docker, para poder obtener parámetros que nos permitan realizar comparaciones con tecnologías similares y sus áreas de aplicación.

-	Tercera: Se investigarán e implementarán posibles soluciones para afrontar los 
aspectos planteados en la primera etapa 



