%-*- ES -*-
%----------------------------------------------------------------------
% capitulo 1: Introduccion
%----------------------------------------------------------------------
% Estructura del capitulo:
%
%----------------------------------------------------------------------


% se pone una introducci�n al tema en general
A lo largo de la carrera, y, particularmente, en una de mis materias preferidas,  
Seguridad en Sistemas, nos han explicado la importancia la protección de los 
datos, como así también la de los canales de comunicación. Al desplegar 
una red segura, se deben considerar los siguientes aspectos centrales de la 
seguridad en sistemas, tales como la confidencialidad, la integridad y la disponibilidad 
(CIA). 

En nuestra corta experiencia hemos observado como las organizaciones 
abordan los temas de seguridad en sus sistemas informáticos y en las redes de 
datos . En general, se 
concentran en los equipos que están expuestos a la red pública, dejando de 
lado los que se encuentran aislados de la misma. Muchas veces 
se piensa que es suficiente, sin embargo, puede traer graves inconvenientes. 
Es por eso que realizaremos un estudio teórico/práctico sobre las consecuencias 
de la navegación en redes internas sin ningún tipo de cifrado de datos ni 
certificaciones, lo cual puede traer graves inconvenientes en cuanto a la 
seguridad interna. Entre las consecuencias podemos ejemplificar con sistemas 
internos que no proveen conexiones seguras y descuidan la autenticación 
confidencial de los usuarios.


\section{Objetivos}

\begin{itemize}
    \setlength\itemsep{-0.6em}
    \item Aplicar los conocimientos en seguridad en redes para desplegar aplicaciones de red.
    \item Aprender el uso de nuevas herramientas de administración, automatización y seguridad en sistemas.
    \item Montar un escenario virtual que sirva de pruebas frente a la gestión de certificados, 
    administración de la infraestructura y la detección de debilidades dentro una red privada.
    \item Se desarrollarán medidas de seguridad en una red interna utilizando Docker como centro de 
    pruebas.
\end{itemize}


\section{Metodología de trabajo}
El trabajo se dividirá en tres etapas:

-	Primera: Se mostrarán escenarios donde la navegación insegura puede traer problemas 
asociados dentro de la organización. 

-	Segunda: Se plantearán escenarios de trabajo buscando adquirir experiencia en la 
utilización de la herramienta Docker, para poder obtener parámetros que nos permitan 
realizar comparaciones con tecnologías similares y sus áreas de aplicación.

-	Tercera: Se definirán posibles soluciones para afrontar 
los aspectos planteados en la primera etapa. 



