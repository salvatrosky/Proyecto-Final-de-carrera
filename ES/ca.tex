\subsubsection*{Creación de nuestra CA en Docker}

La estrategia para crear nuestra ca será seguir los pasos que se deberian realizar en un servidor 
habitual, pero partiendo desde una imagen de docker de ubuntu (de stock), y luego realizando un 
commit de estas configuraciones. Luego, archivos importantes como el el certificado root y la llave
privada deberan resguardarse, o simplemente resguardar el contenedor creado. 

\noindent Como primer paso, corremos una imagen del sistema operativo Ubuntu

\begin{verbatim}
    docker run -it -v $PWD/ca:/root/ca ubuntu
\end{verbatim}

\noindent Hay que ver las cosas que se hacen una vez, y las cosas que deben estar configuradas

\begin{verbatim}
    apt-get update
    apt-get install ntp
    apt-get install openssl
\end{verbatim}

\noindent Setear el hostname al contenedor, hay una linea con la ip del contenedor y el nombre del mismo, 
que es utilizado como hostname, en nuestro caso
\begin{verbatim}
    172.17.0.2      080dec560726
\end{verbatim}

\noindent Lo cambiamos por un hostname con el dominio incluido
\begin{verbatim}
    172.17.0.2      ca.salvadorcatalfamo.intra
\end{verbatim}

\noindent Creamos las carpetas para mejor organizacion
\begin{verbatim}
    mkdr newcerts
    mkdir certs
    mkdir crl
    mkdir private
    mkdir requests
\end{verbatim}

\noindent Creamos un archivo vacio y un archivo que contiene el primer numero de serie para los certificados 
\begin{verbatim}
    touch index.txt
    echo '1234' > serial
\end{verbatim}

\noindent Luego hay que crear la llave privada y el certificado root, en este caso nos pedira una contraseña, si este servidor se usará en un ambiente de producción, deberá ser una contraseña compleja

\begin{verbatim}
    openssl genrsa -aes256 -out private/cakey.pem
\end{verbatim}

\noindent Una vez que generamos la llave privada, la misma será utilizada como entrada en la creación de 
nuestro certificado root. Nos pedirá algunos datos de localización y relacionados a la 
organización

\begin{verbatim}
    open ssl req -new -x509 -key /root/ca/private/cakey.pem -out cacert.pem -days 3650
\end{verbatim}

\noindent Cambiamos los permisos de los archivos que creamos
\begin{verbatim}
    chmod 600 -R /root/ca
\end{verbatim}

\noindent Realizamos unas modificaciones en el archivo de configuración
\begin{verbatim}
    vim /usr/lib/ssl/openssl.cnf
\end{verbatim}

 
Lo único que se tiene que cambiar es la siguiente línea 
 
Por la direcciona de los certificados, en nuestro caso el directorio /root/ca
Luego cambiamos algunas configuraciones opcionales de políticas.

Una vez que realizamos estos pasos, estamos listos para realizar el commit de la imagen, con esto,
todos los pasos que realizamos (instalar los paquetes, modificar los archivos de configuración, etc)
no son necesarios que se ejecuten nuevamente.

Para realizar un commit, y que nuestro contenedor sea fácilmente identificable, deberemos seguir 
los siguientes pasos

\begin{verbatim}
    docker ps -a #identificamos el ultimo contenedor utilizado
\end{verbatim}
 
(IMAGEN DOCKER-PS-A)

\begin{verbatim}
    docker commit {id_del_contenedor} 
    docker image ls #Identificamos la imagen recien creada, no tendrá ni repositorio ni tag
\end{verbatim}

(IMAGEN DOCKER IMAGE LS)

\begin{verbatim}
    docker image ls #Identificamos la imagen recien creada, con repositorio y tag
\end{verbatim}

(IMAGEN DOCKER IMAGE LS 2)

\begin{verbatim}
    docker run -it -v $PWD/ca:/root/ca ca:1.0
\end{verbatim}

Em el comando anterior, estamos asumiendo que queremos compartir los archivos de la CA con el servidor 
host.

Ahora cada vez que querramos correr nuestra ca, lo haremos de la siguiente manera 

\subsubsection*{Nginx con Docker}

Para probar nuestro certificado, utilizaremos una imagen oficial de Nginx, 
los archivos de configuración
son los siguientes:


\begin{verbatim}
docker-compose.yml

web:
  image: nginx
  volumes:
   - ./pagina1:/usr/share/nginx/html:ro
  ports:
   - "80:80"
  environment:
   - NGINX_HOST=pagina1.salvadorcatalfamo.intra
   - NGINX_PORT=80
\end{verbatim}

En el archivo mostrado, le decimos a Docker que utilice la imagen Nginx, 
que nuestros archivos fuentes
van a estar en el directorio \textit{./pagina1} y que exponga el puerto 80, entre otras
 cosas.

\noindent Para ejecutar este contenedor, se debe ejecutar el siguiente comando:

\begin{verbatim}
    docker-compose up -d
\end{verbatim}




\subsubsection*{Creando nuestro certificado}

Para firmar un certificado, el servidor donde se alojará la web debe realizar una solicitud, 
donde nuestra CA retornará el certificado firmado. Desde el servidor web en cuestión, se debe 
crear una llave privada privada:

\begin{verbatim}
    Openssl genrsa -aes256 -out webserver.pem 2048
\end{verbatim}

\noindent Luego, se deberá crear la solicitud de firma de certificado:
\begin{verbatim}
    Openssl req -new -key webserver.pem -out webserver.csr
\end{verbatim}

\noindent Luego enviamos esta solicitud y la firmamos en nuestra CA, esta solicitud la vamos a colocar en el directorio /root/ca/requests
\begin{verbatim}
    Open ssl cd -in webserver.csr -out webserber.crt
\end{verbatim}

\subsubsection*{Configuración del certificado en el servidor web}
\noindent Una vez que tenemos este certificado, lo colocamos en en el servidor web y lo configuramos. Para el caso de ngixn, se debe:

…
…

\noindent Ahora vamos a ver el navegador, podemos ver que aunque tenga el certificado instalado, no es confiable ya que nuestra entidad certificante no esta configurada como confiable.

…

\noindent Para instalar esta nueva CA en windows, debemos seguir los siguientes pasos.

…
…

\noindent Luego, ya podemos ver que nuestra página aparece como una web confiable.




