%-*- ES -*-
%----------------------------------------------------------------------
% Capitulo 7: Conclusiones generales
%----------------------------------------------------------------------

En este proyecto abordamos una amplia cantidad de temas de manera resumida, con un gran potencial 
de estudio por delante. La navegación \emph{web} nos permitió encontrar una 
extensa cantidad conceptos para desarrollar, tales como los protocolos \emph{HTTP}, \emph{HTTPS}, \emph{DNS}, 
y los posibles procesos que se pueden realizar para segurizar nuestra red.

Con respecto a \emph{Docker}, quiero destacar la amplia comunidad existente, ya que ésta 
nos brinda muchas facilidades, tales como ejemplo e instructivos que nos ayuda 
 a la hora de trabajar. El tiempo que 
nos ahorramos con \emph{Docker} fue destinado a entender el funcionamiento de la seguridad en las redes, 
tales como el uso de los certificados \emph{SSL} y las entidades certificantes.

A pesar del contenido teórico propuesto, el trabajo se concentró en las soluciones presentadas 
en el capítulo 4. Allí se explicó brevemente como con un corto conocimiento en el funcionamiento 
del protocolo \emph{SSL} se pueden establecer mecanismos para que el tráfico \emph{web} vaya de manera segura; 
uno de nuestros principales objetivos, junto con el de concientizar a las personas de la 
importancia de conocer los sitios por los que se navega. Aunque en nuestro caso abordamos las 
redes pertenecientes a una organización, el contenido de la información es válido para cualquier 
sitio \emph{web} que se visita, ya sea interno o externo. 
