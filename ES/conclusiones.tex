%-*- ES -*-
%----------------------------------------------------------------------
% Capitulo 7: Conclusiones generales
%----------------------------------------------------------------------

En este proyecto abordamos una amplia cantidad de temas de manera resumida, con un gran potencial 
de estudio por delante. Aunque en un principio se deseaba investigar todo el tráfico que pasaba por 
la red interna, con todas las posibles debilidades, la navegación web nos permitió encontrar una 
extensa cantidad de nuevos conceptos para desarrollar, tales como los protocolos HTTP, HTTPS, DNS, 
y los posibles procesos que se pueden realizar para segurizar nuestra red.

Con respecto a Docker, no nos resultó muy complejo realizar nuestras tareas, ya que la comunidad 
es muy amplia, y las herramientas que nos brindan están al alcance de nuestra mano. El tiempo que 
nos ahorramos con Docker fue destinado a entender el funcionamiento de la seguridad en las redes, 
tales como el uso de los certificados SSL y las entidades certificantes.

A pesar del contenido teórico propuesto, el trabajo se concentró en las soluciones presentadas 
en el capítulo 4. Allí se explicó brevemente como con un corto conocimiento en el funcionamiento 
del protocolo SSL se pueden establecer mecanismos para que el tráfico web vaya de manera segura; 
uno de nuestros principales objetivos, junto con el de concientizar a las personas de la 
importancia de conocer los sitios por los que se navega. Aunque en nuestro caso abordamos las 
redes pertenecientes a una organización, el contenido de la información es válido para cualquier 
sitio web que se visita, ya sea interno o externo. 
