% resumen en castellano
\thispagestyle{empty}
\chapter*{Resumen}

% Aca va el resumen en castellano

\bigskip
A lo largo de la carrera, y, particularmente, en una de mis materias preferidas, 
Seguridad en Sistemas, nos han explicado la importancia la protección de los 
datos, como así también la de los canales de comunicación. Mientras se desarrolla 
una red segura, se debe considerar la confidencialidad, integridad y disponibilidad 
(CIA). 

También, en nuestra corta experiencia hemos visto como las organizaciones 
abordan los temas de seguridad en sus sistemas informáticos. Mayormente, se 
concentran en los equipos que están expuestos a la red pública, dejando de 
lado los que se encuentran aislados de la misma. Erróneamente, muchas veces 
se piensa que es suficiente, sin embargo, puede traer graves inconvenientes. 
Es por eso que realizaremos un estudio teórico/práctico sobre las consecuencias 
de la navegación en redes internas sin ningún tipo de cifrado de datos ni 
certificaciones.

\bigskip
\noindent \textsc{Palabras Clave:} \par

Seguridad e Infraestructura\par
Docker \par
Linux \par
Kali \par
Máquinas Virtuales  \par
Entidad Certificante CA \par
Protocolo SSL \par
Navegación segura \par
Redes internas \par


