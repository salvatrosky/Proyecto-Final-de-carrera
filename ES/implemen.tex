%-*- ES -*-
%----------------------------------------------------------------------
% Capitulo: Implementacion de la solución propuesta
%----------------------------------------------------------------------



Se han investigado tres alternativas enfocadas en redes internas mara mejorar la seguridad de las 
mismas, luego de eso se eligió la que mas ventajas ons ofrecio.
Vimos tres alternativas posibles: Los certificados auto-firmado (Self-signed Certificates), implementar
una entidad certificante interna, y la utilizacion de un certificado emitido por una entidad
certificante conocida

\section{Self-signed Certificates}
Self-signed certificates are the least useful of the three. Firefox
makes it easier to use them safely; you create an exception on the
first visit, after which the self-signed certificate is treated as valid
on subsequent connections. Other browsers make you clickthrough a certificate
 warning every time.Unless you’re
actually checking the certificate fingerprint every time, it is not
possible to make that self-signed certificate safe. Even with
Firefox, it might be difficult to use self-signed certificates safely.

For example, to request an SSL certificate from a trusted CA like Verisign or GoDaddy, you send them a 
Certificate Signing Request (CSR), and they give you a certificate in return, that they signed
 using their root certificate and private key. All browsers have a copy (or access a copy from
  the operating system) of them root certificate, so the browser can verify that your certificate
   was signed by a trusted CA.

When we generate a self-signed certificate, we generate our own root certificate and private 
key. Because you generate a self-signed certificate the browser doesn’t trust it. It’s 
self-signed. It hasn’t been signed by a CA. All certificates that we generate and sign will 
be inherently trusted.

La principal dificultad es que los usuarios siempre encontrarán una advertencia 
donde el navegador diga que se encuentra en un sitio con un certificado autofirmado. 
En la mayoria de los casos, no verificarán que el certificado es el correcto, por lo que 
generará desconfianza en los usuarios.

Today, self-signed certificates are considered insecure because there is no way for average users to
differentiate them from self-signed MITM certificates. In other words, all self-signed certificates
look the same. But, we can use a secure DNS to pin the certificate, thus allowing our user agent to
know that they are using the right one. MITM certificates are easily detected.

In virtually all cases, a much better approach is to use a private
CA. It requires a little more work upfront, but once the
infrastructure is in place and the root key is safely distributed to
all users, such deployments are as secure as the rest of the PKI
ecosystem.
\section{Internal CA}

Aca se tiene que decir: "Como se explicó anteriormente, una entidad de certificacion es ..."
Esta alternativa implica establecer una entidad de certificacion interna a la red privada. Esto se hace 
mediente un servidor dedicado que certifique los certificados que circulen internamente.
Como ventaja se tiene que ... 

Advantages internal Certificate Authority (CA)

• Simplified and ease of management is the main advantage of using internal Certificate 
Authority (CA). There is no need to depend an external entity for certificates.

• In a Microsoft Windows environment, internal Certificate Authority (CA) can be integrated in Active
 Directory. This further simplifies the management of the CA structure.

• There is no cost per certificate wen you are using an internal Certificate Authority (CA).

Disadvantages of internal Certificate Authority (CA)

• Implementing internal Certificate Authority (CA) is more complicated than using external 
Certificate Authority (CA).

• The security and accountability of Public Key Infrastructure (PKI) is completely on the 
organization's shoulder.

• External parties/users normally will not trust a digital certificate signed by an internal Certification 
Authority (CA).

• The certificate management overhead of internal Certification Authority (CA) is higher than that 
of external Certification Authority (CA).

La desventaja por la que decidí ir por una mejor opcion fue que, se debe establecer individualmente en cada uno de los hosts pertenecientes
a la red privada que nuestra entidad certificante es de confianza, lo que puede llevar un gran trabajo de los administradores, y, 
aun asi, pueden suceder que en nuestra red se conecten agentes externos a nuestra organizacion, por lo 
que no podremos realizar la configuración mencionada

\section{Cerfification with let's encrypt}
Esta estrategia consiste en generar un certificado wildrau, y utilizarlo en cada sitio de la organizacion
Para esto se debe tener un dominio, en mi caso, salvadorcatalfamo.com, tener la propiedad del dominio
implica poder manejar registros dns del mismo, que es lo que requiere letsencript para poder verificar el mismo
La verificacion es la minima, que es la de que dice cque soy el dueño del dominio
y la verificacion de que cada sitio es mio es la del dns, donde se hace la verificacion con el
registro dns
Entonces, formalmente los requisitos solución
tener el dominio
agregar el registro txt al dns
solicitar la llave publica y privada 
colocarla en cada sitio

\subsection{Pasos a seguir}
\subsubsection*{Get a Domain}
The firs step is getting a Domain, in my case, I had one: salvadorcatalfamo.com.
This domain points to mi public ip address. For that, I had to create some DNS records:


\begin{longtable}{|l|l|p{5cm}|l|l|} 
   \hline
   \textbf{Tipo} & \textbf{Nombre} & \textbf{Contenido} & \textbf{Prioridad} & \textbf{TTL}
\\ \hline A  & salvadorcatalfamo.com & 181.228.121.12 & 0 & 14400
\\ \hline NS  & salvadorcatalfamo.com & ns1.donweb.com & 0 & 14400
\\ \hline SOA & salvadorcatalfamo.com & ns2.donweb.com & 0 & 14400
\\ \hline SOA & salvadorcatalfamo.com & ns3.hostmar.com \newline root.hostmar.com 
                                       \newline 2021010700 28800 7200 
                                       \newline 2000000 86400
                                       \newline ns2.donweb.com & 0 & 14400                  


\\ \hline
\end{longtable}

\subsubsection*{Installing Let’s Encrypt on the server}
\begin{verbatim}
   sudo add-apt-repository ppa:certbot/certbotsudo 
   apt-get update
   sudo apt-get install python-certbot-nginx
\end{verbatim}

\subsubsection*{Installing Nginx}
\begin{verbatim}
sudo apt-get update
sudo apt-get install nginx
\end{verbatim}


\subsubsection*{Obtaining wildcard ssl certificate from Let’s Encrypt}
\begin{verbatim}
sudo certbot --server https://acme-v02.api.letsencrypt.org/directory 
-d *.salvadorcatalfamo.com --manual --preferred-challenges dns-01 certonly
\end{verbatim}

\subsubsection*{Deploy a DNS TXT record provided by Let’s Encrypt certbot after running the above command}
Then I added an entry to my dns records
\begin{longtable}{|l|l|l|l|l|} 
   \hline
   \textbf{Tipo} & \textbf{Nombre} & \textbf{Contenido} & \textbf{Prioridad} & \textbf{TTL}
\\ \hline TXT  & 	\_acme-challenge.salvadorcatalfamo.com & 11UZJD27bPDb\_jFs6f... & 0 & 14400
\\ \hline
\end{longtable}

\subsubsection*{Configuring Nginx to serve wildcard subdomains}

Create a config file sudo touch /etc/nginx/sites-available/example.com

Open the file sudo vi /etc/nginx/sites-available/example.com

Add the following code in the file

\begin{verbatim}
   server {
      listen 80;
      listen [::]:80;
      server_name *.example.com;
      return 301 https://$host$request_uri;
   }
   server {
      listen 443 ssl;
      server_name *.example.com;  ssl_certificate /etc/letsencrypt/live/example.com/fullchain.pem;
      ssl_certificate_key /etc/letsencrypt/live/example.com/privkey.pem;
      include /etc/letsencrypt/options-ssl-nginx.conf;
      ssl_dhparam /etc/letsencrypt/ssl-dhparams.pem;  root /var/www/example.com;
      index index.html;
      location / {
        try_files $uri $uri/ =404;
      }
    } 
\end{verbatim}

The above server block is listening on port 80 and redirects the request to the server block below 
it that is listening on port 443.

\subsubsection*{Test and restart Nginx}
EXTEDER
Test Nginx configuration using 
sudo nginx -t
If it’s success reload Nginx using 
\begin{verbatim}
sudo /etc/init.d/nginx reload
\end{verbatim}
Nginx is now setup to handle wildcard subdomains.

Ver una posible automatizacion, aunque creo que será con puppet

ventajas
No mas mensajes de errores
Seguridad de encriptacion 
privacidad, etc etc

Desventajas
Tal vez la cantidad de dominios gratis
Tal vez la duracion de validez del certificado
Tal vez la confiabilidad



