
\subsubsection*{Kali Linux}
Kali Linux es una distribución de Linux (basada en Debian) centrada en la seguridad. 
Es una versión renombrada de la famosa distribución de Linux conocida como Backtrack, 
que venía con un enorme repositorio de herramientas de piratería de código abierto, 
para pruebas de penetración de aplicaciones \emph{web}, inalámbricas y de red. 

\begin{center}
    \begin{figure}   
       \begin{center}
          \includegraphics[width=11cm,height=7cm]{ataque-1.png}
       \end{center}
       \caption{Escritorio de Kali Linux}
    \end{figure}
 \end{center}
 

Kali Linux contiene muchas herramientas preinstaladas con 
todas las dependencias y ya está lista para usar. Esto nos permite tener que prestar 
más atención a las pruebas y no a la instalación de la herramienta. Las actualizaciones 
para las herramientas instaladas en Kali Linux se publican con mayor frecuencia, 
lo que le ayuda a mantener las a las mismas actualizadas.

Esta distribución contiene las herramientas necesarias para realizar nuestro
ataque.

\subsubsection*{Wireshark}
Wireshark es uno de los analizadores de protocolos de red más populares, es de 
código abierto y gratuito. Wireshark está preinstalado en Kali y es ideal para la 
resolución de problemas de red, análisis y, para este caso de estudio, una herramienta 
perfecta para monitorear el tráfico de posibles objetivos. Wireshark usa un kit de 
herramientas para implementar su interfaz de usuario y para capturar paquetes. 
Funciona de manera muy similar a un comando \emph{tcpdump}; sin embargo, nos brinda
una interfaz gráfica, posee opciones integradas de clasificación y filtrado.

\begin{center}
    \begin{figure}   
       \begin{center}
          \includegraphics[width=10cm,height=7cm]{wireshark.png}
       \end{center}
       \caption{Interface del Wireshark}
    \end{figure}
 \end{center}
 
\subsubsection*{Ettercap}

\begin{center}
   \begin{figure}   
      \begin{center}
         \includegraphics[width=10 cm,height=7cm]{ataque-2.png}
      \end{center}
      \caption{Ettercap}
   \end{figure}
\end{center}

Ettercap es un paquete completo gratuito y de código abierto para ataques basados 
en intermediarios. Ettercap se puede utilizar para análisis de protocolos de redes 
informáticas y auditorías de seguridad, con funciones de rastreo de conexiones en 
tiempo real y filtrado de contenido. Ettercap funciona configurando la interfaz de red 
del atacante en modo promiscuo y \emph{ARP} para envenenar las máquinas víctimas.

