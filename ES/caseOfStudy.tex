
\subsection{Herramienta Utilizadas} 
    

    \section{Kali Linux}
    BackTrack is one of the most famous Linux distribution systems, as can be proven by
the number of downloads that reached more than four million as of BackTrack Linux
4.0 pre final.

Kali Linux Version 1.0 was released on March 12, 2013. Five days later, Version 1.0.1
was released, which fixed the USB keyboard issue. In those five days, Kali has been
downloaded more than 90,000 times.

    Kali Linux is security-focused Linux distribution based on Debian. It's a rebranded
    version of the famous Linux distribution known as Backtrack, which came with
    a huge repository of open source hacking tools for network, wireless, and web
    application penetration testing. Although Kali Linux contains most of the tools
    from Backtrack, the main aim of Kali Linux is to make it portable so that it could
    be installed on devices based on the ARM architectures such as tablets and
    Chromebook, which makes the tools available at your disposal with much ease.

    Using open source hacking tools comes with a major drawback: they contain a
whole lot of dependencies when installed on Linux and they need to be installed in
a predefined sequence. Moreover, authors of some tools have not released accurate
documentation, which makes our life difficult.

Kali Linux simplifies this process; it contains many tools preinstalled with all the
dependencies and is in ready to use condition so that you can pay more attention
for the actual attack and not on installing the tool. Updates for tools installed in Kali
Linux are more frequently released, which helps you to keep the tools up to date. A
non-commercial toolkit that has all the major hacking tools preinstalled to test realworld networks and applications is a dream of every ethical hacker and the authors
of Kali Linux make every effort to make our life easy, which enables us to spend
more time on finding the actual flaws rather than building a toolkit.

(EXPLICAR COMO Y DONDE LO INSTALE)
(DECIR QUE FUE EL SISTEMA OPERATIVO UTILIZADO QUE CONTIENEN LAS SIGUIENTES HERRAMIENTAS)
\section{Wireshark}
Wireshark is one of the most popular, free, and open source network protocol
analyzers. Wireshark is preinstalled in Kali and ideal for network troubleshooting,
analysis, and for this chapter, a perfect tool to monitor traffic from potential targets
with the goal of capturing session tokens. Wireshark uses a GTK+ widget toolkit to
implement its user interface and pcap to capture packets. It operates very similarly to
a tcpdump command; however, acting as a graphical frontend with integrated sorting
and filtering options.
(HAY MAS, CON GRAFICOS EN)
Joseph Muniz, Aamir Lakhani - Web Penetration Testing with Kali Linux-Packt Publishing (2013)

    \section{Ettrcap}
    Ettercap is a free and open source comprehensive suite for man-in-the-middle-based
attacks.
Ettercap can be used for computer network protocol analysis and security auditing,
featuring sniffing live connections, content filtering, and support for active and passive
dissection of multiple protocols. Ettercap works by putting the attacker's network
interface into promiscuous mode and ARP for poisoning the victim machines.



\subsection{Realización del ataque}

\begin{center}
    \begin{figure}   
       \begin{center}
         
          \includegraphics[width=9cm,height=9cm]{red.png}
       \end{center}
       \caption{Escenario montado}
       \label{escMontado}
    \end{figure}
 \end{center}

La idea principal de esta sección es demostrar que, encontrándose en una red interna
y con ciertas herramientas, es posible realizar un ataque 
sin necesidad de conocer a fondo la implementación de la misma ni de tener mayores
privilegios.

El escenario montado (figuras \ref{escMontado} y \ref{formMontado}) consiste en crear una página \emph{web} con un 
formulario donde se debe completar con usuario y contraseña, y un 
submit el cual envía esta información desde el cliente hasta el 
servidor \emph{web}. El envío de este formulario contiene la información confidencial,
 por lo que en un escenario seguro ningún
intermediario podría obtener estos datos. Dado que este tráfico
circula utilizando el protocolo \emph{HTTP}, mostraremos como nos podemos hacer de las credenciales
ingresadas por el usuario.

\begin{center}
   \begin{figure}   
      \begin{center}
         \includegraphics[width=10.5cm,height=6cm]{form.png}
      \end{center}
      \caption{Formulario montado}
      \label{formMontado}
   \end{figure}
\end{center}

\subsection{Preparando Ettercap para el ataque ARP Poisoning}

Lo primero que debemos hacer, en la lista de aplicaciones, es buscar el apartado 
\emph{Sniffing} y \emph{Spoofing}, ya que es allí donde encontraremos las herramientas necesarias
 para llevar a cabo este ataque. A continuación, abriremos Ettercap.



El siguiente paso es seleccionar la tarjeta de red con la que vamos a trabajar. Para 
ello, en el menú superior de Ettercap seleccionaremos \textsc{Sniff} $>$ \textsc{Unified Sniffing} y, 
cuando nos lo pregunte, seleccionaremos nuestra tarjeta de red (por ejemplo, en 
nuestro caso, \textsc{eth0}).

Luego buscaremos todos los hosts conectados a nuestra red local. Para ello, 
seleccionaremos \textsc{Hosts} del menú de la parte superior y seleccionaremos la primera 
opción, Hosts List (Figura \ref{ettercap}).

Allí veremos todos los hosts o dispositivos conectados a nuestra red. 
Sin embargo, en caso de que no estén todos, podemos realizar una exploración 
completa de la red simplemente abriendo el menú \textsc{Hosts} y seleccionando la opción 
\textsc{Scan for hosts}. Tras unos segundos, la lista de antes se debería actualizar 
mostrando todos los dispositivos, con sus respectivas \emph{IPs} y \emph{MACs}, conectados 
a nuestra red.



\subsection{Nuestro Ettercap ya está listo. Ya podemos empezar con el ataque ARP Poisoning}

En caso de querer realizar un ataque dirigido contra un solo host, por ejemplo, 
suplantar la identidad de la puerta de enlace para monitorear las conexiones 
de la víctima, antes de empezar con el 
ataque debemos establecer los dos objetivos.

Para ello, debajo de la lista de hosts podemos ver tres botones, aunque nosotros 
prestaremos atención a los dos últimos:

    Target 1 – Seleccionamos la \emph{IP} del dispositivo a monitorear, en este caso, 
    la computadora de la víctima, y pulsamos sobre dicho botón.

    Target 2 – Pulsamos la \emph{IP} que queremos suplantar, en este caso, la de la 
    puerta de enlace y la del servidor \emph{web}.

    \begin{center}
        \begin{figure}   
           \begin{center}
              \includegraphics[width=17cm,height=9.5cm]{ataque-7-deta.png}
           \end{center}
           \caption{Ettercap}
           \label{ettercap}
        \end{figure}
     \end{center}

Debemos elegir el menú \textsc{MITM} de la parte superior y, en él, 
escoger la opción \textsc{ARP Poisoning}. Nos aparecerá una pequeña ventana de configuración, 
en la cual debemos asegurarnos de marcar \textsc{Sniff Remote Connections}.
Pulsamos sobre \textsc{ok} y el ataque dará lugar. Ahora ya podemos tener el control 
sobre el host que hayamos establecido como \emph{Target 1}. Lo siguiente que debemos 
hacer es, ejecutar Wireshark para capturar todos los paquetes de 
red y analizarlos en busca de información interesante.

\begin{center}
   \begin{figure}   
      \begin{center}
         \includegraphics[width=17cm,height=7cm]{paquetes.png}
      \end{center}
      \caption{Paquetes capturados}
      \label{pacCap}
   \end{figure}
\end{center}

Como se puede ver en la figura \ref{pacCap}, Wireshark nos permite filtrar el tráfico, y con el 
simple hecho de decirle que queremos mostrar los requerimientos GET
pudimos dar con el paquete que queríamos, en el request podemos ver
el usuario \texttt{salvador@proyectofinal.test} y la contraseña \texttt{PASSWORD SEGURO}
