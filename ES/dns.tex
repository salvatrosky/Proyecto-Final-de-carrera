\section{Servidor dns con Docker}

Se eligio el servidor dns coredns ya que es amigable con Docker, sucede que, por cada versión del 
programa, se generan las imágenes de docker correspondientes. Estas imágenes son publicas y oficiales, 
lo que da confiabilidad y seguridad extra a la hora de utilizarlas. La configuración esta formada por 
los siguientes componentes.

Los archivos de configuración de coredns son el corefile y los sitios que nosotros deseemos, en nuestro 
caso salvadorcatalfamo.

El Corefile es el archivo de configuración de CoreDNS.Este define:
\begin{itemize}
    \item Qué servidores escuchan en qué puertos y qué protocolo.
    \item Para qué zona tiene autoridad cada servidor.
    \item Qué plugins (complementos) se cargan en un servidor.
\end{itemize}

\noindent El formato es el siguiente
\begin{verbatim}
    ZONE: [PORT] {
        [PLUGIN] ...
    }
\end{verbatim}

\noindent ZONE: define la zona de este servidor. El puerto por defecto es el 53º bien el valor que se le indique 
con el flag -dns.port.

\noindent PLUGIN: define los complementos que queremos cargar. Cada plugin puede tener varias propiedades, por 
lo que también podrían tener argumentos de entrada.

\noindent Nuestro archivo de configuración es el siguiente:
\begin{verbatim}
    .:53 {
        forward . 8.8.8.8 9.9.9.9
        log
    }

    salvadorcatalfamo.intra:53 {
        file /etc/coredns/salvadorcatalfamo
        log
        reload 10s
    }    
\end{verbatim}

A grandes rasgos, lo que indica esta configuración es que va a existir una zona 
“salvadorcatalfamo.intra”, que estará definida por el archivo que se encuentra en 
“/etc/coredns/salvadorcatalfamo”. Por otro lado, el tráfico restante será fordwardeado a 
servidores dns externos (8.8.8.8 y 9.9.9.9). Además se establecieron algunos plugins de logeo y 
de refresco de configuración.

Nuestro archivo “/etc/coredns/salvadorcatalfamo” contiene la siguiente información.
\begin{verbatim}
salvadorcatalfamo.intra.        IN  SOA ns1.salvadorcatalfamo.intra. salvador.cata95.gmail.com. 2021031502 7200 3600 1209600 3600
page1.salvadorcatalfamo.intra.  IN  A   192.168.0.201
page2.salvadorcatalfamo.intra.  IN  A   192.168.0.202
page3.salvadorcatalfamo.intra.  IN  A   192.168.0.203    
\end{verbatim}

Esto, en principio es suficiente para nuestro sitio interno, y contiene las direcciones ip de los 
servidores webs y entidades certificantes.

Por el lado de Docker, se utilizó un archivo Docker.compose.yml, y un Dockerfile. El fichero 
Docker-compose sirve para …. En nuestro caso, se definió de la siguiente manera

\begin{verbatim}
version: '3.1'
services:
    coredns:
        build: .
        container_name: coredns
        restart: always
        expose:
            - '53'
            - '53/udp'
        ports:
            - '53:53'
            - '53:53/udp'
        volumes:
            - './config:/etc/coredns'    
\end{verbatim}

Por otro lado, el fichero dockerfile esta compuesto por las siguientes líneas
\begin{verbatim}
FROM coredns/coredns:1.7.0

ENTRYPOINT ["/coredns"]
CMD ["-conf", "/etc/coredns/Corefile"]    
\end{verbatim}

En conjunto, establecen la imagen de coredns que se utilizará, los archivos de configuración y
los puertos que se expondrán, entre otras configuraciones.
