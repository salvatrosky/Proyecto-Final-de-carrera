\subsubsection*{Servidor DNS con Docker}

Para montar esta propuesta de solución, se debió crear un servidor DNS, que lo que hace
es básicamente mapear direcciones IP a nombres de dominio. Por ejemplo: nuestra página que 
se encuentra en la dirección 192.168.0.202 se pasará a llamar pagina1.salvadorcatalfamo.intra, 
donde salvadorcatalfamo.intra abarca toda nuestra organización y las páginas que se encuentren
dentro de ella. Otra razón por la cual se debe 
crear un servidor DNS, es que los certificados no se pueden otorgar a direcciones IP, por lo 
cual es un requisito obligatorio tener las paginas web direccionadas con nombres de dominio.

Se eligió el servidor DNS CoreDNS ya que es amigable con Docker; sucede que, por cada versión del 
programa, se generan las imágenes de Docker correspondientes. Estas imágenes son públicas y oficiales, 
lo que da confiabilidad y seguridad extra a la hora de utilizarlas. La configuración está formada por 
los siguientes componentes: los archivos de configuración de CoreDNS (corefile) y los sitios que 
nosotros deseemos, en nuestro caso pagina1.salvadorcatalfamo.intra

\noindent El Corefile es el archivo de configuración de CoreDNS. Este define:
\begin{itemize}
    \setlength\itemsep{-0.6em}
    \item Qué servidores escuchan en qué puertos y qué protocolo.
    \item Para qué zona tiene autoridad cada servidor.
    \item Qué \emph{plugins} (complementos) se cargan en un servidor.
\end{itemize}

\noindent El formato es el siguiente
\begin{verbatim}
    ZONE: [PORT] {
        [PLUGIN] ...
    }
\end{verbatim}

\noindent ZONE: define la zona de este servidor. El puerto por defecto es el 53, o bien el valor que se le indique 
con el flag -dns.port.

\noindent PLUGIN: define los complementos que queremos cargar. Cada plugin puede tener varias propiedades, por 
lo que también podrían tener argumentos de entrada.

\noindent Nuestro archivo de configuración es el siguiente:
\begin{verbatim}
    .:53 {
        forward . 8.8.8.8 9.9.9.9
        log
    }

    salvadorcatalfamo.intra:53 {
        file /etc/coredns/salvadorcatalfamo
        log
        reload 10s
    }    
\end{verbatim}

A grandes rasgos, lo que indica esta configuración es que va a existir una zona 
“salvadorcatalfamo.intra”, que estará definida por el archivo que se encuentra en 
“/etc/coredns/salvadorcatalfamo”. Por otro lado, el tráfico restante (dominios externos a nuestra red)
será fordwardeado a 
servidores DNS externos (8.8.8.8 y 9.9.9.9). Además se establecieron algunos plugins de logeo y 
de refresco de configuración.

Nuestro archivo “/etc/coredns/salvadorcatalfamo” contiene la siguiente información.
\begin{verbatim}
salvadorcatalfamo.intra.          IN  SOA ns1.salvadorcatalfamo.intra. ...
pagina1.salvadorcatalfamo.intra.  IN  A   192.168.0.124   
\end{verbatim}

Esto en principio es suficiente para nuestro sitio interno, y contiene las direcciones ip de los 
servidores web y entidades certificantes.

Por el lado de Docker, se utilizó un archivo Docker.compose.yml, y un Dockerfile. 
Docker-compose es una herramienta para definir y ejecutar aplicaciones Docker de 
varios contenedores. Compose usa un archivo YAML para configurar los servicios de 
la aplicación. Luego, con un solo comando, se crean e inician todos los servicios 
desde este archivo de configuración. En nuestro caso, se definió de la siguiente 
manera

\begin{verbatim}
version: `3.1'
services:
    coredns:
        build: .
        container_name: coredns
        restart: always
        expose:
            - `53'
            - `53/udp'
        ports:
            - `53:53'
            - `53:53/udp'
        volumes:
            - `./config:/etc/coredns'    
\end{verbatim}

\noindent Por otro lado, el fichero dockerfile esta compuesto por las siguientes líneas:
\begin{verbatim}
FROM coredns/coredns:1.7.0
ENTRYPOINT ["/coredns"]
CMD ["-conf", "/etc/coredns/Corefile"]    
\end{verbatim}

En conjunto, establecen la imagen de CoreDNS que se utilizará, los archivos de configuración y
los puertos que se expondrán, entre otras configuraciones.
