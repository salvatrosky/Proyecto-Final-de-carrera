

(AGREGAR GRAFICOS SI O SI)
Virtualization provides abstraction on top of the actual resources we want to virtualize.
 The level at which this abstraction is applied changes the way that different
  virtualization techniques look.At a higher level, there are two major
   virtualization techniques based on the level of abstraction.
   • Virtual machine (VM)-based
   • Container-based
   Apart from these two virtualizing techniques, there are other techniques, such as unikernels, which are lightweight single-purpose VMs. IBM is currently attempting to run unikernels as processes with projects like Nabla. In this book, we will mainly look at VM-based and container-based virtualizations only.
\section{Máquinas virtuales}
VM-Based  VirtualizationThe VM-based approach virtualizes the complete OS. The abstraction
 it presents to the VM are virtual devices like virtual disks, virtual CPUs, and virtual 
 NICs. In other words, we can state that this is virtualizing the complete ISA
  (instruction set architecture); as an example, the x86 ISA.With virtual machines,
   multiple OSes can share the same hardware resources, with virtualized representations 
   of each of the resources available to the VM. For example, the OS on the virtual
    machine (also called the guest) can continue to do I/O operations on a disk 
    (in this case, it’s a virtual disk), thinking that it’s the only OS running on the 
    physical hardware (also called the host), although in actuality, it is shared by 
    multiple virtual machines as well as by the host OS.VMs are very similar to other 
    processes in the host OS. VMs execute in a hardware-isolated virtual address space 
    and at a lower privilege level than the host OS. The primary difference between a
     process and a VM is the ABI (Application Binary Interface) exposed by the host
      to the VM. In the case of a process, the exposed ABI has constructs like network 
      sockets, FDs, and so on, whereas with a full-fledged OS virtualization, the ABI 
      will have a virtual disk, a virtual CPU, virtual network cards, and so on.

Hypervisors
A special piece of software is used to virtualize the OS, called the hypervisor. The 
hypervisor itself has two parts:
• Virtual Machine Monitor (VMM): Used for trapping and emulating the privileged 
instruction set (which only the kernel of the operating system can perform).
The VMM has to satisfy three properties (Popek and Goldberg, 1973):
    • Isolation :  
Should isolate guests (VMs) from each other.
    • Equivalency : Should behave the same, 
with or without virtualization. This means we run the majority (almost all) of the 
instructions on the physical hardware without any translation, and so on.
    • Performance : 
Should perform as good as it does without any virtualization. This again means that 
the overhead of running a VM is minimal.

Device  Model
The device model of the hypervisor handles the I/O virtualization again by trapping 
and emulating and then delivering interrupts back to the specific virtual machine.
The device model handles:
•Memory  Virtualization
•Shadow Page Table
•CPU  Virtualization
•IO  Virtualization

\section{Container-Based  Virtualization}
Container-Based  VirtualizationThis form of virtualization doesn’t abstract the 
hardware but uses techniques within the Linux kernel to isolate access paths for
 different resources. It carves out a logical boundary within the same operating 
 system. As an example, we get a separate root file system, a separate process tree, 
 a separate network subsystem, and so on.


The Linux kernel is made up of several components and functionalities; the ones related to
containers are as follows:
Control groups (cgroups)
Namespaces
Security-Enhanced Linux (SELinux)
Cgroups
The cgroup functionality allows for limiting and prioritizing resources, such as CPUs,
RAM, the network, the filesystem, and so on. The main goal is to not exceed the
resources—to avoid wasting resources that might be needed for other processes.
Namespaces
The namespace functionality allows for partitioning of kernel resources, such that one set of
processes sees one set of resources, while another set of processes sees a different set of
resources. The feature works by having the same namespace for these resources in the
various sets of processes, but having those names refer to distinct resources. Examples of
resource names that can exist in multiple spaces (so that the named resources will be
partitioned) are process IDs, hostnames, user IDs, filenames, and some names associated
with network access and inter-process communication.
When a Linux system boots; that is, only one namespace is created. Processes and resources
will join the same namespace, until a different namespace is created, resources assigned to
it, and processes join it.
SELinux
SELinux is a module of the Linux kernel that provides a mechanism to enforce the security
of the system, with specific policies.
Basically, SELinux can limit programs from accessing files and network resources. The idea
is to limit the privileges of programs and daemons to a minimum, so that it can limit the
risk of system halt.
The preceding functionalities have been around for many years. Namespaces were first
released in 2002, and cgroups in 2005, by Google (cgroups were first named process
containers, and then cgroups). For example, SunSolaris 5.10, released at the beginning of
2005, provided support for Solaris containers.
Nowadays, Linux containers are the new buzzword, and some people think they are a
new means of virtualization.
Containerization uses resources directly, and does not need an emulator at all; the fewer
resources, the more efficiency. Different applications can run on the same host: isolated at
the kernel level and isolated by namespaces and cgroups. The kernel (that is, the OS) is
shared by all containers, as shown in the following diagram:
Containers
When we talk about containers, we are indirectly referring to two main concepts—a
container image and a running container image.
A container image is the definition of the container, wherein all software stacks are
installed as additional layers, as depicted by the following diagram:
A container image is typically made up of multiple layers.
The first layer is given by the base image, which provides the OS core functionalities, with
all of the tools needed to get started. Teams often work by building their own layers on
these base images. Users can also build on more advanced application images, which not
only have an OS, but which also include language runtimes, debugging tools, and libraries,
as shown in the following diagram:
Base images are built from the same utilities and libraries that are included in an OS. A
good base image provides a secure and stable platform on which to build applications. Red
Hat provides base images for Red Hat Enterprise Linux. These images can be used like a
normal OS. Users can customize them for their applications as necessary, installing
packages and enabling services to start up just like a normal Red Hat Enterprise Linux
Server.
Containers provide isolation by taking advantage of kernel technologies, like cgroups,
kernel namespaces, and SELinux, which have been battle-tested and used for years at
Google and the US Department of Defense, in order to provide application isolation.
Since containers use a shared kernel and container host, they reduce the amount of
resources required for the container itself, and are more lightweight when compared to
VMs. Therefore, containers provide an unmatched agility that is not feasible with VMs; for
example, it only takes a few seconds to start a new container. Furthermore, containers
support a more flexible model when it comes to CPU utilization and memory resources,
and allow for resource burst modes, so that applications can consume more resources when
required, within the defined boundaries.

\section{Docker}
\subsection{Introduction}
Docker is an open-source engine that automates the deployment of
applications into containers. It was written by the team at Docker, Inc
(formerly dotCloud Inc, an early player in the Platform-as-a-Service
(PAAS) market), and released by them under the Apache 2.0 license.
Docker adds an application deployment
engine on top of a virtualized container execution environment. It is
designed to provide a lightweight and fast environment in which to run your
code as well as an efficient workflow to get that code from your laptop to
your test environment and then into production. Docker is incredibly
simple. Indeed, you can get started with Docker on a minimal host running
nothing but a compatible Linux kernel and a Docker binary. 

With Docker, Developers care about their applications running inside
containers, and Operations cares about managing the containers. Docker is
designed to enhance consistency by ensuring the environment in which
your developers write code matches the environments into which your
applications are deployed. This reduces the risk of “worked in dev, now an
ops problem.

\subsection{Docker client and server}
Docker is a client-server application. The Docker client talks to the Docker
server or daemon, which, in turn, does all the work. You’ll also sometimes
see the Docker daemon called the Docker Engine. Docker ships with a
command line client binary, docker, as well as a full RESTful API to
interact with the daemon: dockerd. You can run the Docker daemon and
client on the same host or connect your local Docker client to a remote
daemon running on another host. You can see Docker’s architecture
depicted here:
\subsection{Docker images}
Images are the building blocks of the Docker world. You launch your
containers from images. Images are the “build” part of Docker’s life cycle.
They have a layered format, using Union file systems, that are built step-bystep using a series of instructions. For example:
Add a file.
Run a command.
Open a port.
You can consider images to be the “source code” for your containers. They
are highly portable and can be shared, stored, and updated. In the book,
we’ll learn how to use existing images as well as build our own images.
\subsection{Registries}
Docker stores the images you build in registries. There are two types of
registries: public and private. Docker, Inc., operates the public registry for
images, called the Docker Hub. You can create an account on the Docker
Hub and use it to share and store your own images.
The Docker Hub also contains, at last count, over 10,000 images that other
people have built and shared. Want a Docker image for an Nginx web
server, the Asterisk open source PABX system, or a MySQL database? All
of these are available, along with a whole lot more.
You can also store images that you want to keep private on the Docker Hub.
These images might include source code or other proprietary information
you want to keep secure or only share with other members of your team or
organization.
You can also run your own private registry, and we’ll show you how to do
that in Chapter 4. This allows you to store images behind your firewall,
which may be a requirement for some organizations.
\subsection{Containers}
Docker helps you build and deploy containers inside of which you can
package your applications and services. As we’ve just learned, containers
are launched from images and can contain one or more running processes.
You can think about images as the building or packing aspect of Docker and
the containers as the running or execution aspect of Docker.
A Docker container is:
An image format.
A set of standard operations.
An execution environment.
Docker borrows the concept of the standard shipping container, used to
transport goods globally, as a model for its containers. But instead of
shipping goods, Docker containers ship software.
Each container contains a software image – its ‘cargo’ – and, like its
physical counterpart, allows a set of operations to be performed. For
example, it can be created, started, stopped, restarted, and destroyed.
Like a shipping container, Docker doesn’t care about the contents of the
container when performing these actions; for example, whether a container
is a web server, a database, or an application server. Each container is
loaded the same as any other container.
Docker also doesn’t care where you ship your container: you can build on
your laptop, upload to a registry, then download to a physical or virtual
server, test, deploy to a cluster of a dozen Amazon EC2 hosts, and run. Like
a normal shipping container, it is interchangeable, stackable, portable, and
as generic as possible.
With Docker, we can quickly build an application server, a message bus, a
utility appliance, a CI test bed for an application, or one of a thousand other
possible applications, services, and tools. It can build local, self-contained
test environments or replicate complex application stacks for production or
development purposes. The possible use cases are endless.

