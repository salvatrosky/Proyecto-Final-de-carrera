Como mencionamos al principio del proyecto, en las redes internas, además de
los datos que circulan producto de la navegación web, también circula
otro tipo de tráfico, como correos electrónicos, archivos completos de 
configuración o bien multimedia, como así también las consultas DNS que 
se realizar a estos servidores dedicados. En diversas ocasiones, este 
tipo de tráfico va sin cifrar, y, hablando particularmente del correo electrónico,
es más común de lo que parece, ya que tampoco fue diseñado pensando en la 
seguridad. El estudio de todo el circuito que conlleva desde el envío de 
un correo, hasta la recepción del mismo quedará pendiente para trabajos futuros, 
ya que puede llegar a tomar tanto como este proyecto mismo.

Con respecto a las propuestas de solución, una de ellas fue la creación de una entidad 
certificante
interna. La gestión de la creación de los certificados fue completamente 
manual, esto, es fácilmente automatizable, obviamente estableciendo estándares 
de seguridad, tales como encriptación, autenticación y logeo de los 
certificados generados.