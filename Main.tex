% =================================================================== %
%   Tesis (template)
% =================================================================== %
%
%\documentclass[12pt,twoside,draft]{book}
\documentclass[12pt,twoside]{book}

% la opci�n draft es bueno activarla mientras uno est� escribiendo y despu�s hay que sacarla para la versi�n final 

% packages LaTeX adicionales
\usepackage[spanish]{babel}
\usepackage{amsmath}
\usepackage{amstext}
\usepackage{mathtools} 
\usepackage{amssymb}
\usepackage{longtable}
\usepackage{makeidx}
%\usepackage[dvips]{graphicx}
\usepackage{graphicx}
\graphicspath{ {./images/} }

\input{setbmp}

% idioma espa�ol
\selectlanguage{spanish}

% m�rgenes
\setlength{\textwidth}{150mm}
\setlength{\oddsidemargin}{9.6mm}
\setlength{\evensidemargin}{-0.6mm}
\setlength{\textheight}{217mm}
\setlength{\topmargin}{-5.4mm}
\setlength{\headheight}{10mm}
\setlength{\headsep}{10mm}

% doble espaciado entre lineas
\def\baselinestretch{1.30}

% profundidad en la tabla de contenidos
\setcounter{tocdepth}{2}

% hasta donde pone n�meros en el seccionado (ej 1.3.4.1)
\setcounter{secnumdepth}{4}

% definiciones de comandos especiales
\input{commands}
\makeindex
% =================================================================== %

\begin{document}

\nocite{*}

\frontmatter\pagestyle{empty}

\begin{titlepage}
\begin{titlepage}

\begin{center}
\includegraphics[width=3cm,height=3cm]{uni.png}
\end{center}

\begin{center}

\textbf{\LARGE Universidad Nacional del Sur}\\

\vspace{2cm}

\textsc{\LARGE Proyecto Final de Carrera}\\ \vspace{.1cm}
\textsc{\LARGE Ingenier\'ia en Computaci\'on}\\


\vspace{4cm}

\emph{\LARGE Seguridad en redes LAN: utilizando docker para mejorar la infraestructura.}\\

\vspace{2.5cm}

{\Large Salvador Catalfamo}\\

\vspace{2.5cm}

{\sc\Large Bah\'{\i}a Blanca -- Argentina}\\
\vspace*{.1cm} {\Large 2021}

\end{center}
\end{titlepage}

\end{titlepage}

\begin{titlepage}
\begin{titlepage}

\begin{center}
\includegraphics[width=3cm,height=3cm]{uni.png}
\end{center}

\begin{center}

\textbf{\LARGE Universidad Nacional del Sur}\\

\vspace{2cm}

\textsc{\LARGE Proyecto Final de Carrera}\\ \vspace{.1cm}
\textsc{\LARGE Ingenier\'ia en Computaci\'on}\\


\vspace{4cm}

\emph{\LARGE Seguridad en redes LAN: utilizando docker para mejorar la infraestructura.}\\

\vspace{2.5cm}

{\Large Salvador Catalfamo}\\

\vspace{2.5cm}

{\sc\Large Bah\'{\i}a Blanca -- Argentina}\\
\vspace*{.1cm} {\Large 2021}

\end{center}
\end{titlepage}

\end{titlepage}


% =================================================================== %

% \setcounter{page}{1}

% resumen en castellano
\thispagestyle{empty}
\chapter*{Resumen}

% Aca va el resumen en castellano

\bigskip
A lo largo de la carrera, y, particulamente, en una de mis materias preferidas, 
Seguridad en Sistemas, nos han explicado la importancia la protección de los 
datos, como asi también la de los canales de comunicación. Mientras se desarrolla 
una red segura, se debe considerar la confidencialidad, integridad y disponibilidad 
(CIA). 

También, en nuestra corta experiencia hemos visto como las organizaciones 
abordan los temas de seguridad en sus sistemas informáticos. Mayormente, se 
concentran en los equipos que están expuestos a la red pública, dejando de 
lado los que se encuentran aislados de la misma. Erróneamente, muchas veces 
se piensa que es suficiente, sin embargo, puede traer graves inconvenientes. 
Es por eso que realizaremos un estudio teórico/práctico sobre las consecuencias 
de la navegación en redes internas sin ningún tipo de cifrado de datos ni 
certificaciones.

\bigskip
\noindent \textsc{Palabras Clave:} \par

Seguridad e Infraestructura\par
Docker \par
Linux \par
Kali \par
Máquinas Virtuales  \par




\tableofcontents

%\listoffigures

% =================================================================== %
\mainmatter\pagestyle{headings}


\chapter{Introducci\'on} \pagenumbering{arabic}
    \label{capIntro}

%-*- ES -*-
%----------------------------------------------------------------------
% capitulo 1: Introduccion
%----------------------------------------------------------------------
% Estructura del capitulo:
%
%----------------------------------------------------------------------


% se pone una introducci�n al tema en general

\section{Objetivos}

Hence, not only the nodes but the communication channel should also be secured.
While developing a secure network, Confidentiality, Integrity, Availability (CIA)
needs to be considered

\begin{itemize}
 \item item 1
 \item item 2
\end{itemize}


\begin{verbatim}
 Este es un bien ambiente para 
poner codigo
\end{verbatim}



%En \cite{Davis:1989} se \footnote{Esta es una nota al pie} ve...

\textit{casa} \\
\emph{casa} \\
\texttt{casa} \\
\textsf{casa} \\
\textbf{casa} \\

\begin{center}
\begin{figure}

\begin{center}
    \includegraphics[width=3cm,height=3cm]{uni.png}
\end{center}

\caption{Esta es la figura del escudo de la uns}
\end{figure}
\end{center}

\section{Plan de tesis y principales contribuciones}\label{secContribuciones}
\begin{itemize}
    \item concientizar la implementacion de medidas de seguridad en redes internas
    \item demostrar como es posible realizar realizar acciones que perjudiquen a una organizacion
    \item implementar una mejora en la navegacion en una red interna utilizando docker
   \end{itemize}
   no se si las contribuciones se refiere a la principal fuente de consulta
\section{Trabajos previos relacionados}
% rese�ar los art�culos hechos en el contexto del trabajo de tesis si los hay



% =================================================================== %





\chapter{(Ajustar) ¿Qué circula por una red interna? } 
    \label{capImp}

\section{Introducción}
Network Technology is the key technology for a wide variety of applications such as
email, file transfer, web browsing, online transactions, form fill up for various governmental or private activities, cab booking, etc. However, there is a significant lack of
easy implementation of security methods for these applications. 
\section{Protocolos asociados a la web}
    \subsection{¿Que es el protocolo HTTP?}
The Hypertext Transfer Protocol (HTTP) is a stateless application-level 
request/response protocol that uses extensible semantics and
   self-descriptive message payloads for flexible interaction with
   network-based hypertext information systems.
   HTTP is a generic interface protocol for information systems.  It is
   designed to hide the details of how a service is implemented by
   presenting a uniform interface to clients that is independent of the
   types of resources provided.  Likewise, servers do not need to be
   aware of each client's purpose: an HTTP request can be considered in
   isolation rather than being associated with a specific type of client
   or a predetermined sequence of application steps.  The result is a
   protocol that can be used effectively in many different contexts and
   for which implementations can evolve independently over time.
  
   One consequence of this flexibility is that the protocol cannot be
   defined in terms of what occurs behind the interface.  Instead, we
   are limited to defining the syntax of communication, the intent of
   received communication, and the expected behavior of recipients.  If
   the communication is considered in isolation, then successful actions
   ought to be reflected in corresponding changes to the observable
   interface provided by servers.  However, since multiple clients might
   act in parallel and perhaps at cross-purposes, we cannot require that
   such changes be observable beyond the scope of a single response.
   
\subsection{Arquitectura}
HTTP was created for the World Wide Web (WWW) architecture and has
evolved over time to support the scalability needs of a worldwide
hypertext system.  Much of that architecture is reflected in the
terminology and syntax productions used to define HTTP.

\subsubsection*{Client/Server Messaging}

HTTP is a stateless request/response protocol that operates by
exchanging messages across a reliable transport- or
session-layer "connection".  An HTTP "client" is a
program that establishes a connection to a server for the purpose of
sending one or more HTTP requests.  An HTTP "server" is a program
that accepts connections in order to service HTTP requests by sending
HTTP responses.
Most HTTP communication consists of a retrieval request (GET) for a
representation of some resource identified by a URI.  In the simplest
case, this might be accomplished via a single bidirectional
connection (===) between the user agent (UA) and the origin
server (O).



\begin{center}
   \begin{figure}   
      \begin{center}
         \includegraphics{2.1.png}
      \end{center}
      \caption{Simple comunication method}
   \end{figure}
\end{center}


A client sends an HTTP request to a server in the form of a request
message, beginning with a request-line that includes a method, URI,
and protocol version, followed by header fields
containing request modifiers, client information, and representation
metadata, an empty line to indicate the end of the
header section, and finally a message body containing the payload
body (if any, Section 3.3).
A server responds to a client's request by sending one or more HTTP
   response messages, each beginning with a status line that includes
   the protocol version, a success or error code, and textual reason
   phrase (Section 3.1.2), possibly followed by header fields containing
   server information, resource metadata, and representation metadata
   (Section 3.2), an empty line to indicate the end of the header
   section, and finally a message body containing the payload body (if
   any, Section 3.3).
\subsubsection*{Ejemplo}
The following example illustrates a typical message exchange for a
GET request (Section 4.3.1 of [RFC7231]) on the URI
"http://www.example.com/hello.txt":

Client request:

  GET /hello.txt HTTP/1.1
  User-Agent: curl/7.16.3 libcurl/7.16.3 OpenSSL/0.9.7l zlib/1.2.3
  Host: www.example.com
  Accept-Language: en, mi


Server response:

  HTTP/1.1 200 OK
  Date: Mon, 27 Jul 2009 12:28:53 GMT
  Server: Apache
  Last-Modified: Wed, 22 Jul 2009 19:15:56 GMT
  ETag: "34aa387-d-1568eb00"
  Accept-Ranges: bytes
  Content-Length: 51
  Vary: Accept-Encoding
  Content-Type: text/plain

  Hello World! My payload includes a trailing CRLF.

  HTTP is defined as a stateless protocol, meaning that each request
   message can be understood in isolation.  Many implementations depend
   on HTTP's stateless design in order to reuse proxied connections or
   dynamically load balance requests across multiple servers.  Hence, a
   server MUST NOT assume that two requests on the same connection are
   from the same user agent unless the connection is secured and
   specific to that agent.  Some non-standard HTTP extensions (e.g.,
   [RFC4559]) have been known to violate this requirement, resulting in
   security and interoperability problems.
\subsection{Formato del mensaje (Mejorar intro)}

All HTTP/1.1 messages consist of a start-line followed by a sequence
of octets in a format similar to the Internet Message Format
[RFC5322]: zero or more header fields (collectively referred to as
the "headers" or the "header section"), an empty line indicating the
end of the header section, and an optional message body.
The normal procedure for parsing an HTTP message is to read the
   start-line into a structure, read each header field into a hash table
   by field name until the empty line, and then use the parsed data to
   determine if a message body is expected.  If a message body has been
   indicated, then it is read as a stream until an amount of octets
   equal to the message body length is read or the connection is closed.

 
\subsubsection*{Start Line}
An HTTP message can be either a request from client to server or a
   response from server to client.  Syntactically, the two types of
   message differ only in the start-line, which is either a request-line
   (for requests) or a status-line (for responses), and in the algorithm
   for determining the length of the message body .
   In theory, a client could receive requests and a server could receive
   responses, distinguishing them by their different start-line formats,
   but, in practice, servers are implemented to only expect a request (a
   response is interpreted as an unknown or invalid request method) and
   clients are implemented to only expect a response.

     
   \paragraph*{Request Line}
A request-line begins with a method token, followed by a single space
   (SP), the request-target, another single space (SP), the protocol
   version, and ends with CRLF.

     (podria ir un grafico)

   The method token indicates the request method to be performed on the
   target resource.  The request method is case-sensitive.


   The request-target identifies the target resource upon which to apply
   the request



  
\paragraph*{Status Line}
The first line of a response message is the status-line, consisting
   of the protocol version, a space (SP), the status code, another
   space, a possibly empty textual phrase describing the status code,
   and ending with CRLF.

   (podria ir un grafico)

   The status-code element is a 3-digit integer code describing the
   result of the server's attempt to understand and satisfy the client's
   corresponding request.  The rest of the response message is to be
   interpreted in light of the semantics defined for that status code.
  


\subsubsection* {Header Fields}
Each header field consists of a case-insensitive field name followed
by a colon (":"), optional leading whitespace, the field value, and
optional trailing whitespace.

(podria ir un grafico o con un formato mejor)
header-field   = field-name ":" OWS field-value OWS

The field-name token labels the corresponding field-value as having
the semantics defined by that header field.  

\paragraph*{Field Extensibility}
Header fields are fully extensible: there is no limit on the
introduction of new field names, each presumably defining new
semantics, nor on the number of header fields used in a given
message.  Existing fields are defined in each part of this
specification and in many other specifications outside this document
set.

New header fields can be defined such that, when they are understood
by a recipient, they might override or enhance the interpretation of
previously defined header fields, define preconditions on request
evaluation, or refine the meaning of responses.

\paragraph*{Field Order}
The order in which header fields with differing field names are
   received is not significant.  However, it is good practice to send
   header fields that contain control data first, such as Host on
   requests and Date on responses, so that implementations can decide
   when not to handle a message as early as possible. 


\paragraph*{Whitespace}
This specification uses three rules to denote the use of linear
whitespace: OWS (optional whitespace), RWS (required whitespace), and
BWS ("bad" whitespace).

The OWS rule is used where zero or more linear whitespace octets
might appear.  
The RWS rule is used when at least one linear whitespace octet is
required to separate field tokens.  

The BWS rule is used where the grammar allows optional whitespace
only for historical reasons. 

\paragraph*{Field Parsing}
Messages are parsed using a generic algorithm, independent of the
individual header field names.  The contents within a given field
value are not parsed until a later stage of message interpretation
(usually after the message's entire header section has been
processed).  Consequently, this specification does not use ABNF rules
to define each "Field-Name: Field Value" pair, as was done in
previous editions.  Instead, this specification uses ABNF rules that
are named according to each registered field name, wherein the rule
defines the valid grammar for that field's corresponding field values
(i.e., after the field-value has been extracted from the header
section by a generic field parser).

No whitespace is allowed between the header field-name and colon.  In
the past, differences in the handling of such whitespace have led to
security vulnerabilities in request routing and response handling. 


\paragraph*{Field Limits }
HTTP does not place a predefined limit on the length of each header
field or on the length of the header section as a whole.  Various 
ad hoc limitations on individual header
field length are found in practice, often depending on the specific
field semantics.


\paragraph*{Field Value Components}
(no entiendo, tal vez se puede sacar)
Most HTTP header field values are defined using common syntax
components (token, quoted-string, and comment) separated by
whitespace or specific delimiting characters.  Delimiters are chosen
from the set of US-ASCII visual characters not allowed in a token
%(DQUOTE and "(),/:;<=>?@[\]{}").

  token          = 1*tchar

%  tchar          = "!" / "#" / "$" / "%" / "&" / "'" / "*"
%                 / "+" / "-" / "." / "^" / "_" / "`" / "|" / "~"
                 / DIGIT / ALPHA
                 ; any VCHAR, except delimiters

A string of text is parsed as a single value if it is quoted using
double-quote marks.

  quoted-string  = DQUOTE *( qdtext / quoted-pair ) DQUOTE
  qdtext         = HTAB / SP /%x21 / %x23-5B / %x5D-7E / obs-text
  obs-text       = %x80-FF

Comments can be included in some HTTP header fields by surrounding
the comment text with parentheses.  Comments are only allowed in
fields containing "comment" as part of their field value definition.

  comment        = "(" *( ctext / quoted-pair / comment ) ")"
  ctext          = HTAB / SP / %x21-27 / %x2A-5B / %x5D-7E / obs-text

The backslash octet ("\") can be used as a single-octet quoting
mechanism within quoted-string and comment constructs.  Recipients
that process the value of a quoted-string MUST handle a quoted-pair
as if it were replaced by the octet following the backslash.

  quoted-pair    = "\" ( HTAB / SP / VCHAR / obs-text )

A sender SHOULD NOT generate a quoted-pair in a quoted-string except
where necessary to quote DQUOTE and backslash octets occurring within
that string.  A sender SHOULD NOT generate a quoted-pair in a comment
except where necessary to quote parentheses ["(" and ")"] and
backslash octets occurring within that comment.
\subsubsection*{Message Body }
The message body (if any) of an HTTP message is used to carry the
payload body of that request or response.  The message body is
identical to the payload body unless a transfer coding has been
applied.


The rules for when a message body is allowed in a message differ for
requests and responses.

The presence of a message body in a request is signaled by a
Content-Length or Transfer-Encoding header field.  Request message
framing is independent of method semantics, even if the method does
not define any use for a message body.

The presence of a message body in a response depends on both the
request method to which it is responding and the response status code.  
Responses to the HEAD request method 
never include a message body because the associated
response header fields 
, if present, indicate only what their values would have been if
the request method had been GET (Section 4.3.1 of [RFC7231]). 2xx
(Successful) responses to a CONNECT request method (Section 4.3.6 of
[RFC7231]) switch to tunnel mode instead of having a message body.
All 1xx (Informational), 204 (No Content), and 304 (Not Modified)
responses do not include a message body.  All other responses do
include a message body, although the body might be of zero length.

\paragraph*{Transfer-Encoding }

The Transfer-Encoding header field lists the transfer coding names
corresponding to the sequence of transfer codings that have been (or
will be) applied to the payload body in order to form the message
body.  Transfer codings are defined in Section 4.



Transfer-Encoding is analogous to the Content-Transfer-Encoding field
of MIME, which was designed to enable safe transport of binary data
over a 7-bit transport service .  However, safe
transport has a different focus for an 8bit-clean transfer protocol.
In HTTP's case, Transfer-Encoding is primarily intended to accurately
delimit a dynamically generated payload and to distinguish payload
encodings that are only applied for transport efficiency or security
from those that are characteristics of the selected resource.
. 

For example,

  Transfer-Encoding: gzip, chunked

indicates that the payload body has been compressed using the gzip
coding and then chunked using the chunked coding while forming the
message body.

Unlike Content-Encoding ,
Transfer-Encoding is a property of the message, not of the
representation, and any recipient along the request/response chain
MAY decode the received transfer coding(s) or apply additional
transfer coding(s) to the message body, assuming that corresponding
changes are made to the Transfer-Encoding field-value.  Additional
information about the encoding parameters can be provided by other
header fields not defined by this specification.



Transfer-Encoding was added in HTTP/1.1.  It is generally assumed
that implementations advertising only HTTP/1.0 support will not
understand how to process a transfer-encoded payload. 

\paragraph*{Content-Length }
When a message does not have a Transfer-Encoding header field, a
   Content-Length header field can provide the anticipated size, as a
   decimal number of octets, for a potential payload body.  For messages
   that do include a payload body, the Content-Length field-value
   provides the framing information necessary for determining where the
   body (and message) ends.  For messages that do not include a payload
   body, the Content-Length indicates the size of the selected
   representation .

    

\paragraph*{Message Body Length }

(no entiendo, ver diferencia con content-length)
The length of a message body is determined by one of the following
(in order of precedence):

1.  Any response to a HEAD request and any response with a 1xx
    (Informational), 204 (No Content), or 304 (Not Modified) status
    code is always terminated by the first empty line after the
    header fields, regardless of the header fields present in the
    message, and thus cannot contain a message body.

2.  Any 2xx (Successful) response to a CONNECT request implies that
    the connection will become a tunnel immediately after the empty
    line that concludes the header fields.  A client MUST ignore any
    Content-Length or Transfer-Encoding header fields received in
    such a message.

3.  If a Transfer-Encoding header field is present and the chunked
    transfer coding (Section 4.1) is the final encoding, the message
    body length is determined by reading and decoding the chunked
    data until the transfer coding indicates the data is complete.

    If a Transfer-Encoding header field is present in a response and
    the chunked transfer coding is not the final encoding, the
    message body length is determined by reading the connection until
    it is closed by the server.  If a Transfer-Encoding header field
    is present in a request and the chunked transfer coding is not
    the final encoding, the message body length cannot be determined
    reliably; the server MUST respond with the 400 (Bad Request)
    status code and then close the connection.

    If a message is received with both a Transfer-Encoding and a
    Content-Length header field, the Transfer-Encoding overrides the
    Content-Length.  Such a message might indicate an attempt to
    perform request smuggling (Section 9.5) or response splitting
    (Section 9.4) and ought to be handled as an error.  A sender MUST
    remove the received Content-Length field prior to forwarding such
    a message downstream.

4.  If a message is received without Transfer-Encoding and with
    either multiple Content-Length header fields having differing
    field-values or a single Content-Length header field having an
    invalid value, then the message framing is invalid and the
    recipient MUST treat it as an unrecoverable error.  If this is a
    request message, the server MUST respond with a 400 (Bad Request)
    status code and then close the connection.  If this is a response
    message received by a proxy, the proxy MUST close the connection
    to the server, discard the received response, and send a 502 (Bad
    Gateway) response to the client.  If this is a response message
    received by a user agent, the user agent MUST close the
    connection to the server and discard the received response.

5.  If a valid Content-Length header field is present without
    Transfer-Encoding, its decimal value defines the expected message
    body length in octets.  If the sender closes the connection or
    the recipient times out before the indicated number of octets are
    received, the recipient MUST consider the message to be
    incomplete and close the connection.

6.  If this is a request message and none of the above are true, then
    the message body length is zero (no message body is present).

7.  Otherwise, this is a response message without a declared message
    body length, so the message body length is determined by the
    number of octets received prior to the server closing the
    connection.

Since there is no way to distinguish a successfully completed,
close-delimited message from a partially received message interrupted
by network failure, a server SHOULD generate encoding or
length-delimited messages whenever possible.  The close-delimiting
feature exists primarily for backwards compatibility with HTTP/1.0.

A server MAY reject a request that contains a message body but not a
Content-Length by responding with 411 (Length Required).

Unless a transfer coding other than chunked has been applied, a
client that sends a request containing a message body SHOULD use a
valid Content-Length header field if the message body length is known
in advance, rather than the chunked transfer coding, since some
existing services respond to chunked with a 411 (Length Required)
status code even though they understand the chunked transfer coding.
This is typically because such services are implemented via a gateway
that requires a content-length in advance of being called and the
server is unable or unwilling to buffer the entire request before
processing.

A user agent that sends a request containing a message body MUST send
a valid Content-Length header field if it does not know the server
will handle HTTP/1.1 (or later) requests; such knowledge can be in
the form of specific user configuration or by remembering the version
of a prior received response.

If the final response to the last request on a connection has been
completely received and there remains additional data to read, a user
agent MAY discard the remaining data or attempt to determine if that
data belongs as part of the prior response body, which might be the
case if the prior message's Content-Length value is incorrect.  A
client MUST NOT process, cache, or forward such extra data as a
separate response, since such behavior would be vulnerable to cache
poisoning.


\subsection{Métodos del protocolo HTTP}
4.3.1.  GET

   The GET method requests transfer of a current selected representation
   for the target resource.  GET is the primary mechanism of information
   retrieval and the focus of almost all performance optimizations.
   Hence, when people speak of retrieving some identifiable information
   via HTTP, they are generally referring to making a GET request.

   It is tempting to think of resource identifiers as remote file system
   pathnames and of representations as being a copy of the contents of
   such files.  In fact, that is how many resources are implemented 
   .  However, there are
   no such limitations in practice.  The HTTP interface for a resource
   is just as likely to be implemented as a tree of content objects, a
   programmatic view on various database records, or a gateway to other
   information systems.  Even when the URI mapping mechanism is tied to
   a file system, an origin server might be configured to execute the
   files with the request as input and send the output as the
   representation rather than transfer the files directly.  Regardless,
   only the origin server needs to know how each of its resource
   identifiers corresponds to an implementation and how each
   implementation manages to select and send a current representation of
   the target resource in a response to GET.

  

4.3.2.  HEAD

   The HEAD method is identical to GET except that the server MUST NOT
   send a message body in the response (i.e., the response terminates at
   the end of the header section).  The server SHOULD send the same
   header fields in response to a HEAD request as it would have sent if
   the request had been a GET, except that the payload header fields
    MAY be omitted.  This method can be used for obtaining
   metadata about the selected representation without transferring the
   representation data and is often used for testing hypertext links for
   validity, accessibility, and recent modification.

   

4.3.3.  POST

   The POST method requests that the target resource process the
   representation enclosed in the request according to the resource's
   own specific semantics.  For example, POST is used for the following
   functions (among others):

   o  Providing a block of data, such as the fields entered into an HTML
      form, to a data-handling process;
      o  Posting a message to a bulletin board, newsgroup, mailing list,
      blog, or similar group of articles;

   o  Creating a new resource that has yet to be identified by the
      origin server; and

   o  Appending data to a resource's existing representation(s).

  


   4.3.4.  PUT

   The PUT method requests that the state of the target resource be
   created or replaced with the state defined by the representation
   enclosed in the request message payload.  A successful PUT of a given
   representation would suggest that a subsequent GET on that same
   target resource will result in an equivalent representation being
   sent in a 200 (OK) response.  However, there is no guarantee that 
   such a state change will be observable, since the target resource
   might be acted upon by other user agents in parallel, or might be
   subject to dynamic processing by the origin server, before any
   subsequent GET is received.  A successful response only implies that
   the user agent's intent was achieved at the time of its processing by
   the origin server.


  
    
    4.3.5.  DELETE
    
       The DELETE method requests that the origin server remove the
       association between the target resource and its current
       functionality.  In effect, this method is similar to the rm command
       in UNIX: it expresses a deletion operation on the URI mapping of the
       origin server rather than an expectation that the previously
       associated information be deleted.
    
      
    
       Relatively few resources allow the DELETE method -- its primary use
       is for remote authoring environments, where the user has some
       direction regarding its effect.  For example, a resource that was
       previously created using a PUT request, or identified via the
       Location header field after a 201 (Created) response to a POST
       request, might allow a corresponding DELETE request to undo those
       actions.  Similarly, custom user agent implementations that implement an authoring function, such as revision control clients using HTTP
       for remote operations, might use DELETE based on an assumption that
       the server's URI space has been crafted to correspond to a version
       repository.
    
      
    
    4.3.6.  CONNECT
    
       The CONNECT method requests that the recipient establish a tunnel to
       the destination origin server identified by the request-target and,
       if successful, thereafter restrict its behavior to blind forwarding
       of packets, in both directions, until the tunnel is closed.  Tunnels
       are commonly used to create an end-to-end virtual connection, through
       one or more proxies, which can then be secured using TLS (Transport
       Layer Security, ).
    
       CONNECT is intended only for use in requests to a proxy.  
         However, most origin servers do not implement CONNECT.
     
    
       There are significant risks in establishing a tunnel to arbitrary
       servers, particularly when the destination is a well-known or
       reserved TCP port that is not intended for Web traffic.  For example,
       a CONNECT to a request-target of "example.com:25" would suggest that
       the proxy connect to the reserved port for SMTP traffic; if allowed,
       that could trick the proxy into relaying spam email.  Proxies that
       support CONNECT SHOULD restrict its use to a limited set of known
       ports or a configurable whitelist of safe request targets.
    
    
    4.3.7.  OPTIONS
    
       The OPTIONS method requests information about the communication
       options available for the target resource, at either the origin
       server or an intervening intermediary.  This method allows a client
       to determine the options and/or requirements associated with a
       resource, or the capabilities of a server, without implying a
       resource action.
       An OPTIONS request with an asterisk ("*") as the request-target
    applies to the server in general rather
   than to a specific resource.  Since a server's communication options
   typically depend on the resource, the "*" request is only useful as a
   "ping" or "no-op" type of method; it does nothing beyond allowing the
   client to test the capabilities of the server.  For example, this can
   be used to test a proxy for HTTP/1.1 conformance (or lack thereof).

  
   A server generating a successful response to OPTIONS SHOULD send any
   header fields that might indicate optional features implemented by
   the server and applicable to the target resource (e.g., Allow),
   including potential extensions not defined by this specification.
   The response payload, if any, might also describe the communication
   options in a machine or human-readable representation.  A standard
   format for such a representation is not defined by this
   specification, but might be defined by future extensions to HTTP.  A
   server MUST generate a Content-Length field with a value of "0" if no
   payload body is to be sent in the response.

4.3.8.  TRACE

   The TRACE method requests a remote, application-level loop-back of
   the request message.  The final recipient of the request SHOULD
   reflect the message received, excluding some fields described below,
   back to the client as the message body of a 200 (OK) response with a
   Content-Type of "message/http" .  The
   final recipient is either the origin server or the first server to
   receive a Max-Forwards value of zero (0) in the request
  .

   TRACE allows the client to see what is being received at the other
   end of the request chain and use that data for testing or diagnostic
   information.  The value of the Via header field  is of particular interest, since it acts as a trace of the
   request chain.  Use of the Max-Forwards header field allows the
   client to limit the length of the request chain, which is useful for
   testing a chain of proxies forwarding messages in an infinite loop.



\subsection{Response Status Codes} 
The status-code element is a three-digit integer code giving the
   result of the attempt to understand and satisfy the request.

   HTTP status codes are extensible.  HTTP clients are not required to
   understand the meaning of all registered status codes, though such
   understanding is obviously desirable.  However, a client MUST
   understand the class of any status code, as indicated by the first
   digit, and treat an unrecognized status code as being equivalent to
   the x00 status code of that class, with the exception that a
   recipient MUST NOT cache a response with an unrecognized status code.

   For example, if an unrecognized status code of 471 is received by a
   client, the client can assume that there was something wrong with its
   request and treat the response as if it had received a 400 (Bad
   Request) status code.  The response message will usually contain a
   representation that explains the status.

   The first digit of the status-code defines the class of response.
   The last two digits do not have any categorization role.  There are
   five values for the first digit:

   o  1xx (Informational): The request was received, continuing process

   o  2xx (Successful): The request was successfully received,
      understood, and accepted

   o  3xx (Redirection): Further action needs to be taken in order to
      complete the request

   o  4xx (Client Error): The request contains bad syntax or cannot be
      fulfilled
      o  5xx (Server Error): The server failed to fulfill an apparently
      valid request

\subsection{Overview of Status Codes}

   The status codes listed below are defined in this specification,
   Section 4 of [RFC7232], Section 4 of [RFC7233], and Section 3 of
   [RFC7235].  The reason phrases listed here are only recommendations
   -- they can be replaced by local equivalents without affecting the
   protocol.

   Responses with status codes that are defined as cacheable by default
   (e.g., 200, 203, 204, 206, 300, 301, 404, 405, 410, 414, and 501 in
   this specification) can be reused by a cache with heuristic
   expiration unless otherwise indicated by the method definition or
   explicit cache controls [RFC7234]; all other status codes are not
   cacheable by default.

   \begin{longtable}{|l|l|} 
      \hline
      \textbf{Code} & \textbf{Reason-Phrase}
   \\ \hline 100  & Continue                      
   \\ \hline 101  & Switching Protocols           
   \\ \hline 200  & OK                            
   \\ \hline 201  & Created                       
   \\ \hline 202  & Accepted                      
   \\ \hline 203  & Non-Authoritative Information 
   \\ \hline 204  & No Content                    
   \\ \hline 205  & Reset Content                 
   \\ \hline 206  & Partial Content               
   \\ \hline 300  & Multiple Choices              
   \\ \hline 301  & Moved Permanently             
   \\ \hline 302  & Found                         
   \\ \hline 303  & See Other                     
   \\ \hline 304  & Not Modified                  
   \\ \hline 305  & Use Proxy                     
   \\ \hline 307  & Temporary Redirect            
   \\ \hline 400  & Bad Request                   
   \\ \hline 401  & Unauthorized                  
   \\ \hline 402  & Payment Required              
   \\ \hline 403  & Forbidden                     
   \\ \hline 404  & Not Found                     
   \\ \hline 405  & Method Not Allowed            
   \\ \hline 406  & Not Acceptable                
   \\ \hline 407  & Proxy Authentication Required 
   \\ \hline 408  & Request Timeout               
   \\ \hline 409  & Conflict                      
   \\ \hline 410  & Gone                          
   \\ \hline 411  & Length Required               
   \\ \hline 412  & Precondition Failed           
   \\ \hline 413  & Payload Too Large             
   \\ \hline 414  & URI Too Long                  
   \\ \hline 415  & Unsupported Media Type        
   \\ \hline 416  & Range Not Satisfiable         
   \\ \hline 417  & Expectation Failed            
   \\ \hline 426  & Upgrade Required              
   \\ \hline 500  & Internal Server Error         
   \\ \hline 501  & Not Implemented               
   \\ \hline 502  & Bad Gateway                   
   \\ \hline 503  & Service Unavailable           
   \\ \hline 504  & Gateway Timeout               
   \\ \hline 505  & HTTP Version Not Supported    
   \\ \hline
\end{longtable}

   
   
   \subsubsection*{Informational 1xx}

   The 1xx (Informational) class of status code indicates an interim
   response for communicating connection status or request progress
   prior to completing the requested action and sending a final
   response. 1xx responses are terminated by the first empty line after
   the status-line (the empty line signaling the end of the header
   section).  Since HTTP/1.0 did not define any 1xx status codes, a
   server MUST NOT send a 1xx response to an HTTP/1.0 client.

  


   \subsubsection*{Successful 2xx}

   The 2xx (Successful) class of status code indicates that the client's
   request was successfully received, understood, and accepted.



   \subsubsection*{Redirection 3xx}

   The 3xx (Redirection) class of status code indicates that further
   action needs to be taken by the user agent in order to fulfill the
   request.  If a Location header field (Section 7.1.2) is provided, the
   user agent MAY automatically redirect its request to the URI
   referenced by the Location field value, even if the specific status
   code is not understood.  Automatic redirection needs to done with
   care for methods not known to be safe, as defined in Section 4.2.1,
   since the user might not wish to redirect an unsafe request.

   There are several types of redirects:

   1.  Redirects that indicate the resource might be available at a
       different URI, as provided by the Location field, as in the
       status codes 301 (Moved Permanently), 302 (Found), and 307
       (Temporary Redirect).

   2.  Redirection that offers a choice of matching resources, each
       capable of representing the original request target, as in the
       300 (Multiple Choices) status code.

   3.  Redirection to a different resource, identified by the Location
       field, that can represent an indirect response to the request, as
       in the 303 (See Other) status code.

   4.  Redirection to a previously cached result, as in the 304 (Not
       Modified) status code.



\subsubsection*{Client Error 4xx}

   The 4xx (Client Error) class of status code indicates that the client
   seems to have erred.  Except when responding to a HEAD request, the
   server SHOULD send a representation containing an explanation of the
   error situation, and whether it is a temporary or permanent
   condition.  These status codes are applicable to any request method.
   User agents SHOULD display any included representation to the user.




\subsubsection*{Server Error 5xx}

   The 5xx (Server Error) class of status code indicates that the server
   is aware that it has erred or is incapable of performing the
   requested method.  Except when responding to a HEAD request, the
   server SHOULD send a representation containing an explanation of the
   error situation, and whether it is a temporary or permanent   condition.  A user agent SHOULD display any included representation
   to the user.  These response codes are applicable to any request
   method.

  

\subsection{HTTPS con SSL} 

With an understanding of some of the key concepts of cryptography,
we can now look closely at the operation of the Secure Sockets Layer
(ssl) protocol. Although ssl is not an extremely complicated protocol, 
it does offer several options and variations. 
The ssl protocol consists of a set of messages and rules about when
to send (and not to send) each one. In this chapter, we consider what
those messages are, the general information they contain, and how
systems use the different messages in a communications session. 

\subsubsection*{SSL Roles}
The Secure Sockets Layer protocol defines two different roles for the
communicating parties. One system is always a client, while the other
is a server. The distinction is very important, because ssl requires the
two systems to behave very differently. The client is the system that
initiates the secure communications; the server responds to the client’s request. 
In the most common use of ssl, secure Web browsing,
the Web browser is the ssl client and the Web site is the ssl server.
For ssl itself, the most important distinctions between clients and
servers are their actions during the negotiation of security parameters. 
Since the client initiates a communication, it has the
responsibility of proposing a set of ssl options to use for the
exchange. The server selects from the client’s proposed options,
deciding what the two systems will actually use. Although the final
decision rests with the server, the server can only choose from among
those options that the client originally proposed.

\subsubsection*{SSL Messages}
When ssl clients and servers communicate, they do so by exchanging ssl messages. 
this chapter will show how systems use these messages in their communications.
The most basic function that an ssl client and server can perform is
establishing a channel for encrypted communications. 
This section looks at these steps
in more detail by considering each message in the exchange.


GRAFIQUITO DE LOS MENSAJES 

\paragraph*{ClientHello}
The ClientHello message starts the ssl communication between the
two parties. The client uses this message to ask the server to begin
negotiating security services by using ssl. 

The Version field of the ClientHello message contains the highest
version number of ssl that the client can support. The current ssl
version is 3.0, and it is by far the most widely deployed on the Internet. 

The RandomNumber field, as you might expect, contains a random
number. This random value, along with a similar random value that
the server creates, provides the seed for critical cryptographic calculations.
  The ssl specification suggests that
four of this field’s 32 bytes consist of the time and date. 

The next field in the ClientHello message is SessionID. Although all
ClientHello messages may include this field, in this example, the
field is meaningless and would be empty. 
The CipherSuites field allows a client to list the various cryptographic
services that the client can support, including exact algorithms and
key sizes. The server actually makes the final decision as to which
cryptographic services will be used for the communication, but it is
limited to choosing from this list. 

The CompressionMethods field is, in theory, similar to the CipherSuites field.
 In it, the client may list all of the various data compression methods that 
 it can support. Compression methods are an
important part of ssl because encryption has significant consequences on the 
effectiveness of any data compression techniques. Encryption changes the 
mathematical properties of information in a
way that makes data compression virtually impossible. 

\paragraph*{ServerHello}
When the server receives the ClientHello message, it responds with a
ServerHello. 
The Version field is the first example of a server making a final 
decision for the communications. The ClientHello’s version simply identifies 
which ssl versions the client can support. The ServerHello’s
version, on the other hand, determines the ssl version that the communication
 will use. 
The RandomNumber field of the ServerHello is essentially the same
as in the ClientHello, though this random value is chosen by the
server. Along with the client’s value, this number seeds important
cryptographic calculations. 
The SessionID field of a ServerHello may contain a value, unlike the
ClientHello’s field just discussed. The value in this case uniquely
identifies this particular ssl communication, or session. The main reason for 
explicitly identifying a particular ssl session is to refer to it
again later.
The CipherSuite field (note that the name is singular, not plural, as in
the case of a ClientHello) determines the exact cryptographic parameters, 
specifically algorithms and key sizes, to be used for the session. 
The server must select a single cipher suite from among those
listed by the client in its ClientHello message.
The CompressionMethod field is also singular for a ServerHello. In
theory, the server uses this field to identify the data compression to
be used for the session. Again, the server must pick from among
those listed in the ClientHello. Current ssl versions have not defined
any compression methods, however, so this field has no practical utility.

\paragraph*{ServerKeyExchange}
In this example, the server follows its ServerHello message with a
ServerKeyExchange message. This message complements the CipherSuite field 
of the ServerHello. While the CipherSuite field indicates
the cryptographic algorithms and key sizes, this message contains the
public key information itself. The exact format of the key information depends
 on the particular public key algorithm used. For the rsa
algorithm, for example, the server includes the modulus and public
exponent of the server’s rsa public key.
Note that the ServerKeyExchange message is transmitted without
encryption, so that only public key information can be safely included within it.
 The client will use the server’s public key to encrypt
a session key, which the parties will use to actually encrypt the application
 data for the session.
\paragraph*{ServerHelloDone}
The ServerHelloDone message tells the client that the server has finished with
 its initial negotiation messages. The message itself contains no other 
 information, but it is important to the client, because
once the client receives a ServerHelloDone, it can move to the next
phase of establishing the secure communications.
\paragraph*{ClientKeyExchange}
When the server has finished its part of the initial ssl negotiation,
the client responds with a ClientKeyExchange message. Just as the
ServerKeyExchange provides the key information for the server, the
ClientKeyExchange tells the server the client’s key information. In

this case, however, the key information is for the symmetric encryption 
algorithm both parties will use for the session. Furthermore, the
information in the client’s message is encrypted using the public key
of the server. This encryption protects the key information as it traverses
 the network, and it allows the client to verify that the server
truly possesses the private key corresponding to its public key. Otherwise,
 the server won’t be able to decrypt this message. This operation is an 
 important protection against an attacker that intercepts
messages from a legitimate server and pretends to be that server by
forwarding the messages to an unsuspecting client. Since a fake
server won’t know the real server’s private key, it won’t be able to decrypt 
the ClientKeyExchange message. Without the information in
that message, communication between the two parties cannot succeed.
\paragraph*{ChangeCipherSpec}
After the client sends key information in a ClientKeyExchange message, 
the preliminary ssl negotiation is complete. At that point, the
parties are ready to begin using the security services they have negotiated.
 The ssl protocol defines a special message—
ChangeCipherSpec—to explicitly indicate that the security services
should now be invoked.
Since the transition to secured communication is critical, and both
parties have to get it exactly right, the ssl specification is very precise
in describing the process. First, it identifies the set of information
that defines security services. That information includes a specific
symmetric encryption algorithm, a specific message integrity algorithm, 
and specific key material for those algorithms. The ssl specification also 
recognizes that some of that information (in particular,
the key material) will be different for each direction of communication. 
In other words, one set of keys will secure data the client sends
to the server, and a different set of keys will secure data the server
sends to the client. (In principle, the actual algorithms could differ as
well, but ssl does not define a way to negotiate such an option.) For
any given system, whether it is a client or a server, ssl defines a write
state and a read state. The write state defines the security information

for data that the system sends, and the read state defines the security
information for data that the system receives.
The ChangeCipherSpec message serves as the cue for a system to
begin using its security information. Before a client or server sends a
ChangeCipherSpec message, it must know the complete security information it 
is about to activate. As soon as the system sends this
message, it activates its write state. Similarly, as soon as a system receives
 a ChangeCipherSpec from its peer, the system activates its
read state.

GRAFIQUITO Figures 3-2 and 3-3

\paragraph*{Finished}
Immediately after sending their ChangeCipherSpec messages, each
system also sends a Finished message. The Finished messages allow
both systems to verify that the negotiation has been successful and
that security has not been compromised. Two aspects of the Finished
message contribute to this security. First, as the previous subsection
explained, the Finished message itself is subject to the negotiated cipher suite. 
That means that it is encrypted and authenticated according to that suite.
 If the receiving party cannot successfully decrypt
and verify the message, then clearly something has gone awry with
the security negotiation.
The contents of the Finished message also serve to protect the security of the
 ssl negotiation. Each Finished message contains a cryptographic hash of 
 important information about the just-finished
negotiation.  Notice that protected data includes the exact content of all
handshake messages used in the exchange (though ChangeCipherSpec messages 
are not considered “handshake” messages in the strict
sense of the word, and thus are not included). This protects against
an attacker who manages to insert fictitious messages or remove 
legitimate messages from the communication. If an attacker were able
to do so, the client’s and server’s hash calculations would not match,
and they would detect the compromise.

\paragraph*{Ending Secure Communications}

Although as a practical matter it is rarely used (primarily due to the
nature of Web sessions), ssl does have a defined procedure for ending a
 secure communication between two parties. In this procedure,
 the two systems each send a special ClosureAlert to the other.
Explicitly closing a session protects against a truncation attack, in
which an attacker is able to compromise security by prematurely terminating
 a communication. 

\subsubsection*{Authenticating the Server’s Identity}
previously it was explained how ssl can establish encrypted
communications between two parties, that may not really add that
much security to the communication. With encryption alone neither
party can really be sure of the other’s identity. The typical reason for
using encryption in the first place is to keep information secret from
some third party. But if that third party were able to successfully
masquerade as the intended recipient of the information, then encryption 
would serve no purpose. The data would be encrypted, but
the attacker would have all the data necessary to decrypt it.
To avoid this type of attack, ssl includes mechanisms that allow each
party to authenticate the identity of the other. With these mechanisms, 
each party can be sure that the other is genuine, and not a
masquerading attacker. In this section, we’ll look at how ssl enables a
server to authenticate itself.
A natural question is, of course, if authenticating identities is so important,
 why don’t we always authenticate both parties? 
 Aca un ejemplo que sirva en una red interna
 The answer
lies in the nature of Web commerce. When you want to purchase
something using your Web browser, it’s very important that the Web
site you’re browsing is authentic. You wouldn’t want to send your
credit card number to some imposter posing as your favorite merchant. 
The merchant, on the other hand, has other means for
authenticating your identity. Once it receives a credit card number,
for example, it can validate that number. Since the server doesn’t
need ssl to authenticate your identity, the ssl protocol allows for
server authentication only. 

\paragraph*{Certificate}
When authenticating your identity, the server sends a Certificate message 
in place of the ServerKeyExchange message previously described. The Certificate 
message simply contains a certificate chain
that begins with the server’s public key certificate and ends with the
certificate authority’s root certificate.
The client has the responsibility to make sure it can trust the certificate it 
receives from the server. That responsibility includes verifying
the certificate signatures, validity times, and revocation status. It also
means ensuring that the certificate authority is one that the client
trusts. Typically, clients make this determination by knowing the
public key of trusted certificate authorities in advance, through some
trusted means. Netscape and Microsoft, for example, preload their
browser software with public keys for well-known certificate authorities. 
Web servers that want to rely on this trust mechanism can only
obtain their certificates (at least indirectly) from one of these wellknown
 authorities.

\paragraph*{ClientKeyExchange}
The client’s ClientKeyExchange message also differs in server authentication, 
though the difference is not major. When encryption
only is to be used, the client encrypts the information in the ClientKeyExchange
 using the public key the server provides in its
ServerKeyExchange message. In this case, of course, the server is authenticating
 itself and, thus, has sent a Certificate message instead of
a ServerKeyExchange. The client, therefore, encrypts its ClientKeyExchange information 
using the public key contained in the
server’s certificate. This step is important because it allows the client
to make sure that the party with whom it is communicating actually
possesses the server’s private key. Only a system with the actual private key will 
be able to decrypt this message and successfully continue the communication.

\subsubsection*{Certificate Functionality}

\paragraph*{Single Domain}
As the name suggests, a single domain SSL certificate can only be used on a single 
domain or IP. This is considered the default SSL certificate type. The DV SSL type is
available at all validation levels.
\paragraph*{Multi-Domain}
This SSL type is a jack-of-all-trades certificate. Multi-Domain Wildcards can encrypt
 up to 250 different domains and unlimited sub-domains. Unfortunately, it's not available
  in EV.
\paragraph*{Wildcard}
Wildcards are specifically designed to encrypt one domain and all of its accompanying 
sub-domains. Unlimited sub-domains. Unfortunately, Wildcards are only available at the 
DV and OV levels.
\paragraph*{Multi-Domain Wildcard}
These are the jack-of-all-trades certificates. Multi-Domain Wildcards can encrypt up 
to 250 different domains and unlimited sub-domains. Unfortunately, it's not available 
in EV.

\subsubsection*{Validation Level}
There are three types of SSL Certificate available today; Extended Validation 
(EV SSL), Organization Validated (OV SSL) and Domain Validated (DV SSL). The 
encryption levels are the same for each certificate, what differs is the vetting
 and verification processes needed to obtain the certificate.
\paragraph*{Domain Validation (DV)}
Domain Validation SSL or DV SSL represents the base-level for SSL types. These 
are perfect for websites that just need encryption and nothing more. DV SSL 
certificates are typically inexpensive and they can be issued within minutes. That's 
because the validation process is fully automated. Just prove you own your domain 
and the DV SSL certificate is yours.
\paragraph*{Organization Validation (OV)}
Organization Validation SSL or OV SSL represents the middle ground for SSL certificate 
types. To obtains OV SSL, your company or organization must undergo light business
 vetting. This can take up to three business days because someone has to verify 
 your business information. OV SSL displays the same visual indicators as DV SSL,
  but provides a way for your customers to check your verified business information 
  in the certificate details section.
\paragraph*{Extended Validation (EV)}
Extended Validation SSL or EV SSL requires extensive business vetting by Comodo. This
 may sound like a lot, but it's really not if your business has publicly available 
 records. EV SSL activates a unique visual indicator – your verified organization name
  shown in the browser.

  
\subsubsection*{Identifier Validation Challenges}

(CAMBIAR LA INTRO, NO ME GUSTA LO DE IDENTIFICADOR, RELACIONARLO MAS CON UN DOMINIO)

ACME uses an extensible challenge/response framework for identifier
validation.  The server presents a set of challenges in the
authorization object it sends to a client (as objects in the
"challenges" array), and the client responds by sending a response
object in a POST request to a challenge URL.

   Different challenges allow the server to obtain proof of different
   aspects of control over an identifier.  In some challenges, like HTTP
   and DNS, the client directly proves its ability to do certain things
   related to the identifier.  The choice of which challenges to offer
   to a client under which circumstances is a matter of server policy.


   

\paragraph*{HTTP Challenge}

   With HTTP validation, the client in an ACME transaction proves its
   control over a domain name by proving that it can provision HTTP
   resources on a server accessible under that domain name.  The ACME
   server challenges the client to provision a file at a specific path,
   with a specific string as its content.

   This is the most common challenge type today. The server gives a
   token to your ACME client, and your ACME client puts a file on your web 
   server at {http://\<YOUR\_DOMAIN\>/.well-known/acme-challenge/\<TOKEN\>}. That 
   file contains the token, plus a thumbprint of your account key. 

   Once the client tells to the server that the file is ready, the server 
   tries retrieving it.On receiving a response, the server constructs and stores the key
   authorization from the challenge "token" value and the current client
   account key.

   Given a challenge/response pair, the server verifies the client's
   control of the domain by verifying that the resource was provisioned
   as expected.

   (TAL VEZ PARA LA PRESENTACION)

   Pros:

    It’s easy to automate without extra knowledge about a domain’s configuration.
    It allows hosting providers to issue certificates for domains CNAMEd to them.
    It works with off-the-shelf web servers.

   Cons:

    It doesn’t work if your ISP blocks port 80 (this is rare, but some residential ISPs do this).
    Let’s Encrypt doesn’t let you use this challenge to issue wildcard certificates.
    If you have multiple web servers, you have to make sure the file is available on all of them.

   (EXPLICAR POR QUE NO PUEDO USAR ESTE DESAFIO)

\paragraph*{DNS Challenge}
   When the identifier being validated is a domain name, the client can
   prove control of that domain by provisioning a TXT resource record
   containing a designated value for a specific validation domain name.

   A client fulfills this challenge by constructing a key authorization
   from the "token" value provided in the challenge and the client's
   account key.  The client then computes the SHA-256 digest 
   of the key authorization.

   The record provisioned to the DNS contains the base64url encoding of
   this digest.  The client constructs the validation domain name by
   prepending the label {\_acme-challenge} to the domain name being
   validated, then provisions a TXT record with the digest value under
   that name.  For example, if the domain name being validated is
   "www.example.org", then the client would provision the following DNS
   record:
   \_acme-challenge.www.example.org. 300 IN TXT "gfj9Xq...Rg85nM"
   
   On receiving a response, the server constructs and stores the key
   authorization from the challenge "token" value and the current client
   account key.

   To validate a DNS challenge, the server performs the following steps:

   1.  Compute the SHA-256 digest of the stored key
       authorization

   2.  Query for TXT records for the validation domain name

   3.  Verify that the contents of one of the TXT records match the
       digest value

   If all of the above verifications succeed, then the validation is
   successful.  If no DNS record is found, or DNS record and response
   payload do not pass these checks, then the validation fails.

   The client SHOULD de-provision the resource record(s) provisioned for
   this challenge once the challenge is complete, i.e., once the
   "status" field of the challenge has the value "valid" or "invalid".



\section{Protocolos asociados a la consulta de un sitio (DNS)}
    \subsection{¿Que es el protocolo DNS?} 







\chapter{Network Security Attacks: basic concepts} 
    \label{capDesc}
%-*- ES -*-
%----------------------------------------------------------------------
% Capitulo 7: Descripción del problema

En éste capítulo se verá por qué los protocolos http y smtp son inseguros, introducción al snnifin, spoofing, arp attack
%----------------------------------------------------------------------

\section{El problema de los protocolos http y smtp}


(Agregar una intro)

\subsection{Establishing Authority}

HTTP relies on the notion of an authoritative response: a response
that has been determined by (or at the direction of) the authority
identified within the target URI to be the most appropriate response
for that request given the state of the target resource at the time
of response message origination.  Providing a response from a
non-authoritative source, such as a shared cache, is often useful to
improve performance and availability, but only to the extent that the
source can be trusted or the distrusted response can be safely used.

Unfortunately, establishing authority can be difficult.  For example,
phishing is an attack on the user's perception of authority, where
that perception can be misled by presenting similar branding in
hypertext, possibly aided by userinfo obfuscating the authority
component (see Section 2.7.1).  User agents can reduce the impact of
phishing attacks by enabling users to easily inspect a target URI
prior to making an action, by prominently distinguishing (or
rejecting) userinfo when present, and by not sending stored
credentials and cookies when the referring document is from an
unknown or untrusted source.

When a registered name is used in the authority component, the "http"
URI scheme (Section 2.7.1) relies on the user's local name resolution
service to determine where it can find authoritative responses.  This
means that any attack on a user's network host table, cached names,
or name resolution libraries becomes an avenue for attack on
establishing authority.  Likewise, the user's choice of server for
Domain Name Service (DNS), and the hierarchy of servers from which it
obtains resolution results, could impact the authenticity of address
mappings; DNS Security Extensions are one way to
improve authenticity.

Furthermore, after an IP address is obtained, establishing authority
for an "http" URI is vulnerable to attacks on Internet Protocol
routing.


\subsection{Risks of Intermediaries}

By their very nature, HTTP intermediaries are men-in-the-middle and,
thus, represent an opportunity for man-in-the-middle attacks.
Compromise of the systems on which the intermediaries run can result
in serious security and privacy problems.  Intermediaries might have
access to security-related information, personal information about
individual users and organizations, and proprietary information
belonging to users and content providers.  A compromised
intermediary, or an intermediary implemented or configured without
regard to security and privacy considerations, might be used in the
commission of a wide range of potential attacks.

Intermediaries that contain a shared cache are especially vulnerable
to cache poisoning attacks.
Implementers need to consider the privacy and security implications
of their design and coding decisions, and of the configuration
options they provide to operators (especially the default
configuration).

Users need to be aware that intermediaries are no more trustworthy
than the people who run them; HTTP itself cannot solve this problem.

\subsection{Attacks via Protocol Element Length}

Because HTTP uses mostly textual, character-delimited fields, parsers
are often vulnerable to attacks based on sending very long (or very
slow) streams of data, particularly where an implementation is
expecting a protocol element with no predefined length.



\subsection{Response Splitting}

Response splitting (a.k.a, CRLF injection) is a common technique,
used in various attacks on Web usage, that exploits the line-based
nature of HTTP message framing and the ordered association of
requests to responses on persistent connections [Klein].  This
technique can be particularly damaging when the requests pass through
a shared cache.

Response splitting exploits a vulnerability in servers (usually
within an application server) where an attacker can send encoded data
within some parameter of the request that is later decoded and echoed
within any of the response header fields of the response.  If the
decoded data is crafted to look like the response has ended and a
subsequent response has begun, the response has been split and the
content within the apparent second response is controlled by the
attacker.  The attacker can then make any other request on the same
persistent connection and trick the recipients (including
intermediaries) into believing that the second half of the split is
an authoritative answer to the second request.

For example, a parameter within the request-target might be read by
an application server and reused within a redirect, resulting in the
same parameter being echoed in the Location header field of the
response.  If the parameter is decoded by the application and not
properly encoded when placed in the response field, the attacker can
send encoded CRLF octets and other content that will make the
application's single response look like two or more responses.


\subsection{Request Smuggling}

Request smuggling ([Linhart]) is a technique that exploits
differences in protocol parsing among various recipients to hide
additional requests (which might otherwise be blocked or disabled by
policy) within an apparently harmless request.  Like response
splitting, request smuggling can lead to a variety of attacks on HTTP
usage.


\subsection{Message Integrity}

HTTP does not define a specific mechanism for ensuring message
integrity, instead relying on the error-detection ability of
underlying transport protocols and the use of length or
chunk-delimited framing to detect completeness.  Additional integrity
mechanisms, such as hash functions or digital signatures applied to
the content, can be selectively added to messages via extensible
metadata header fields.  Historically, the lack of a single integrity
mechanism has been justified by the informal nature of most HTTP
communication.  However, the prevalence of HTTP as an information
access mechanism has resulted in its increasing use within
environments where verification of message integrity is crucial.

User agents are encouraged to implement configurable means for
detecting and reporting failures of message integrity such that those
means can be enabled within environments for which integrity is
necessary.  For example, a browser being used to view medical history
or drug interaction information needs to indicate to the user when
such information is detected by the protocol to be incomplete,
expired, or corrupted during transfer.  Such mechanisms might be
selectively enabled via user agent extensions or the presence of
message integrity metadata in a response.  At a minimum, user agents
ought to provide some indication that allows a user to distinguish
between a complete and incomplete response message (Section 3.4) when
such verification is desired.

\subsection{Message Confidentiality}

HTTP relies on underlying transport protocols to provide message
confidentiality when that is desired.  HTTP has been specifically
designed to be independent of the transport protocol, such that it
can be used over many different forms of encrypted connection, with
the selection of such transports being identified by the choice of
URI scheme or within user agent configuration.

\section{Network Security Attacks: basic concepts}
There are mainly two types of network attacks – passive attack and active attack

Passive: This type of attack happens when sensitive information is monitored and
analyzed, possibly compromising the security of enterprises and their customers.
In short, network intruder intercepts data traveling through the network.
• Active: This type of attack happens when information is modified, altered or
demolished entirely by a hacker. Here the interloper starts instructions to disturb
the network’s regular process.

So the motives behind passive attackers and active attackers are totally different.
Whereas the motive of passive attackers is simply to steal sensitive information and
to analyze the traffic to steal future messages, the motive of active attackers is to stop
normal communication between two legitimate entities.
\subsection{Passive Attacks}
Passive attackers aremainly interested in stealing sensitive information. This happens
without the knowledge of the victim. As such passive attacks are difficult to detect
and thereby secure the network. The following are some of the passive attacks that
are in existence [7].
\begin{itemize}
    \item Traffic Analysis: Attacker senses the communication path between the sender
    and receiver.
    \item Monitoring: Attacker can read the confidential data, but he cannot edit or modify
    the data.
    \item Eavesdropping: This type of attack occurs in the mobile ad-hoc network where
    basically the attacker finds out some secret or confidential information from
    communication.
\end{itemize}

\subsection{Active Attacks}
The active attacks happen in such a manner so as to notify the victims that their
systems have been compromised. As a result, the victim stops communication with
the other party. Some of the active attacks are as follows [8].
\begin{itemize}
    \item Modification: Some alterations in the routing route is performed by the malicious
    node. This results in causing the sender to send messages through the long route,
    which causes communication delay. This is an attack on integrity as shown in
    Fig. 2.
    \item Wormhole: This attack is also called a tunneling attack. A packet is received
    by an attacker at one point. He then tunnels it to another malicious node in the
    network. This causes a beginner to assume that he found the shortest path in the
    network as shown in Fig. 3.
    \item Fabrication: A malicious node generates a false routing message that causes the
    generation of incorrect information about the route between devices. This is an
    attack on authenticity as shown in Fig. 4.    
    \item Spoofing: A malicious node miss-present his identity so that the sender changes
    his topology as shown in Fig. 5.
    \item Denial of services: A malicious node sends a message to the node and consumes
    the bandwidth of the network as given in Fig.
    \item  Man-in-the-middle: Attack—Also called a hijacking attack, it is an attack where
    the attacker secretly alters and relays the communications between two legitimate
    parties without their knowledge. These parties in turn are unaware of the secret
    hacker consider that they are doing direct communication with each other [13].
    Figure 12 depicts this attack.
\end{itemize}

\section{Herramientas utilizadas}
    \subsection{Kali Linux}
    BackTrack is one of the most famous Linux distribution systems, as can be proven by
the number of downloads that reached more than four million as of BackTrack Linux
4.0 pre final.

Kali Linux Version 1.0 was released on March 12, 2013. Five days later, Version 1.0.1
was released, which fixed the USB keyboard issue. In those five days, Kali has been
downloaded more than 90,000 times.

    Kali Linux is security-focused Linux distribution based on Debian. It's a rebranded
    version of the famous Linux distribution known as Backtrack, which came with
    a huge repository of open source hacking tools for network, wireless, and web
    application penetration testing. Although Kali Linux contains most of the tools
    from Backtrack, the main aim of Kali Linux is to make it portable so that it could
    be installed on devices based on the ARM architectures such as tablets and
    Chromebook, which makes the tools available at your disposal with much ease.

    Using open source hacking tools comes with a major drawback: they contain a
whole lot of dependencies when installed on Linux and they need to be installed in
a predefined sequence. Moreover, authors of some tools have not released accurate
documentation, which makes our life difficult.

Kali Linux simplifies this process; it contains many tools preinstalled with all the
dependencies and is in ready to use condition so that you can pay more attention
for the actual attack and not on installing the tool. Updates for tools installed in Kali
Linux are more frequently released, which helps you to keep the tools up to date. A
non-commercial toolkit that has all the major hacking tools preinstalled to test realworld networks and applications is a dream of every ethical hacker and the authors
of Kali Linux make every effort to make our life easy, which enables us to spend
more time on finding the actual flaws rather than building a toolkit.

(EXPLICAR COMO Y DONDE LO INSTALE)
(DECIR QUE FUE EL SISTEMA OPERATIVO UTILIZADO QUE CONTIENEN LAS SIGUIENTES HERRAMIENTAS)
\subsection{Wireshark}
Wireshark is one of the most popular, free, and open source network protocol
analyzers. Wireshark is preinstalled in Kali and ideal for network troubleshooting,
analysis, and for this chapter, a perfect tool to monitor traffic from potential targets
with the goal of capturing session tokens. Wireshark uses a GTK+ widget toolkit to
implement its user interface and pcap to capture packets. It operates very similarly to
a tcpdump command; however, acting as a graphical frontend with integrated sorting
and filtering options.
(HAY MAS, CON GRAFICOS EN)
Joseph Muniz, Aamir Lakhani - Web Penetration Testing with Kali Linux-Packt Publishing (2013)

    \subsection{Ettrcap}
    Ettercap is a free and open source comprehensive suite for man-in-the-middle-based
attacks.
Ettercap can be used for computer network protocol analysis and security auditing,
featuring sniffing live connections, content filtering, and support for active and passive
dissection of multiple protocols. Ettercap works by putting the attacker's network
interface into promiscuous mode and ARP for poisoning the victim machines.
\section{Virtualización}
(AGREGAR GRAFICOS SI O SI)
Virtualization provides abstraction on top of the actual resources we want to virtualize.
 The level at which this abstraction is applied changes the way that different
  virtualization techniques look.At a higher level, there are two major
   virtualization techniques based on the level of abstraction.
   • Virtual machine (VM)-based
   • Container-based
   Apart from these two virtualizing techniques, there are other techniques, such as unikernels, which are lightweight single-purpose VMs. IBM is currently attempting to run unikernels as processes with projects like Nabla. In this book, we will mainly look at VM-based and container-based virtualizations only.
\subsection{Máquinas virtuales}
VM-Based  VirtualizationThe VM-based approach virtualizes the complete OS. The abstraction
 it presents to the VM are virtual devices like virtual disks, virtual CPUs, and virtual 
 NICs. In other words, we can state that this is virtualizing the complete ISA
  (instruction set architecture); as an example, the x86 ISA.With virtual machines,
   multiple OSes can share the same hardware resources, with virtualized representations 
   of each of the resources available to the VM. For example, the OS on the virtual
    machine (also called the guest) can continue to do I/O operations on a disk 
    (in this case, it’s a virtual disk), thinking that it’s the only OS running on the 
    physical hardware (also called the host), although in actuality, it is shared by 
    multiple virtual machines as well as by the host OS.VMs are very similar to other 
    processes in the host OS. VMs execute in a hardware-isolated virtual address space 
    and at a lower privilege level than the host OS. The primary difference between a
     process and a VM is the ABI (Application Binary Interface) exposed by the host
      to the VM. In the case of a process, the exposed ABI has constructs like network 
      sockets, FDs, and so on, whereas with a full-fledged OS virtualization, the ABI 
      will have a virtual disk, a virtual CPU, virtual network cards, and so on.

Hypervisors
A special piece of software is used to virtualize the OS, called the hypervisor. The 
hypervisor itself has two parts:
• Virtual Machine Monitor (VMM): Used for trapping and emulating the privileged 
instruction set (which only the kernel of the operating system can perform).
The VMM has to satisfy three properties (Popek and Goldberg, 1973):
    • Isolation :  
Should isolate guests (VMs) from each other.
    • Equivalency : Should behave the same, 
with or without virtualization. This means we run the majority (almost all) of the 
instructions on the physical hardware without any translation, and so on.
    • Performance : 
Should perform as good as it does without any virtualization. This again means that 
the overhead of running a VM is minimal.

Device  Model
The device model of the hypervisor handles the I/O virtualization again by trapping 
and emulating and then delivering interrupts back to the specific virtual machine.
The device model handles:
•Memory  Virtualization
•Shadow Page Table
•CPU  Virtualization
•IO  Virtualization

\subsection{Container-Based  Virtualization}
Container-Based  VirtualizationThis form of virtualization doesn’t abstract the 
hardware but uses techniques within the Linux kernel to isolate access paths for
 different resources. It carves out a logical boundary within the same operating 
 system. As an example, we get a separate root file system, a separate process tree, 
 a separate network subsystem, and so on.


The Linux kernel is made up of several components and functionalities; the ones related to
containers are as follows:
Control groups (cgroups)
Namespaces
Security-Enhanced Linux (SELinux)
Cgroups
The cgroup functionality allows for limiting and prioritizing resources, such as CPUs,
RAM, the network, the filesystem, and so on. The main goal is to not exceed the
resources—to avoid wasting resources that might be needed for other processes.
Namespaces
The namespace functionality allows for partitioning of kernel resources, such that one set of
processes sees one set of resources, while another set of processes sees a different set of
resources. The feature works by having the same namespace for these resources in the
various sets of processes, but having those names refer to distinct resources. Examples of
resource names that can exist in multiple spaces (so that the named resources will be
partitioned) are process IDs, hostnames, user IDs, filenames, and some names associated
with network access and inter-process communication.
When a Linux system boots; that is, only one namespace is created. Processes and resources
will join the same namespace, until a different namespace is created, resources assigned to
it, and processes join it.
SELinux
SELinux is a module of the Linux kernel that provides a mechanism to enforce the security
of the system, with specific policies.
Basically, SELinux can limit programs from accessing files and network resources. The idea
is to limit the privileges of programs and daemons to a minimum, so that it can limit the
risk of system halt.
The preceding functionalities have been around for many years. Namespaces were first
released in 2002, and cgroups in 2005, by Google (cgroups were first named process
containers, and then cgroups). For example, SunSolaris 5.10, released at the beginning of
2005, provided support for Solaris containers.
Nowadays, Linux containers are the new buzzword, and some people think they are a
new means of virtualization.
Containerization uses resources directly, and does not need an emulator at all; the fewer
resources, the more efficiency. Different applications can run on the same host: isolated at
the kernel level and isolated by namespaces and cgroups. The kernel (that is, the OS) is
shared by all containers, as shown in the following diagram:
Containers
When we talk about containers, we are indirectly referring to two main concepts—a
container image and a running container image.
A container image is the definition of the container, wherein all software stacks are
installed as additional layers, as depicted by the following diagram:
A container image is typically made up of multiple layers.
The first layer is given by the base image, which provides the OS core functionalities, with
all of the tools needed to get started. Teams often work by building their own layers on
these base images. Users can also build on more advanced application images, which not
only have an OS, but which also include language runtimes, debugging tools, and libraries,
as shown in the following diagram:
Base images are built from the same utilities and libraries that are included in an OS. A
good base image provides a secure and stable platform on which to build applications. Red
Hat provides base images for Red Hat Enterprise Linux. These images can be used like a
normal OS. Users can customize them for their applications as necessary, installing
packages and enabling services to start up just like a normal Red Hat Enterprise Linux
Server.
Containers provide isolation by taking advantage of kernel technologies, like cgroups,
kernel namespaces, and SELinux, which have been battle-tested and used for years at
Google and the US Department of Defense, in order to provide application isolation.
Since containers use a shared kernel and container host, they reduce the amount of
resources required for the container itself, and are more lightweight when compared to
VMs. Therefore, containers provide an unmatched agility that is not feasible with VMs; for
example, it only takes a few seconds to start a new container. Furthermore, containers
support a more flexible model when it comes to CPU utilization and memory resources,
and allow for resource burst modes, so that applications can consume more resources when
required, within the defined boundaries.

\subsection{Docker}
Docker
Docker is an open-source engine that automates the deployment of
applications into containers. It was written by the team at Docker, Inc
(formerly dotCloud Inc, an early player in the Platform-as-a-Service
(PAAS) market), and released by them under the Apache 2.0 license.
Docker adds an application deployment
engine on top of a virtualized container execution environment. It is
designed to provide a lightweight and fast environment in which to run your
code as well as an efficient workflow to get that code from your laptop to
your test environment and then into production. Docker is incredibly
simple. Indeed, you can get started with Docker on a minimal host running
nothing but a compatible Linux kernel and a Docker binary. 

With Docker, Developers care about their applications running inside
containers, and Operations cares about managing the containers. Docker is
designed to enhance consistency by ensuring the environment in which
your developers write code matches the environments into which your
applications are deployed. This reduces the risk of “worked in dev, now an
ops problem.
\subsubsection{Docker components}
Let’s look at the core components that compose the Docker Community
Edition:
The Docker client and server, also called the Docker Engine.
Docker Images
Registries
Docker Containers
\paragraph{Docker client and server}
Docker is a client-server application. The Docker client talks to the Docker
server or daemon, which, in turn, does all the work. You’ll also sometimes
see the Docker daemon called the Docker Engine. Docker ships with a
command line client binary, docker, as well as a full RESTful API to
interact with the daemon: dockerd. You can run the Docker daemon and
client on the same host or connect your local Docker client to a remote
daemon running on another host. You can see Docker’s architecture
depicted here:
\paragraph{Docker images}
Images are the building blocks of the Docker world. You launch your
containers from images. Images are the “build” part of Docker’s life cycle.
They have a layered format, using Union file systems, that are built step-bystep using a series of instructions. For example:
Add a file.
Run a command.
Open a port.
You can consider images to be the “source code” for your containers. They
are highly portable and can be shared, stored, and updated. In the book,
we’ll learn how to use existing images as well as build our own images.
\paragraph{Registries}
Docker stores the images you build in registries. There are two types of
registries: public and private. Docker, Inc., operates the public registry for
images, called the Docker Hub. You can create an account on the Docker
Hub and use it to share and store your own images.
The Docker Hub also contains, at last count, over 10,000 images that other
people have built and shared. Want a Docker image for an Nginx web
server, the Asterisk open source PABX system, or a MySQL database? All
of these are available, along with a whole lot more.
You can also store images that you want to keep private on the Docker Hub.
These images might include source code or other proprietary information
you want to keep secure or only share with other members of your team or
organization.
You can also run your own private registry, and we’ll show you how to do
that in Chapter 4. This allows you to store images behind your firewall,
which may be a requirement for some organizations.
\paragraph{Containers}
Docker helps you build and deploy containers inside of which you can
package your applications and services. As we’ve just learned, containers
are launched from images and can contain one or more running processes.
You can think about images as the building or packing aspect of Docker and
the containers as the running or execution aspect of Docker.
A Docker container is:
An image format.
A set of standard operations.
An execution environment.
Docker borrows the concept of the standard shipping container, used to
transport goods globally, as a model for its containers. But instead of
shipping goods, Docker containers ship software.
Each container contains a software image – its ‘cargo’ – and, like its
physical counterpart, allows a set of operations to be performed. For
example, it can be created, started, stopped, restarted, and destroyed.
Like a shipping container, Docker doesn’t care about the contents of the
container when performing these actions; for example, whether a container
is a web server, a database, or an application server. Each container is
loaded the same as any other container.
Docker also doesn’t care where you ship your container: you can build on
your laptop, upload to a registry, then download to a physical or virtual
server, test, deploy to a cluster of a dozen Amazon EC2 hosts, and run. Like
a normal shipping container, it is interchangeable, stackable, portable, and
as generic as possible.
With Docker, we can quickly build an application server, a message bus, a
utility appliance, a CI test bed for an application, or one of a thousand other
possible applications, services, and tools. It can build local, self-contained
test environments or replicate complex application stacks for production or
development purposes. The possible use cases are endless.
   \section{Casos de estudio}
\subsection{Caso de estudio: Sniffing de la red para obtener credenciales}
(HAY MAS, CON GRAFICOS EN)
Joseph Muniz, Aamir Lakhani - Web Penetration Testing with Kali Linux-Packt Publishing (2013)
Aca yo segui un tutorial, buscarlo

La idea principal de esta sección es demostrar que, encontrandose en una red interna
y con con herramientas ya desarrolladas y libres es posible realizar un ataque 
sin necesidad de conocer a fondo la implementacion de la misma ni de tener mayores
privilegios

Recordar que esto fue realizado en una red interna donde son todos equipos de nuestra propiedad

\subsubsection{Diagrama de explicacion}
IMAGEN de le red

Tiene que estar:
-Router

-Origen de la pagina

-Consumidor de la pagina

-El atacante

\subsubsection{Preparando Ettercap para el ataque ARP Poisoning}

Lo primero que debemos hacer, en la lista de aplicaciones, es buscar el apartado 
«9. Sniffing y Spoofing«, ya que es allí donde encontraremos las herramientas necesarias
 para llevar a cabo este ataque.

IMAGEN Kali Linux Spoofing

A continuación, abriremos «Ettercap» y veremos una ventana similar a la siguiente.

IMAGEN  Kali Linux Ettercap

El siguiente paso es seleccionar la tarjeta de red con la que vamos a trabajar. Para ello, en el menú superior de Ettercap seleccionaremos «Sniff > Unified Sniffing» y, cuando nos lo pregunte, seleccionaremos nuestra tarjeta de red (por ejemplo, en nuestro caso, eth0).

IMAGEN Kali Linux - Ettercap - Tarjeta de red

El siguiente paso es buscar todos los hosts conectados a nuestra red local. Para ello, seleccionaremos «Hosts» del menú de la parte superior y seleccionaremos la primera opción, «Hosts List«.

IMAGEN Kali Linux - Ettercap - Lista de hosts

Aquí deberían salirnos todos los hosts o dispositivos conectados a nuestra red. Sin embargo, en caso de que no salgan todos, podemos realizar una exploración completa de la red simplemente abriendo el menú «Hosts» y seleccionando la opción «Scan for hosts«. Tras unos segundos, la lista de antes se debería actualizar mostrando todos los dispositivos, con sus respectivas IPs y MACs, conectados a nuestra red.

IMAGEN Kali Linux - Ettercap - Lista de hosts 2

\subsubsection{Nuestro Ettercap ya está listo. Ya podemos empezar con el ataque ARP Poisoning}

En caso de querer realizar un ataque dirigido contra un solo host, por ejemplo, suplantar la identidad de la puerta de enlace para monitorizar las conexiones del iPad que nos aparece en la lista de dispositivos, antes de empezar con el ataque debemos establecer los dos objetivos.

Para ello, debajo de la lista de hosts podemos ver tres botones, aunque nosotros prestaremos atención a los dos últimos:

    Target 1 – Seleccionamos la IP del dispositivo a monitorizar, en este caso, el iPad, y pulsamos sobre dicho botón.
    Target 2 – Pulsamos la IP que queremos suplantar, en este caso, la de la puerta de enlace.

IMAGEN Objetivos Ettercap

Todo listo. Ahora solo debemos elegir el menú «MITM» de la parte superior y, en él, escoger la opción «ARP Poisoning«.

IMAGEN Kali Linux - Ettercap - Ataques MITM

Nos aparecerá una pequeña ventana de configuración, en la cual debemos asegurarnos de marcar «Sniff Remote Connections«.

IMAGEN Comenzar MITM ARP Poisoning

Pulsamos sobre «Ok» y el ataque dará lugar. Ahora ya podemos tener el control sobre el host que hayamos establecido como «Target 1«. Lo siguiente que debemos hacer es, por ejemplo, ejecutar Wireshark para capturar todos los paquetes de red y analizarlos en busca de información interesante o recurrir a los diferentes plugins que nos ofrece Ettercap, como, por ejemplo, el navegador web remoto, donde nos cargará todas las webs que visite el objetivo.

Plugins Ettercap


\chapter{Virtualización} \pagenumbering{arabic}
    \label{capVirtu}


Luego de haber explicado las debilidades del protocolo \emph{HTTP}, vamos a explorar las 
distintas soluciones encontradas, empezando con un pequeño marco teórico, el cual 
nos permitirá probar estas implementaciones y aplicarlas en un escenario de prueba.

La herramienta utilizada es \emph{Docker}, con ella es posible simular 
escenarios sin necesidad de tener cada uno de los servidores de manera física. Para
entenderlo un poco mejor, explicaremos lo que es la virtualización y profundizaremos 
en la virtualización basada en contenedores.
    
\section{Máquinas virtuales}

La virtualización brinda la capacidad de ejecutar aplicaciones, sistemas 
operativos o servicios del sistema en un entorno distinto. 
Obviamente, los mismos tienen que estar ejecutándose en un determinado 
sistema informático en un momento dado, pero la virtualización proporciona 
un nivel de abstracción lógica que libera las aplicaciones, los servicios 
del sistema e incluso el sistema operativo de estar vinculado a una 
pieza específica de \emph{hardware}. El enfoque de la virtualización hace 
que todo esto sea portátil a través de diferentes sistemas informáticos 
físicos.

El enfoque basado en \emph{VM} virtualiza el sistema operativo completo. Con esto nos referimos 
a la virtualización de discos, \emph{CPU} y \emph{NIC}. En otras palabras, podemos afirmar que se 
trata de virtualizar la arquitectura de conjunto de instrucciones completa, como ejemplo, 
la arquitectura \emph{x86}. 

\begin{center}
    \begin{figure}   
       \begin{center}
          \includegraphics[width=12cm,height=7cm]{vm.png}
       \end{center}
       \caption{Virtualización}
    \end{figure}
 \end{center}

Para virtualizar un sistema operativo, se utiliza un \emph{software} especial, denominado
\emph{hipervisor}, y es el encargado, entre otras cosas, de aislar el sistema 
operativo de las máquinas virtuales y 
permitir crearlas y gestionarlas.

Una \emph{VM} debe satisfacer tres propiedades:
\begin{itemize}
    \setlength\itemsep{-0.6em}
    \item Aislamiento: debe aislar a los invitados entre sí. 
    \item Equivalencia: debe comportarse igual, con o sin virtualización. Esto 
    significa que se deben ejecutar la mayoría de las instrucciones en el \emph{hardware} 
    físico sin ninguna traducción hacia el \emph{hardware} virtual.
    \item Rendimiento: Debería funcionar tan bien como lo hace sin ninguna 
    virtualización. Esto nuevamente significa que la sobrecarga de ejecutar 
    una VM debe ser mínima.
\end{itemize}

\section{Virtualización basada en Contenedores}

Esta forma de virtualización no abstrae el \emph{hardware}, sino que utiliza técnicas 
dentro del \emph{kernel} de Linux para aislar las rutas de acceso para diferentes recursos. 
Establece un límite lógico dentro del mismo sistema operativo. Como resultado, obtenemos 
un sistema de archivos raíz separado, un árbol de procesos separado, un subsistema 
de red separado, etc.

\begin{center}
    \begin{figure}   
       \begin{center}
          \includegraphics[width=12cm,height=7cm]{contenedores.png}
       \end{center}
       \caption{Virtualización basada en Contenedores}
    \end{figure}
 \end{center}

El \emph{kernel} de Linux se compone de varios componentes y funcionalidades; los relacionados 
con contenedores son los siguientes:
\begin{itemize}
    \setlength\itemsep{-0.6em}
    \item Grupos de control (\emph{Cgroups})
    \item Espacios de nombres (Namespaces)
    \item Linux con seguridad mejorada (SELinux)
\end{itemize}

\subsection*{Cgroups}
La funcionalidad de \emph{cgroup} permite limitar y priorizar recursos, como \emph{CPU}, RAM, 
la red, el sistema de archivos, etc. El objetivo principal es no exceder los 
recursos, para evitar desperdiciar recursos que podrían ser necesarios para 
otros procesos.

\subsection*{Espacios de nombres}
La funcionalidad del espacio de nombres permite particionar los recursos del kernel, 
de modo que un conjunto de procesos ve un conjunto de recursos, mientras que otro 
conjunto de procesos ve un conjunto diferente de recursos. La característica funciona 
al tener el mismo espacio de nombres para estos recursos en los distintos conjuntos 
de procesos, pero esos nombres se refieren a recursos distintos. Algunos nombres 
de recursos que pueden existir en varios espacios son los ID de proceso, nombres 
de host, ID de usuario, nombres de archivo y algunos nombres asociados con el 
acceso a la red y la comunicación entre procesos. 

Cuando se inicia un sistema 
Linux, solo se crea un espacio de nombres. Los procesos y recursos se unirán 
al mismo espacio de nombres, hasta que se cree un espacio de nombres diferente, 
se le asignen recursos y los procesos se unan a él.

\subsection*{SELinux}
SELinux es un módulo del \emph{kernel} de Linux que proporciona un mecanismo para hacer 
cumplir la seguridad del sistema, con políticas específicas. Básicamente, SELinux 
puede limitar el acceso de los programas a archivos y recursos de red. La idea es 
limitar los privilegios de los programas y demonios al mínimo, de modo que pueda 
limitar el riesgo de que el sistema se detenga.

Lo contenedores utilizan los recursos directamente y no necesitan de un emulador en 
absoluto, cuantos menos recursos, mayor eficiencia. Se pueden ejecutar diferentes 
aplicaciones en el mismo host: aisladas a nivel de \emph{kernel} y aisladas por 
espacios de nombres y \emph{cgroups}. El \emph{kernel}, es decir, el Sistema Operativo,
 lo comparten todos los contenedores.

\subsection*{Contenedores}
Cuando hablamos de contenedores, nos referimos indirectamente a dos conceptos 
principales: una imagen de contenedor y una imagen de contenedor en ejecución. 
Una imagen de contenedor es la definición del contenedor, en donde el \emph{software} 
restante se instala como capas adicionales, como se muestra en el diagrama.

\begin{center}
    \begin{figure}   
       \begin{center}
          \includegraphics[width=7.5cm,height=5cm]{contenedores-capas.png}
       \end{center}
       \caption{Capas de contenedores}
       \label{capasContenedores}
    \end{figure}
 \end{center}

Una imagen de contenedor suele estar formada por varias capas (figura \ref{capasContenedores}).
La primera capa está dada por la imagen base, que proporciona las funcionalidades 
centrales del sistema operativo, con todas las herramientas necesarias para comenzar. 
Los equipos de desarrolladores a menudo trabajan construyendo sus propias capas sobre 
estas imágenes base. Los usuarios también pueden crear imágenes de aplicaciones más 
avanzadas, que no solo tienen un sistema operativo, sino que también incluyen 
herramientas de depuración y bibliotecas.

Los contenedores brindan aislamiento al aprovechar las tecnologías del \emph{kernel}, como 
\emph{cgroups}, espacios de nombres del \emph{kernel} y SELinux. Dado que  
utilizan un \emph{kernel} compartido y un host de contenedor, se reduce la cantidad de 
recursos necesarios para el contenedor en sí y son más livianos en comparación 
con las máquinas virtuales. 

Por lo mencionado anteriormente, podemos afirmar que los contenedores brindan una agilidad 
que no es factible con las \emph{VM}, y una prueba de esto es que solo se 
necesitan unos segundos para iniciar un nuevo contenedor. 
Además, los mismos admiten un 
modelo más flexible en lo que respecta a la utilización de la \emph{CPU} y los recursos 
de memoria, y permiten modos de ráfaga de recursos, donde las aplicaciones 
pueden consumir más recursos cuando es requerido, dentro de los límites definidos.


\section{\emph{Docker}}

\emph{Docker} es una herramienta de código abierto que automatiza la implementación de aplicaciones 
en contenedores. Fue escrito por el equipo de \emph{Docker} y publicado por ellos bajo la 
licencia Apache 2.0. Está diseñado para proporcionar 
un entorno ligero y rápido para implementar escenarios determinados, así como un flujo de 
trabajo 
eficiente para llevar ese código desde una computadora portátil a su entorno de prueba 
y luego a producción. De hecho, se puede comenzar con \emph{Docker} en un host mínimo que 
no ejecute nada más que un \emph{kernel} de Linux compatible y un binario de \emph{Docker}.



\subsection{Imágenes}
Las imágenes son los componentes básicos del mundo de \emph{Docker}. El uso común es 
lanzando los contenedores a partir de imágenes. Tienen un formato en capas, que 
se forman paso a paso utilizando una serie de instrucciones.

Se puede considerar a las imágenes como el “código fuente” de los contenedores. 
Son muy portátiles y se pueden compartir, almacenar y actualizar. .

\subsection{Registros}
\emph{Docker} almacena las imágenes que se crean en registros. Hay dos tipos de 
registros: públicos y privados. La herramienta opera el registro público de imágenes, 
llamado \emph{Docker} Hub. Se puede crear una cuenta y usarla para 
compartir y almacenar nuestras propias imágenes. También es posible almacenar 
las imágenes que desee y mantenerlas privadas en \emph{Docker} Hub. Estas imágenes 
pueden incluir código fuente u otra información de propiedad que se quiera 
mantener segura o solo compartir con otros miembros de su equipo u organización.

\subsection{Contenedores}
El \emph{software} permite construir e implementar contenedores dentro de los cuales se puede 
empaquetar aplicaciones y servicios. Como acabamos de mencionar, los contenedores 
se lanzan a partir de imágenes y estos pueden contener uno o más procesos en ejecución.

\noindent Un contenedor es:
\begin{itemize}
    \setlength\itemsep{-0.6em}
    \item Un formato de imagen.
    \item Un conjunto de operaciones estándar.
    \item Un entorno de ejecución.
\end{itemize}

Se puede hacer una analogía entre los contenedores de \emph{Docker} y los contenedores 
de envío estándar, utilizado para transportar mercancías a nivel mundial. 
En lugar de enviar mercancías, los contenedores de \emph{Docker} envían \emph{software}. 
Cada contenedor contiene una imagen de \emph{software}, su “carga”, y, al igual 
que su contraparte física, permite realizar un conjunto de operaciones. 
Por ejemplo, se puede crear, iniciar, detener, reiniciar y destruir. Como 
un contenedor de envío, \emph{Docker} no se preocupa por el contenido del contenedor 
cuando realiza estas acciones; por ejemplo, si un contenedor es un servidor \emph{web}, 
una base de datos o un servidor de aplicaciones. Cada contenedor se carga 
igual que cualquier otro. A \emph{Docker} tampoco le importa dónde envía su contenedor: 
se puede compilar en una computadora portátil, cargarlo en un registro, luego 
descargarlo en un servidor físico o virtual, probarlo, implementarlo en un 
clúster de una docena de \emph{hosts} y ejecutarlo. 




\chapter{Herramienta Utilizadas} 
    \label{capTools}

    \section{Kali Linux}
    BackTrack is one of the most famous Linux distribution systems, as can be proven by
the number of downloads that reached more than four million as of BackTrack Linux
4.0 pre final.

Kali Linux Version 1.0 was released on March 12, 2013. Five days later, Version 1.0.1
was released, which fixed the USB keyboard issue. In those five days, Kali has been
downloaded more than 90,000 times.

    Kali Linux is security-focused Linux distribution based on Debian. It's a rebranded
    version of the famous Linux distribution known as Backtrack, which came with
    a huge repository of open source hacking tools for network, wireless, and web
    application penetration testing. Although Kali Linux contains most of the tools
    from Backtrack, the main aim of Kali Linux is to make it portable so that it could
    be installed on devices based on the ARM architectures such as tablets and
    Chromebook, which makes the tools available at your disposal with much ease.

    Using open source hacking tools comes with a major drawback: they contain a
whole lot of dependencies when installed on Linux and they need to be installed in
a predefined sequence. Moreover, authors of some tools have not released accurate
documentation, which makes our life difficult.

Kali Linux simplifies this process; it contains many tools preinstalled with all the
dependencies and is in ready to use condition so that you can pay more attention
for the actual attack and not on installing the tool. Updates for tools installed in Kali
Linux are more frequently released, which helps you to keep the tools up to date. A
non-commercial toolkit that has all the major hacking tools preinstalled to test realworld networks and applications is a dream of every ethical hacker and the authors
of Kali Linux make every effort to make our life easy, which enables us to spend
more time on finding the actual flaws rather than building a toolkit.

(EXPLICAR COMO Y DONDE LO INSTALE)
(DECIR QUE FUE EL SISTEMA OPERATIVO UTILIZADO QUE CONTIENEN LAS SIGUIENTES HERRAMIENTAS)
\section{Wireshark}
Wireshark is one of the most popular, free, and open source network protocol
analyzers. Wireshark is preinstalled in Kali and ideal for network troubleshooting,
analysis, and for this chapter, a perfect tool to monitor traffic from potential targets
with the goal of capturing session tokens. Wireshark uses a GTK+ widget toolkit to
implement its user interface and pcap to capture packets. It operates very similarly to
a tcpdump command; however, acting as a graphical frontend with integrated sorting
and filtering options.
(HAY MAS, CON GRAFICOS EN)
Joseph Muniz, Aamir Lakhani - Web Penetration Testing with Kali Linux-Packt Publishing (2013)

    \section{Ettrcap}
    Ettercap is a free and open source comprehensive suite for man-in-the-middle-based
attacks.
Ettercap can be used for computer network protocol analysis and security auditing,
featuring sniffing live connections, content filtering, and support for active and passive
dissection of multiple protocols. Ettercap works by putting the attacker's network
interface into promiscuous mode and ARP for poisoning the victim machines.



\chapter{Caso de estudio: Sniffing de la red para obtener credenciales } 
    \label{capCaseOfStudy}

\subsection{Herramienta Utilizadas} 
%    \label{capTools}

    \section{Kali Linux}
    BackTrack is one of the most famous Linux distribution systems, as can be proven by
the number of downloads that reached more than four million as of BackTrack Linux
4.0 pre final.

Kali Linux Version 1.0 was released on March 12, 2013. Five days later, Version 1.0.1
was released, which fixed the USB keyboard issue. In those five days, Kali has been
downloaded more than 90,000 times.

    Kali Linux is security-focused Linux distribution based on Debian. It's a rebranded
    version of the famous Linux distribution known as Backtrack, which came with
    a huge repository of open source hacking tools for network, wireless, and web
    application penetration testing. Although Kali Linux contains most of the tools
    from Backtrack, the main aim of Kali Linux is to make it portable so that it could
    be installed on devices based on the ARM architectures such as tablets and
    Chromebook, which makes the tools available at your disposal with much ease.

    Using open source hacking tools comes with a major drawback: they contain a
whole lot of dependencies when installed on Linux and they need to be installed in
a predefined sequence. Moreover, authors of some tools have not released accurate
documentation, which makes our life difficult.

Kali Linux simplifies this process; it contains many tools preinstalled with all the
dependencies and is in ready to use condition so that you can pay more attention
for the actual attack and not on installing the tool. Updates for tools installed in Kali
Linux are more frequently released, which helps you to keep the tools up to date. A
non-commercial toolkit that has all the major hacking tools preinstalled to test realworld networks and applications is a dream of every ethical hacker and the authors
of Kali Linux make every effort to make our life easy, which enables us to spend
more time on finding the actual flaws rather than building a toolkit.

(EXPLICAR COMO Y DONDE LO INSTALE)
(DECIR QUE FUE EL SISTEMA OPERATIVO UTILIZADO QUE CONTIENEN LAS SIGUIENTES HERRAMIENTAS)
\section{Wireshark}
Wireshark is one of the most popular, free, and open source network protocol
analyzers. Wireshark is preinstalled in Kali and ideal for network troubleshooting,
analysis, and for this chapter, a perfect tool to monitor traffic from potential targets
with the goal of capturing session tokens. Wireshark uses a GTK+ widget toolkit to
implement its user interface and pcap to capture packets. It operates very similarly to
a tcpdump command; however, acting as a graphical frontend with integrated sorting
and filtering options.
(HAY MAS, CON GRAFICOS EN)
Joseph Muniz, Aamir Lakhani - Web Penetration Testing with Kali Linux-Packt Publishing (2013)

    \section{Ettrcap}
    Ettercap is a free and open source comprehensive suite for man-in-the-middle-based
attacks.
Ettercap can be used for computer network protocol analysis and security auditing,
featuring sniffing live connections, content filtering, and support for active and passive
dissection of multiple protocols. Ettercap works by putting the attacker's network
interface into promiscuous mode and ARP for poisoning the victim machines.



(HAY MAS, CON GRAFICOS EN)
Joseph Muniz, Aamir Lakhani - Web Penetration Testing with Kali Linux-Packt Publishing (2013)
Aca yo segui un tutorial, buscarlo

La idea principal de esta sección es demostrar que, encontrandose en una red interna
y con con herramientas ya desarrolladas y libres es posible realizar un ataque 
sin necesidad de conocer a fondo la implementacion de la misma ni de tener mayores
privilegios

Recordar que esto fue realizado en una red interna donde son todos equipos de nuestra propiedad

\subsection{Diagrama de explicacion}
IMAGEN de le red

Tiene que estar:
-Router

-Origen de la pagina

-Consumidor de la pagina

-El atacante

\subsection{Preparando Ettercap para el ataque ARP Poisoning}

Lo primero que debemos hacer, en la lista de aplicaciones, es buscar el apartado 
«9. Sniffing y Spoofing«, ya que es allí donde encontraremos las herramientas necesarias
 para llevar a cabo este ataque.

IMAGEN Kali Linux Spoofing

A continuación, abriremos «Ettercap» y veremos una ventana similar a la siguiente.

IMAGEN  Kali Linux Ettercap

El siguiente paso es seleccionar la tarjeta de red con la que vamos a trabajar. Para ello, en el menú superior de Ettercap seleccionaremos «Sniff > Unified Sniffing» y, cuando nos lo pregunte, seleccionaremos nuestra tarjeta de red (por ejemplo, en nuestro caso, eth0).

IMAGEN Kali Linux - Ettercap - Tarjeta de red

El siguiente paso es buscar todos los hosts conectados a nuestra red local. Para ello, seleccionaremos «Hosts» del menú de la parte superior y seleccionaremos la primera opción, «Hosts List«.

IMAGEN Kali Linux - Ettercap - Lista de hosts

Aquí deberían salirnos todos los hosts o dispositivos conectados a nuestra red. Sin embargo, en caso de que no salgan todos, podemos realizar una exploración completa de la red simplemente abriendo el menú «Hosts» y seleccionando la opción «Scan for hosts«. Tras unos segundos, la lista de antes se debería actualizar mostrando todos los dispositivos, con sus respectivas IPs y MACs, conectados a nuestra red.

IMAGEN Kali Linux - Ettercap - Lista de hosts 2

\subsection{Nuestro Ettercap ya está listo. Ya podemos empezar con el ataque ARP Poisoning}

En caso de querer realizar un ataque dirigido contra un solo host, por ejemplo, suplantar la identidad de la puerta de enlace para monitorizar las conexiones del iPad que nos aparece en la lista de dispositivos, antes de empezar con el ataque debemos establecer los dos objetivos.

Para ello, debajo de la lista de hosts podemos ver tres botones, aunque nosotros prestaremos atención a los dos últimos:

    Target 1 – Seleccionamos la IP del dispositivo a monitorizar, en este caso, el iPad, y pulsamos sobre dicho botón.
    Target 2 – Pulsamos la IP que queremos suplantar, en este caso, la de la puerta de enlace.

IMAGEN Objetivos Ettercap

Todo listo. Ahora solo debemos elegir el menú «MITM» de la parte superior y, en él, escoger la opción «ARP Poisoning«.

IMAGEN Kali Linux - Ettercap - Ataques MITM

Nos aparecerá una pequeña ventana de configuración, en la cual debemos asegurarnos de marcar «Sniff Remote Connections«.

IMAGEN Comenzar MITM ARP Poisoning

Pulsamos sobre «Ok» y el ataque dará lugar. Ahora ya podemos tener el control sobre el host que hayamos establecido como «Target 1«. Lo siguiente que debemos hacer es, por ejemplo, ejecutar Wireshark para capturar todos los paquetes de red y analizarlos en busca de información interesante o recurrir a los diferentes plugins que nos ofrece Ettercap, como, por ejemplo, el navegador web remoto, donde nos cargará todas las webs que visite el objetivo.

Plugins Ettercap




\chapter{Mejorando la seguridad en la navegación} 
    \label{capImp}

%-*- ES -*-
%----------------------------------------------------------------------
% Capitulo: Implementacion de la solución propuesta
%----------------------------------------------------------------------

\section{Propuestas: Introducción}

Se han investigado tres alternativas enfocadas en redes internas para mejorar la seguridad de las 
mismas, luego de eso se eligió la que más ventajas nos ofreció.
Las alternativas son las siguientes: Los certificados auto-firmados (\emph{Self-signed Certificates}), implementar
una entidad certificante interna, y la utilización de un certificado emitido por una entidad
certificante conocida.

\section{Propuesta 1: Self-signed Certificates}
Los certificados autofirmados son los menos útiles de los tres. Firefox facilita su uso 
de forma segura; crea una excepción en la primera visita, después de lo cual el 
certificado autofirmado se considera válido en las conexiones posteriores. Otros 
navegadores hacen que haga clic en una advertencia de certificado cada vez. A menos 
que esté comprobando la huella digital del certificado cada vez, no es posible hacer 
que ese certificado autofirmado sea seguro. Incluso con Firefox, puede resultar 
difícil utilizar estos certificados de forma segura.

\begin{center}
   \begin{figure}   
      \begin{center}
         \includegraphics[width=13cm,height=9cm]{self-signed.png}
      \end{center}
      \caption{Advertencia Certificado Autofirmado}
   \end{figure}
\end{center}

Por ejemplo, para solicitar un certificado SSL de una CA de confianza como Verisign o 
GoDaddy, se debe enviar una Solicitud de firma de certificado (CSR) y te dan un 
certificado a cambio, que firmaron con su certificado raíz y clave privada. Todos 
los navegadores tienen una copia (o acceden a una copia desde el sistema operativo) 
del certificado raíz, por lo que el navegador puede verificar que su certificado 
fue firmado por una CA confiable.

Cuando generamos un certificado autofirmado, generamos nuestro propio certificado 
raíz y clave privada. Debido a que genera un certificado autofirmado, el navegador 
no confía en él. Está autofirmado. No ha sido firmado por una CA. Todos los 
certificados que generamos y firmamos serán de confianza inherente.

La principal dificultad es que los usuarios siempre encontrarán una advertencia 
donde el navegador diga que se encuentra en un sitio con un certificado autofirmado. 
En la mayoría de los casos, no verificarán que el certificado es el correcto, por 
lo que generará desconfianza en los usuarios.

En prácticamente todos los casos, un enfoque mucho mejor es utilizar una CA privada, 
que es nuestra próxima propuesta. Requiere un poco más de trabajo por adelantado, 
pero una vez que la infraestructura está establecida y la clave raíz se distribuye 
de manera segura a todos los usuarios, dichas implementaciones son tan seguras como 
el resto del ecosistema PKI.

\section{Propuesta 2: Internal CA}

Como se explicó anteriormente, una entidad de certificación es un agente en el que 
confiamos para emitir 
certificados que confirman las identidades de los suscriptores, o bien de los 
sitios a los cuales visitamos. 

Esta propuesta de solución implica establecer una entidad de certificación interna 
a la red privada, y por consiguiente hacer que los equipos que se encuentran 
dentro de la misma confíen el ella. Esto se hace 
mediante un servidor dedicado que firme los certificados que se le solicitan.

Ventajas de la autoridad de certificación interna (CA)
\begin{itemize}
   \setlength\itemsep{-0.6em}
   \item No es necesario depender de una entidad externa para los certificados.
   \item En un entorno de Microsoft Windows, la Autoridad de certificación (CA) interna se puede 
   integrar en Active Directory. Esto simplifica aún más la gestión de la estructura de la CA.
   \item No hay ningún costo por certificado cuando utiliza una Autoridad de certificación (CA) 
   interna.
\end{itemize}

Desventajas de la autoridad certificadora (CA) interna

\begin{itemize}
   \setlength\itemsep{-0.6em}
   \item Implementar una autoridad certificadora (CA) interna es más complicado que utilizar una 
   autoridad certificadora (CA) externa.
   \item La seguridad y la responsabilidad de la infraestructura de clave pública (PKI) está 
   completamente a cargo de la organización.
   \item Los usuarios externos que se conecten a nuestra red normalmente no confiarán en un 
   certificado digital 
   firmado por una Autoridad de Certificación (CA) interna.

\end{itemize}

Pese a que esta propuesta es de las más implementadas, decidimos buscar una opción que nos 
permita desligarnos de ciertas responsabilidades, como así también no tener que realizar 
configuraciones individuales a los hosts de nuestra organización. 

\subsection{Caso de estudio: Creando nuestra Entidad Certificante privada}
\subsubsection*{Servidor DNS con Docker}

Para montar esta propuesta de solución, se debió crear un servidor DNS, que lo que hace
es básicamente mapear direcciones IP a nombres de dominio. Por ejemplo: nuestra página que 
se encuentra en la dirección \textit{192.168.0.202} se pasará a llamar pagina1.salvadorcatalfamo.intra, 
donde salvadorcatalfamo.intra abarca toda nuestra organización y las páginas que se encuentren
dentro de ella. Otra razón por la cual se debe 
crear un servidor DNS, es que los certificados no se pueden otorgar a direcciones IP, por lo 
cual es un requisito obligatorio tener las páginas web direccionadas con nombres de dominio.

Se eligió el servidor DNS CoreDNS ya que es amigable con Docker; sucede que, por cada versión del 
programa, se generan las imágenes de Docker correspondientes. Estas imágenes son públicas y oficiales, 
lo que da confiabilidad y seguridad extra a la hora de utilizarlas. La configuración está formada por 
los siguientes componentes: los archivos de configuración de CoreDNS (corefile) y los sitios que 
nosotros deseemos, en nuestro caso \textit{pagina1.salvadorcatalfamo.intra}

\noindent El Corefile es el archivo de configuración de CoreDNS. Este define:
\begin{itemize}
    \setlength\itemsep{-0.6em}
    \item Qué servidores escuchan en que puertos y que protocolo.
    \item Para qué zona tiene autoridad cada servidor.
    \item Qué \emph{plugins} (complementos) se cargan en un servidor.
\end{itemize}

\noindent El formato es el siguiente
\begin{verbatim}
    ZONE: [PORT] {
        [PLUGIN] ...
    }
\end{verbatim}

\noindent ZONE: define la zona de este servidor. El puerto por defecto es el \textsf{53}, o bien el valor que se le indique 
con el flag \textsf{-dns.port}.

\noindent PLUGIN: define los complementos que queremos cargar. Cada plugin puede tener varias propiedades, por 
lo que también podrían tener argumentos de entrada.

\noindent Nuestro archivo de configuración es el siguiente:
\begin{verbatim}
    .:53 {
        forward . 8.8.8.8 9.9.9.9
        log
    }

    salvadorcatalfamo.intra:53 {
        file /etc/coredns/salvadorcatalfamo
        log
        reload 10s
    }    
\end{verbatim}

A grandes rasgos, lo que indica esta configuración es que va a existir una zona 
“salvadorcatalfamo.intra”, que estará definida por el archivo que se encuentra en 
\textit{/etc/coredns/salvadorcatalfamo}. Por otro lado, el tráfico restante (dominios 
externos a nuestra red) será fordwardeado a 
servidores DNS externos (\textit{8.8.8.8} y \textit{9.9.9.9}). Además se establecieron algunos plugins de logeo y 
de refresco de configuración.

Nuestro archivo \textit{/etc/coredns/salvadorcatalfamo} contiene la siguiente información.
\begin{verbatim}
salvadorcatalfamo.intra.          IN  SOA ns1.salvadorcatalfamo.intra. ...
pagina1.salvadorcatalfamo.intra.  IN  A   192.168.0.124   
\end{verbatim}

Esto en principio es suficiente para nuestro sitio interno, y contiene las direcciones ip de los 
servidores web y entidades certificantes.

Por el lado de Docker, se utilizó un archivo Docker-compose.yml, y un Dockerfile. 
Docker-compose es una herramienta para definir y ejecutar aplicaciones Docker de 
varios contenedores. Compose usa un archivo YAML para configurar los servicios de 
la aplicación. Luego, con un solo comando, se crean e inician todos los servicios 
desde este archivo de configuración. En nuestro caso, se definió de la siguiente 
manera

\begin{verbatim}
version: `3.1'
services:
    coredns:
        build: .
        container_name: coredns
        restart: always
        expose:
            - `53'
            - `53/udp'
        ports:
            - `53:53'
            - `53:53/udp'
        volumes:
            - `./config:/etc/coredns'    
\end{verbatim}

\noindent Por otro lado, el fichero dockerfile está compuesto por las siguientes líneas:
\begin{verbatim}
FROM coredns/coredns:1.7.0
ENTRYPOINT ["/coredns"]
CMD ["-conf", "/etc/coredns/Corefile"]    
\end{verbatim}

En conjunto, establecen la imagen de CoreDNS que se utilizará, los archivos de configuración y
los puertos que se expondrán, entre otras configuraciones.

\subsubsection*{Creación de nuestra CA en \emph{Docker}}

La estrategia para crear nuestra \emph{CA} será seguir los pasos que se deberían realizar en un servidor 
habitual, pero partiendo desde una imagen de Docker de Ubuntu (de stock), y luego realizando un 
commit de estas configuraciones. Luego, archivos importantes como el certificado root y la llave
privada deberán resguardarse, o simplemente resguardar el contenedor creado. 

Se utilizó esta estrategia ya que no había imágenes oficiales que nos sirva para tal fin, por el 
simple hecho de que únicamente se requiere tener instalado OpenSSL y tenerlo configurado.

\noindent Como primer paso, corremos una imagen del sistema operativo Ubuntu

\begin{verbatim}
    docker run -it -v $PWD/ca:/root/ca ubuntu
\end{verbatim}

\noindent Dentro del contenedor, ejecutamos los siguientes comandos:
\begin{verbatim}
    apt-get update
    apt-get install ntp
    apt-get install openssl
\end{verbatim}

\noindent Establecer el hostname al contenedor, hay una línea con la ip del contenedor y el nombre del mismo, 
que es utilizado como hostname, en nuestro caso
\begin{verbatim}
    172.17.0.2      080dec560726
\end{verbatim}

\noindent Lo cambiamos por un hostname con el dominio incluido
\begin{verbatim}
    172.17.0.2      ca.salvadorcatalfamo.intra
\end{verbatim}

\noindent Creamos las carpetas para mejor organización
\begin{verbatim}
    mkdir /root/newcerts
    mkdir /root/certs
    mkdir /root/crl
    mkdir /root/private
    mkdir /root/requests
\end{verbatim}

\noindent Creamos un archivo vacío y un archivo que contiene el primer número de serie para los certificados 
\begin{verbatim}
    touch index.txt
    echo '1234' > serial
\end{verbatim}

\noindent Luego hay que crear la llave privada y el certificado root, en este caso nos pedirá una contraseña, si este servidor 
se usará en un ambiente de producción, deberá ser una contraseña compleja.

\begin{verbatim}
    openssl genrsa -aes256 -out private/cakey.pem

\end{verbatim}

\noindent Una vez que generamos la llave privada, la misma será utilizada como entrada en la creación de 
nuestro certificado root. Nos pedirá algunos datos de localización y relacionados a la 
organización

\begin{verbatim}
    openssl req -new -x509 -key /root/ca/private/cakey.pem 
            -out cacert.pem -days 3650
\end{verbatim}

\noindent Cambiamos los permisos de los archivos que creamos
\begin{verbatim}
    chmod 600 -R /root/ca
\end{verbatim}

\noindent Realizamos unas modificaciones en el archivo de configuración, donde indicaremos 
la dirección de los certificados, y algunas opciones de configuración adicionales
\begin{verbatim}
    vim /usr/lib/ssl/openssl.cnf
\end{verbatim}

Una vez que realizamos estos pasos, estamos listos para realizar el commit de la imagen, con esto,
todos los pasos que realizamos (instalar los paquetes, modificar los archivos de configuración, etc)
no son necesarios que se ejecuten nuevamente.

Para realizar un commit, y que nuestro contenedor sea fácilmente identificable, deberemos seguir 
los siguientes pasos

\begin{verbatim}
    docker ps -a        #identificamos el último contenedor utilizado
    docker commit {id_del_contenedor} 
    docker image ls     #Identificamos la imagen recién creada
                        #no tendrá ni repositorio ni tag
    docker image tag {id_de_imagen} ca:1.0  #nombramos nuestra imagen
\end{verbatim}




Ahora cada vez que queramos correr nuestra \emph{CA}, lo haremos de la siguiente manera:
\begin{verbatim}
    docker run -it -v $PWD/ca:/root/ca ca:1.0
\end{verbatim}

Em el comando anterior, estamos asumiendo que queremos compartir los archivos de la \emph{CA} con el servidor 
\emph{host}.

 

\subsubsection*{Nginx con Docker}

Para probar nuestro certificado, utilizaremos una imagen oficial de Nginx, 
los archivos de configuración
son los siguientes:


\begin{verbatim}
docker-compose.yml

web:
  image: nginx
  volumes:
   - ./pagina1:/usr/share/nginx/html:ro
  ports:
   - "80:80"
  environment:
   - NGINX_HOST=pagina1.salvadorcatalfamo.intra
   - NGINX_PORT=80
\end{verbatim}

En el archivo mostrado, le decimos a Docker que utilice la imagen Nginx, 
que nuestros archivos fuentes
van a estar en el directorio ./pagina1 y que exponga el puerto 80, entre otras
 cosas.

\noindent Para ejecutar este contenedor, se debe ejecutar el siguiente comando:

\begin{verbatim}
    docker-compose up -d
\end{verbatim}




\subsubsection*{Creando nuestro certificado}

Para firmar un certificado, tenemos que seguir los pasos mostrados en la figura \ref{figSolCert}
: el servidor donde se alojará la web debe realizar una solicitud, 
luego nuestra \emph{CA} retornará el certificado firmado. Desde el servidor web en cuestión, se debe 
crear una llave privada:

\begin{verbatim}
    openssl genrsa -aes256 -out pagina1.pem 2048
\end{verbatim}

\noindent Luego, se deberá crear la solicitud de firma de certificado:
\begin{verbatim}
    openssl req -new -key pagina1.pem -out pagina1.csr
\end{verbatim}

\noindent Luego enviamos esta solicitud y la firmamos en nuestra \emph{CA}, esta solicitud la vamos a colocar 
en el directorio \textit{/root/ca/requests}
\begin{verbatim}
    openssl ca -in pagina1.csr -out pagina1.crt
\end{verbatim}

\subsubsection*{Configuración del certificado en el servidor web}
Una vez que tenemos este certificado, lo colocamos en el servidor web y 
lo configuramos. Para el caso de Nginx, se debe editar el archivo de configuración 
correspondiente a nuestra web, que en este caso es \textit{pagina1.salvadorcatalfamo.intra}
con el fin de que el mismo pueda localizar correctamente los certificados firmados recientemente.
Adicionalmente, se puede obligar a que cada requerimiento sea redirigido a una conexión
segura mediante \emph{SSL}.

Como se puede ver en la figura \ref{figCAdesc}, aunque el certificado esté configurado, 
el mismo no es confiable ya que nuestra entidad certificante no está importada a nuestro almacén 
local.

\begin{center}
    \begin{figure}   
       \begin{center}
          \includegraphics[width=15cm,height=6cm]{adv-2.png}
       \end{center}
       \caption{CA Desconocida}
       \label{figCAdesc}
    \end{figure}
 \end{center}

Luego de establecer en nuestra computadora (particularmente en el navegador Mozilla) que la 
entidad certificante creada es confiable, es posible ver que nuestra conexión es segura, como 
se muestra en la figura \ref{resulCA}.

\begin{center}
    \begin{figure}   
       \begin{center}
          \includegraphics[width=15cm,height=7.5cm]{ca-HTTPS.png}
       \end{center}
       \caption{Resultado de la solución}
       \label{resulCA}
    \end{figure}
 \end{center}







\section{Propuesta 3: Certificación con Let's Encrypt}
Esta estrategia consiste en generar un certificado wildcard, y utilizarlo en cada sitio de la organización.
Para esto se debe tener un dominio (en mi caso, salvadorcatalfamo.com) y demostrar la propiedad 
del mismo. Como se vio anteriormente, hay dos maneras que utiliza Let's Encrypt para demostrar la propiedad 
de un dominio, pero la que nos sirve en el caso de una red interna es la que intervienen los registros DNS. 
La verificación que ofrecen con este tipo de certificado es la mínima (DV) y el proceso es explicado a 
continuación.


\subsection{Pasos a seguir}
\subsubsection*{Obtener un dominio}
El primer paso es conseguir un dominio, en mi caso ya tenía uno:
salvadorcatalfamo.com. Este dominio apunta a nuestra ip pública. La configuración DNS
es la siguiente:  


\begin{longtable}{|l|l|p{5cm}|l|l|} 
   \hline
   \textbf{Tipo} & \textbf{Nombre} & \textbf{Contenido} & \textbf{Prioridad} & \textbf{TTL}
\\ \hline A  & salvadorcatalfamo.com & 181.228.121.12 & 0 & 14400
\\ \hline NS  & salvadorcatalfamo.com & ns1.donweb.com & 0 & 14400
\\ \hline SOA & salvadorcatalfamo.com & ns2.donweb.com & 0 & 14400
\\ \hline SOA & salvadorcatalfamo.com & ns3.hostmar.com \newline root.hostmar.com 
                                       \newline 2021010700 28800 7200 
                                       \newline 2000000 86400
                                       \newline ns2.donweb.com & 0 & 14400                  


\\ \hline
\end{longtable}

\subsubsection*{Instalando Let’s Encrypt en el servidor}
\begin{verbatim}
   sudo add-apt-repository ppa:certbot/certbotsudo 
   apt-get update
   sudo apt-get install python-certbot-nginx
\end{verbatim}

\subsubsection*{Instalando Nginx}
\begin{verbatim}
sudo apt-get update
sudo apt-get install nginx
\end{verbatim}


\subsubsection*{Obteniendo un certificado SSL de tipo wildcard desde Let’s Encrypt}
\begin{verbatim}
sudo certbot --server https://acme-v02.api.letsencrypt.org/directory 
-d *.salvadorcatalfamo.com --manual --preferred-challenges dns-01 certonly
\end{verbatim}

\subsubsection*{Configuración DNS}
Luego de ejecutar el comando anterior, Let's Encrypt nos da un contenido que 
se debe agregar a un registro DNS. El tipo de registro es TXT y se muestra en
la siguiente tabla.

\begin{longtable}{|l|l|l|l|l|} 
   \hline
   \textbf{Tipo} & \textbf{Nombre} & \textbf{Contenido} & \textbf{Prioridad} & \textbf{TTL}
\\ \hline TXT  & 	\_acme-challenge.salvadorcatalfamo.com & 11UZJD27bPDb\_jFs6f... & 0 & 14400
\\ \hline
\end{longtable}

\subsubsection*{Configuración de Nginx para servir a subdominios}

Se debe modificar el siguiente archivo de 
configuración 

/etc/nginx/sites-available/salvadorcatalfamo.com como se muestra a 
continuación:

\begin{verbatim}
server {
   listen 80;
   listen [::]:80;
   server_name *.salvadorcatalfamo.com;
   return 301 https://$host$request_uri;
}
server {
   listen 443 ssl;
   server_name *.example.com;  ssl_certificate /.../.../fullchain.pem;
   ssl_certificate_key /.../privkey.pem;
   include /etc/letsencrypt/options-ssl-nginx.conf;
   ssl_dhparam /.../ssl-dhparams.pem;  root /.../salvadorcatalfamo.com;
   index index.html;
   location / {
      try_files $uri $uri/ =404;
   }
} 
\end{verbatim}


\subsubsection*{Test la configuración y reinicio del servicio}

La configuración se puede testear con el siguiente comando: 
\begin{verbatim}
      sudo nginx -t
\end{verbatim}

Si tiene éxito, se debe volver a cargar Nginx usando
\begin{verbatim}
      sudo /etc/init.d/nginx reload
\end{verbatim}

Ahora Nginx contiene un certificado de tipo wildcard, un certificado SSL con respaldo de una 
entidad certificante como Let's Encrypt

\begin{center}
   \begin{figure}   
      \begin{center}
         \includegraphics[width=15cm,height=16cm]{lets.png}
      \end{center}
      \caption{Web interna con certificado de Let’s Encrypt}
   \end{figure}
\end{center}

Le hemos visto un gran potencial a esta solución, aunque también es poco implementada. Tenemos la gran ventaja de no
tener que instalar ningún requerimiento en las computadoras dentro de la organización. Con esto, no se 
mostrarán mensajes de seguridad en los navegadores, no importa cuál sea el programa, ya que Let's Encrypt es 
una entidad de confianza para diversos navegadores y sistemas operativos. Todo esto y la seguridad extra al 
saber que nuestros datos van por un canal seguro gracias al protocolo SSL. 

Otra gran ventaja que vimos es la facilidad con la que se llevó a cabo, 
en este proyecto se logró implementar antes esta solución que la entidad certificante, y 
con mucha mayor facilidad. 

Como desventajas podemos decir que no tenemos la posibilidad de obtener la validación extendida (EV), ya que no está
disponible actualmente. Por otro lado, que las llaves privadas y el certificado (que es único para todo el dominio) 
estén en diversos servidores a la vez, implica
que se deben tener mayores recaudos a la hora de utilizarlos, ya que se debe asegurar el control de éstos. Aunque a 
nosotros no nos sucedió, puede llegar a suceder que si se pierde conexión a Internet, el certificado no se pueda validar,
producto de no tener toda la cadena de certificación hasta llegar a la raíz. 

Aunque se puede llegar a pensar, administrar los certificados y las llaves puede llegar a ser 
un gran desafío para los administradores de sistema, sin embargo, hay muchas herramientas que nos proveen
automatización y monitoreo para realizar esta clase de tareas. 


\section{Caso de estudio: Buscando credenciales en tráfico seguro}

Luego de ver las diversas soluciones propuestas, una parte importante de 
nuestro proyecto fue verificar que verdaderamente aumenta la seguridad
cuando nuestro tráfico va encriptado. Para este caso de estudio, se utilizó
el mismo formulario propuesto en la sección \ref{secCaseOfStudy}, lo 
único que con servidores en distintas direcciones.

Dado que el proceso de capturar el tráfico en una red interna fue
explicado previamente, se van a mostrar únicamente los paquetes capturados
desde la primera solicitud hasta el envío del formulario.

\begin{center}
   \begin{figure}   
      \begin{center}
         \includegraphics[width=15cm,height=9cm]{verificacion-ssl-1-v2.png}
      \end{center}
      \caption{Tráfico capturado}
   \end{figure}
\end{center}

En esta captura podemos observar que:
\begin{itemize}
   \setlength\itemsep{-0.6em}
   \item No es posible determinar a simple vista cual es el paquete 
   en el cual se envía la información crítica.
   \item Abriendo e investigando el contenido de cada uno de los paquetes mostrados,
   tampoco es posible ver las credenciales completadas en el formulario, 
   que obviamente son de nuestro conocimiento.
   \item Se establece una conexión segura a través del protocolo TCP, algo que no 
   vimos en el caso de estudio de HTTP.
\end{itemize}







% =================================================================== %

\chapter{Conclusiones y Resultados Obtenidos}
    \label{capConc}

%-*- ES -*-
%----------------------------------------------------------------------
% Capitulo 7: Conclusiones generales
%----------------------------------------------------------------------

En este proyecto abordamos una amplia cantidad de temas de manera resumida, con un gran potencial 
de estudio por delante. La navegación \emph{web} nos permitió encontrar una 
extensa cantidad conceptos para desarrollar, tales como los protocolos \emph{HTTP}, \emph{HTTPS}, \emph{DNS}, 
y los posibles procesos que se pueden realizar para segurizar nuestra red.

Con respecto a \emph{Docker}, quiero destacar la amplia comunidad existente, ya que ésta 
nos brinda muchas facilidades, tales como ejemplo e instructivos que nos ayuda 
 a la hora de trabajar. El tiempo que 
nos ahorramos con \emph{Docker} fue destinado a entender el funcionamiento de la seguridad en las redes, 
tales como el uso de los certificados \emph{SSL} y las entidades certificantes.

A pesar del contenido teórico propuesto, el trabajo se concentró en las soluciones presentadas 
en el capítulo 4. Allí se explicó brevemente como con un corto conocimiento en el funcionamiento 
del protocolo \emph{SSL} se pueden establecer mecanismos para que el tráfico \emph{web} vaya de manera segura; 
uno de nuestros principales objetivos, junto con el de concientizar a las personas de la 
importancia de conocer los sitios por los que se navega. Aunque en nuestro caso abordamos las 
redes pertenecientes a una organización, el contenido de la información es válido para cualquier 
sitio \emph{web} que se visita, ya sea interno o externo. 


% =================================================================== %

\appendix
\chapter{Glosario}

%-*- ES -*-
\section{Terminolog\'{\i}a}

\begin{longtable}{|l|l|} \hline
\textbf{T\'ermino en ingl\'es} & \textbf{Traducci\'on utilizada}
\\ \hline \hline stateless & sin estado
\\ \hline Secure Sockets Layer & capa de Sockets seguros
\\ \hline wildcard & comodín
\\ \hline framework & sistema
\\ \hline man-in-the-middle & hombre en el medio
\\ \hline Sniffing & olfateo
\\ \hline Spoofing & engañar
\\ \hline Self-signed Certificates & certificados auto-firmados
\\ \hline Namespaces & espacios de nombres  \\ \hline
\end{longtable}




%bibliografia
\bibliography{tesis}
\bibliographystyle{acm}
%\bibliographystyle{alpha}


\end{document}
