%-*- ES -*-
%----------------------------------------------------------------------
% capitulo 1: Introduccion
%----------------------------------------------------------------------
% Estructura del capitulo:
%
%----------------------------------------------------------------------


% se pone una introducci�n al tema en general

\section{Objetivos}

Hence, not only the nodes but the communication channel should also be secured.
While developing a secure network, Confidentiality, Integrity, Availability (CIA)
needs to be considered

\begin{itemize}
 \item item 1
 \item item 2
\end{itemize}


\begin{verbatim}
 Este es un bien ambiente para 
poner codigo
\end{verbatim}



%En \cite{Davis:1989} se \footnote{Esta es una nota al pie} ve...

\textit{casa} \\
\emph{casa} \\
\texttt{casa} \\
\textsf{casa} \\
\textbf{casa} \\

\begin{center}
\begin{figure}

\begin{center}
    \includegraphics[width=3cm,height=3cm]{uni.png}
\end{center}

\caption{Esta es la figura del escudo de la uns}
\end{figure}
\end{center}

\section{Plan de tesis y principales contribuciones}\label{secContribuciones}
\begin{itemize}
    \item concientizar la implementacion de medidas de seguridad en redes internas
    \item demostrar como es posible realizar realizar acciones que perjudiquen a una organizacion
    \item implementar una mejora en la navegacion en una red interna utilizando docker
   \end{itemize}
   no se si las contribuciones se refiere a la principal fuente de consulta
\section{Trabajos previos relacionados}
% rese�ar los art�culos hechos en el contexto del trabajo de tesis si los hay

