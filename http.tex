\subsection{¿Que es el protocolo HTTP?}
The Hypertext Transfer Protocol (HTTP) is a stateless application-
   level request/response protocol that uses extensible semantics and
   self-descriptive message payloads for flexible interaction with
   network-based hypertext information systems.
   HTTP is a generic interface protocol for information systems.  It is
   designed to hide the details of how a service is implemented by
   presenting a uniform interface to clients that is independent of the
   types of resources provided.  Likewise, servers do not need to be
   aware of each client's purpose: an HTTP request can be considered in
   isolation rather than being associated with a specific type of client
   or a predetermined sequence of application steps.  The result is a
   protocol that can be used effectively in many different contexts and
   for which implementations can evolve independently over time.
   HTTP is also designed for use as an intermediation protocol for
   translating communication to and from non-HTTP information systems.
   HTTP proxies and gateways can provide access to alternative
   information services by translating their diverse protocols into a
   hypertext format that can be viewed and manipulated by clients in the
   same way as HTTP services.
   One consequence of this flexibility is that the protocol cannot be
   defined in terms of what occurs behind the interface.  Instead, we
   are limited to defining the syntax of communication, the intent of
   received communication, and the expected behavior of recipients.  If
   the communication is considered in isolation, then successful actions
   ought to be reflected in corresponding changes to the observable
   interface provided by servers.  However, since multiple clients might
   act in parallel and perhaps at cross-purposes, we cannot require that
   such changes be observable beyond the scope of a single response.
\subsection{Arquitectura}
HTTP was created for the World Wide Web (WWW) architecture and has
evolved over time to support the scalability needs of a worldwide
hypertext system.  Much of that architecture is reflected in the
terminology and syntax productions used to define HTTP.
\subsubsection{Client/Server Messaging}
HTTP is a stateless request/response protocol that operates by
exchanging messages (Section 3) across a reliable transport- or
session-layer "connection" (Section 6).  An HTTP "client" is a
program that establishes a connection to a server for the purpose of
sending one or more HTTP requests.  An HTTP "server" is a program
that accepts connections in order to service HTTP requests by sending
HTTP responses.
Most HTTP communication consists of a retrieval request (GET) for a
representation of some resource identified by a URI.  In the simplest
case, this might be accomplished via a single bidirectional
connection (===) between the user agent (UA) and the origin
server (O).

         request   ->
    UserAgent ======================================= Origin server
                                <   response

A client sends an HTTP request to a server in the form of a request
message, beginning with a request-line that includes a method, URI,
and protocol version (Section 3.1.1), followed by header fields
containing request modifiers, client information, and representation
metadata (Section 3.2), an empty line to indicate the end of the
header section, and finally a message body containing the payload
body (if any, Section 3.3).
A server responds to a client's request by sending one or more HTTP
   response messages, each beginning with a status line that includes
   the protocol version, a success or error code, and textual reason
   phrase (Section 3.1.2), possibly followed by header fields containing
   server information, resource metadata, and representation metadata
   (Section 3.2), an empty line to indicate the end of the header
   section, and finally a message body containing the payload body (if
   any, Section 3.3).
\subsubsection{Ejemplo}
The following example illustrates a typical message exchange for a
GET request (Section 4.3.1 of [RFC7231]) on the URI
"http://www.example.com/hello.txt":

Client request:

  GET /hello.txt HTTP/1.1
  User-Agent: curl/7.16.3 libcurl/7.16.3 OpenSSL/0.9.7l zlib/1.2.3
  Host: www.example.com
  Accept-Language: en, mi


Server response:

  HTTP/1.1 200 OK
  Date: Mon, 27 Jul 2009 12:28:53 GMT
  Server: Apache
  Last-Modified: Wed, 22 Jul 2009 19:15:56 GMT
  ETag: "34aa387-d-1568eb00"
  Accept-Ranges: bytes
  Content-Length: 51
  Vary: Accept-Encoding
  Content-Type: text/plain

  Hello World! My payload includes a trailing CRLF.

  HTTP is defined as a stateless protocol, meaning that each request
   message can be understood in isolation.  Many implementations depend
   on HTTP's stateless design in order to reuse proxied connections or
   dynamically load balance requests across multiple servers.  Hence, a
   server MUST NOT assume that two requests on the same connection are
   from the same user agent unless the connection is secured and
   specific to that agent.  Some non-standard HTTP extensions (e.g.,
   [RFC4559]) have been known to violate this requirement, resulting in
   security and interoperability problems.
\subsection{Formato del mensaje}
All HTTP/1.1 messages consist of a start-line followed by a sequence
of octets in a format similar to the Internet Message Format
[RFC5322]: zero or more header fields (collectively referred to as
the "headers" or the "header section"), an empty line indicating the
end of the header section, and an optional message body.
The normal procedure for parsing an HTTP message is to read the
   start-line into a structure, read each header field into a hash table
   by field name until the empty line, and then use the parsed data to
   determine if a message body is expected.  If a message body has been
   indicated, then it is read as a stream until an amount of octets
   equal to the message body length is read or the connection is closed.

   A recipient MUST parse an HTTP message as a sequence of octets in an
   encoding that is a superset of US-ASCII [USASCII].  Parsing an HTTP
   message as a stream of Unicode characters, without regard for the
   specific encoding, creates security vulnerabilities due to the
   varying ways that string processing libraries handle invalid
   multibyte character sequences that contain the octet LF (%x0A).
   String-based parsers can only be safely used within protocol elements
   after the element has been extracted from the message, such as within
   a header field-value after message parsing has delineated the
   individual fields.

   An HTTP message can be parsed as a stream for incremental processing
   or forwarding downstream.  However, recipients cannot rely on
   incremental delivery of partial messages, since some implementations
   will buffer or delay message forwarding for the sake of network
   efficiency, security checks, or payload transformations.

   A sender MUST NOT send whitespace between the start-line and the
   first header field.  A recipient that receives whitespace between the
   start-line and the first header field MUST either reject the message
   as invalid or consume each whitespace-preceded line without further
   processing of it (i.e., ignore the entire line, along with any
   subsequent lines preceded by whitespace, until a properly formed
   header field is received or the header section is terminated).

   The presence of such whitespace in a request might be an attempt to
   trick a server into ignoring that field or processing the line after
   it as a new request, either of which might result in a security
   vulnerability if other implementations within the request chain
   interpret the same message differently.  Likewise, the presence of
   such whitespace in a response might be ignored by some clients or
   cause others to cease parsing.

  HTTP-message   = start-line
                   *( header-field CRLF )
                   CRLF
                   [ message-body ]

\subsubsection{Start Line}
An HTTP message can be either a request from client to server or a
   response from server to client.  Syntactically, the two types of
   message differ only in the start-line, which is either a request-line
   (for requests) or a status-line (for responses), and in the algorithm
   for determining the length of the message body (Section 3.3).
   In theory, a client could receive requests and a server could receive
   responses, distinguishing them by their different start-line formats,
   but, in practice, servers are implemented to only expect a request (a
   response is interpreted as an unknown or invalid request method) and
   clients are implemented to only expect a response.

     start-line     = request-line / status-line
\paragraph{Request Line}
A request-line begins with a method token, followed by a single space
   (SP), the request-target, another single space (SP), the protocol
   version, and ends with CRLF.

     request-line   = method SP request-target SP HTTP-version CRLF

   The method token indicates the request method to be performed on the
   target resource.  The request method is case-sensitive.

     method         = token

   The request methods defined by this specification can be found in
   Section 4 of [RFC7231], along with information regarding the HTTP
   method registry and considerations for defining new methods.

   The request-target identifies the target resource upon which to apply
   the request, as defined in Section 5.3.

   Recipients typically parse the request-line into its component parts
   by splitting on whitespace (see Section 3.5), since no whitespace is
   allowed in the three components.  Unfortunately, some user agents
   fail to properly encode or exclude whitespace found in hypertext
   references, resulting in those disallowed characters being sent in a
   request-target.

   Recipients of an invalid request-line SHOULD respond with either a
   400 (Bad Request) error or a 301 (Moved Permanently) redirect with
   the request-target properly encoded.  A recipient SHOULD NOT attempt
   to autocorrect and then process the request without a redirect, since
   the invalid request-line might be deliberately crafted to bypass
   security filters along the request chain.

   HTTP does not place a predefined limit on the length of a
   request-line, as described in Section 2.5.  A server that receives a
   method longer than any that it implements SHOULD respond with a 501
   (Not Implemented) status code.  A server that receives a
   request-target longer than any URI it wishes to parse MUST respond
   with a 414 (URI Too Long) status code (see Section 6.5.12 of
   [RFC7231]).

   Various ad hoc limitations on request-line length are found in
   practice.  It is RECOMMENDED that all HTTP senders and recipients
   support, at a minimum, request-line lengths of 8000 octets.
\paragraph{Status Line}
The first line of a response message is the status-line, consisting
   of the protocol version, a space (SP), the status code, another
   space, a possibly empty textual phrase describing the status code,
   and ending with CRLF.

     status-line = HTTP-version SP status-code SP reason-phrase CRLF

   The status-code element is a 3-digit integer code describing the
   result of the server's attempt to understand and satisfy the client's
   corresponding request.  The rest of the response message is to be
   interpreted in light of the semantics defined for that status code.
   See Section 6 of [RFC7231] for information about the semantics of
   status codes, including the classes of status code (indicated by the
   first digit), the status codes defined by this specification,
   considerations for the definition of new status codes, and the IANA
   registry.

     status-code    = 3DIGIT

   The reason-phrase element exists for the sole purpose of providing a
   textual description associated with the numeric status code, mostly
   out of deference to earlier Internet application protocols that were
   more frequently used with interactive text clients.  A client SHOULD
   ignore the reason-phrase content.

     reason-phrase  = *( HTAB / SP / VCHAR / obs-text )

\subsubsection {Header Fields}
Each header field consists of a case-insensitive field name followed
by a colon (":"), optional leading whitespace, the field value, and
optional trailing whitespace.
header-field   = field-name ":" OWS field-value OWS

field-name     = token
field-value    = *( field-content / obs-fold )
field-content  = field-vchar [ 1*( SP / HTAB ) field-vchar ]
field-vchar    = VCHAR / obs-text

obs-fold       = CRLF 1*( SP / HTAB )
               ; obsolete line folding
               ; see Section 3.2.4

The field-name token labels the corresponding field-value as having
the semantics defined by that header field.  For example, the Date
header field is defined in Section 7.1.1.2 of [RFC7231] as containing
the origination timestamp for the message in which it appears.
\paragraph{Field Extensibility}
Header fields are fully extensible: there is no limit on the
introduction of new field names, each presumably defining new
semantics, nor on the number of header fields used in a given
message.  Existing fields are defined in each part of this
specification and in many other specifications outside this document
set.

New header fields can be defined such that, when they are understood
by a recipient, they might override or enhance the interpretation of
previously defined header fields, define preconditions on request
evaluation, or refine the meaning of responses.

A proxy MUST forward unrecognized header fields unless the field-name
is listed in the Connection header field (Section 6.1) or the proxy
is specifically configured to block, or otherwise transform, such
fields.  Other recipients SHOULD ignore unrecognized header fields.
These requirements allow HTTP's functionality to be enhanced without
requiring prior update of deployed intermediaries.

All defined header fields ought to be registered with IANA in the
"Message Headers" registry, as described in Section 8.3 of [RFC7231].
\paragraph{Field Order}
The order in which header fields with differing field names are
   received is not significant.  However, it is good practice to send
   header fields that contain control data first, such as Host on
   requests and Date on responses, so that implementations can decide
   when not to handle a message as early as possible.  A server MUST NOT
   apply a request to the target resource until the entire request
   header section is received, since later header fields might include
   conditionals, authentication credentials, or deliberately misleading
   duplicate header fields that would impact request processing.

   A sender MUST NOT generate multiple header fields with the same field
   name in a message unless either the entire field value for that
   header field is defined as a comma-separated list (ej: incompleto) 
   %[i.e., #(values)]
   or the header field is a well-known exception (as noted below).

   A recipient MAY combine multiple header fields with the same field
   name into one "field-name: field-value" pair, without changing the
   semantics of the message, by appending each subsequent field value to
   the combined field value in order, separated by a comma.  The order
   in which header fields with the same field name are received is
   therefore significant to the interpretation of the combined field
   value; a proxy MUST NOT change the order of these field values when
   forwarding a message.

      Note: In practice, the "Set-Cookie" header field ([RFC6265]) often
      appears multiple times in a response message and does not use the
      list syntax, violating the above requirements on multiple header
      fields with the same name.  Since it cannot be combined into a
      single field-value, recipients ought to handle "Set-Cookie" as a
      special case while processing header fields.  (See Appendix A.2.3
      of [Kri2001] for details.)
\paragraph{Whitespace}
This specification uses three rules to denote the use of linear
whitespace: OWS (optional whitespace), RWS (required whitespace), and
BWS ("bad" whitespace).

The OWS rule is used where zero or more linear whitespace octets
might appear.  For protocol elements where optional whitespace is
preferred to improve readability, a sender SHOULD generate the
optional whitespace as a single SP; otherwise, a sender SHOULD NOT
generate optional whitespace except as needed to white out invalid or
unwanted protocol elements during in-place message filtering.

The RWS rule is used when at least one linear whitespace octet is
required to separate field tokens.  A sender SHOULD generate RWS as a
single SP.

The BWS rule is used where the grammar allows optional whitespace
only for historical reasons.  A sender MUST NOT generate BWS in
messages.  A recipient MUST parse for such bad whitespace and remove
it before interpreting the protocol element.
OWS            = *( SP / HTAB )
; optional whitespace
RWS            = 1*( SP / HTAB )
; required whitespace
BWS            = OWS
; "bad" whitespace
\paragraph{Field Parsing}
Messages are parsed using a generic algorithm, independent of the
individual header field names.  The contents within a given field
value are not parsed until a later stage of message interpretation
(usually after the message's entire header section has been
processed).  Consequently, this specification does not use ABNF rules
to define each "Field-Name: Field Value" pair, as was done in
previous editions.  Instead, this specification uses ABNF rules that
are named according to each registered field name, wherein the rule
defines the valid grammar for that field's corresponding field values
(i.e., after the field-value has been extracted from the header
section by a generic field parser).

No whitespace is allowed between the header field-name and colon.  In
the past, differences in the handling of such whitespace have led to
security vulnerabilities in request routing and response handling.  A
server MUST reject any received request message that contains
whitespace between a header field-name and colon with a response code
of 400 (Bad Request).  A proxy MUST remove any such whitespace from a
response message before forwarding the message downstream.

A field value might be preceded and/or followed by optional
whitespace (OWS); a single SP preceding the field-value is preferred
for consistent readability by humans.  The field value does not
include any leading or trailing whitespace: OWS occurring before the
first non-whitespace octet of the field value or after the last
non-whitespace octet of the field value ought to be excluded by
parsers when extracting the field value from a header field.

Historically, HTTP header field values could be extended over
multiple lines by preceding each extra line with at least one space
or horizontal tab (obs-fold).  This specification deprecates such
line folding except within the message/http media type
(Section 8.3.1).  A sender MUST NOT generate a message that includes
line folding (i.e., that has any field-value that contains a match to
the obs-fold rule) unless the message is intended for packaging
within the message/http media type.
A server that receives an obs-fold in a request message that is not
within a message/http container MUST either reject the message by
sending a 400 (Bad Request), preferably with a representation
explaining that obsolete line folding is unacceptable, or replace
each received obs-fold with one or more SP octets prior to
interpreting the field value or forwarding the message downstream.

A proxy or gateway that receives an obs-fold in a response message
that is not within a message/http container MUST either discard the
message and replace it with a 502 (Bad Gateway) response, preferably
with a representation explaining that unacceptable line folding was
received, or replace each received obs-fold with one or more SP
octets prior to interpreting the field value or forwarding the
message downstream.

A user agent that receives an obs-fold in a response message that is
not within a message/http container MUST replace each received
obs-fold with one or more SP octets prior to interpreting the field
value.

Historically, HTTP has allowed field content with text in the
ISO-8859-1 charset [ISO-8859-1], supporting other charsets only
through use of [RFC2047] encoding.  In practice, most HTTP header
field values use only a subset of the US-ASCII charset [USASCII].
Newly defined header fields SHOULD limit their field values to
US-ASCII octets.  A recipient SHOULD treat other octets in field
content (obs-text) as opaque data.
\paragraph{Field Limits }
HTTP does not place a predefined limit on the length of each header
field or on the length of the header section as a whole, as described
in Section 2.5.  Various ad hoc limitations on individual header
field length are found in practice, often depending on the specific
field semantics.

A server that receives a request header field, or set of fields,
larger than it wishes to process MUST respond with an appropriate 4xx
(Client Error) status code.  Ignoring such header fields would
increase the server's vulnerability to request smuggling attacks
(Section 9.5).

A client MAY discard or truncate received header fields that are
larger than the client wishes to process if the field semantics are
such that the dropped value(s) can be safely ignored without changing
the message framing or response semantics.
\paragraph{Field Value Components}

Most HTTP header field values are defined using common syntax
components (token, quoted-string, and comment) separated by
whitespace or specific delimiting characters.  Delimiters are chosen
from the set of US-ASCII visual characters not allowed in a token
%(DQUOTE and "(),/:;<=>?@[\]{}").

  token          = 1*tchar

%  tchar          = "!" / "#" / "$" / "%" / "&" / "'" / "*"
%                 / "+" / "-" / "." / "^" / "_" / "`" / "|" / "~"
                 / DIGIT / ALPHA
                 ; any VCHAR, except delimiters

A string of text is parsed as a single value if it is quoted using
double-quote marks.

  quoted-string  = DQUOTE *( qdtext / quoted-pair ) DQUOTE
  qdtext         = HTAB / SP /%x21 / %x23-5B / %x5D-7E / obs-text
  obs-text       = %x80-FF

Comments can be included in some HTTP header fields by surrounding
the comment text with parentheses.  Comments are only allowed in
fields containing "comment" as part of their field value definition.

  comment        = "(" *( ctext / quoted-pair / comment ) ")"
  ctext          = HTAB / SP / %x21-27 / %x2A-5B / %x5D-7E / obs-text

The backslash octet ("\") can be used as a single-octet quoting
mechanism within quoted-string and comment constructs.  Recipients
that process the value of a quoted-string MUST handle a quoted-pair
as if it were replaced by the octet following the backslash.

  quoted-pair    = "\" ( HTAB / SP / VCHAR / obs-text )

A sender SHOULD NOT generate a quoted-pair in a quoted-string except
where necessary to quote DQUOTE and backslash octets occurring within
that string.  A sender SHOULD NOT generate a quoted-pair in a comment
except where necessary to quote parentheses ["(" and ")"] and
backslash octets occurring within that comment.
\subsubsection{Message Body }
The message body (if any) of an HTTP message is used to carry the
payload body of that request or response.  The message body is
identical to the payload body unless a transfer coding has been
applied, as described in Section 3.3.1.

  message-body = *OCTET

The rules for when a message body is allowed in a message differ for
requests and responses.

The presence of a message body in a request is signaled by a
Content-Length or Transfer-Encoding header field.  Request message
framing is independent of method semantics, even if the method does
not define any use for a message body.

The presence of a message body in a response depends on both the
request method to which it is responding and the response status code
(Section 3.1.2).  Responses to the HEAD request method (Section 4.3.2
of [RFC7231]) never include a message body because the associated
response header fields (e.g., Transfer-Encoding, Content-Length,
etc.), if present, indicate only what their values would have been if
the request method had been GET (Section 4.3.1 of [RFC7231]). 2xx
(Successful) responses to a CONNECT request method (Section 4.3.6 of
[RFC7231]) switch to tunnel mode instead of having a message body.
All 1xx (Informational), 204 (No Content), and 304 (Not Modified)
responses do not include a message body.  All other responses do
include a message body, although the body might be of zero length.
\paragraph{Transfer-Encoding }

The Transfer-Encoding header field lists the transfer coding names
corresponding to the sequence of transfer codings that have been (or
will be) applied to the payload body in order to form the message
body.  Transfer codings are defined in Section 4.

%Transfer-Encoding = 1#transfer-coding

Transfer-Encoding is analogous to the Content-Transfer-Encoding field
of MIME, which was designed to enable safe transport of binary data
over a 7-bit transport service ([RFC2045], Section 6).  However, safe
transport has a different focus for an 8bit-clean transfer protocol.
In HTTP's case, Transfer-Encoding is primarily intended to accurately
delimit a dynamically generated payload and to distinguish payload
encodings that are only applied for transport efficiency or security
from those that are characteristics of the selected resource.
A recipient MUST be able to parse the chunked transfer coding
(Section 4.1) because it plays a crucial role in framing messages
when the payload body size is not known in advance.  A sender MUST
NOT apply chunked more than once to a message body (i.e., chunking an
already chunked message is not allowed).  If any transfer coding
other than chunked is applied to a request payload body, the sender
MUST apply chunked as the final transfer coding to ensure that the
message is properly framed.  If any transfer coding other than
chunked is applied to a response payload body, the sender MUST either
apply chunked as the final transfer coding or terminate the message
by closing the connection.

For example,

  Transfer-Encoding: gzip, chunked

indicates that the payload body has been compressed using the gzip
coding and then chunked using the chunked coding while forming the
message body.

Unlike Content-Encoding (Section 3.1.2.1 of [RFC7231]),
Transfer-Encoding is a property of the message, not of the
representation, and any recipient along the request/response chain
MAY decode the received transfer coding(s) or apply additional
transfer coding(s) to the message body, assuming that corresponding
changes are made to the Transfer-Encoding field-value.  Additional
information about the encoding parameters can be provided by other
header fields not defined by this specification.

Transfer-Encoding MAY be sent in a response to a HEAD request or in a
304 (Not Modified) response (Section 4.1 of [RFC7232]) to a GET
request, neither of which includes a message body, to indicate that
the origin server would have applied a transfer coding to the message
body if the request had been an unconditional GET.  This indication
is not required, however, because any recipient on the response chain
(including the origin server) can remove transfer codings when they
are not needed.

A server MUST NOT send a Transfer-Encoding header field in any
response with a status code of 1xx (Informational) or 204 (No
Content).  A server MUST NOT send a Transfer-Encoding header field in
any 2xx (Successful) response to a CONNECT request (Section 4.3.6 of
[RFC7231]).

Transfer-Encoding was added in HTTP/1.1.  It is generally assumed
that implementations advertising only HTTP/1.0 support will not
understand how to process a transfer-encoded payload.  A client MUST
NOT send a request containing Transfer-Encoding unless it knows the
server will handle HTTP/1.1 (or later) requests; such knowledge might
   be in the form of specific user configuration or by remembering the
   version of a prior received response.  A server MUST NOT send a
   response containing Transfer-Encoding unless the corresponding
   request indicates HTTP/1.1 (or later).

   A server that receives a request message with a transfer coding it
   does not understand SHOULD respond with 501 (Not Implemented).

\paragraph{Content-Length }
When a message does not have a Transfer-Encoding header field, a
   Content-Length header field can provide the anticipated size, as a
   decimal number of octets, for a potential payload body.  For messages
   that do include a payload body, the Content-Length field-value
   provides the framing information necessary for determining where the
   body (and message) ends.  For messages that do not include a payload
   body, the Content-Length indicates the size of the selected
   representation (Section 3 of [RFC7231]).

     Content-Length = 1*DIGIT

   An example is

     Content-Length: 3495

   A sender MUST NOT send a Content-Length header field in any message
   that contains a Transfer-Encoding header field.

   A user agent SHOULD send a Content-Length in a request message when
   no Transfer-Encoding is sent and the request method defines a meaning
   for an enclosed payload body.  For example, a Content-Length header
   field is normally sent in a POST request even when the value is 0
   (indicating an empty payload body).  A user agent SHOULD NOT send a
   Content-Length header field when the request message does not contain
   a payload body and the method semantics do not anticipate such a
   body.

   A server MAY send a Content-Length header field in a response to a
   HEAD request (Section 4.3.2 of [RFC7231]); a server MUST NOT send
   Content-Length in such a response unless its field-value equals the
   decimal number of octets that would have been sent in the payload
   body of a response if the same request had used the GET method.

   A server MAY send a Content-Length header field in a 304 (Not
   Modified) response to a conditional GET request (Section 4.1 of
   [RFC7232]); a server MUST NOT send Content-Length in such a response
   unless its field-value equals the decimal number of octets that would
   have been sent in the payload body of a 200 (OK) response to the same
   request.

   A server MUST NOT send a Content-Length header field in any response
   with a status code of 1xx (Informational) or 204 (No Content).  A
   server MUST NOT send a Content-Length header field in any 2xx
   (Successful) response to a CONNECT request (Section 4.3.6 of
   [RFC7231]).

   Aside from the cases defined above, in the absence of
   Transfer-Encoding, an origin server SHOULD send a Content-Length
   header field when the payload body size is known prior to sending the
   complete header section.  This will allow downstream recipients to
   measure transfer progress, know when a received message is complete,
   and potentially reuse the connection for additional requests.

   Any Content-Length field value greater than or equal to zero is
   valid.  Since there is no predefined limit to the length of a
   payload, a recipient MUST anticipate potentially large decimal
   numerals and prevent parsing errors due to integer conversion
   overflows (Section 9.3).

   If a message is received that has multiple Content-Length header
   fields with field-values consisting of the same decimal value, or a
   single Content-Length header field with a field value containing a
   list of identical decimal values (e.g., "Content-Length: 42, 42"),
   indicating that duplicate Content-Length header fields have been
   generated or combined by an upstream message processor, then the
   recipient MUST either reject the message as invalid or replace the
   duplicated field-values with a single valid Content-Length field
   containing that decimal value prior to determining the message body
   length or forwarding the message.

      Note: HTTP's use of Content-Length for message framing differs
      significantly from the same field's use in MIME, where it is an
      optional field used only within the "message/external-body"
      media-type.

\paragraph{Message Body Length }
The length of a message body is determined by one of the following
(in order of precedence):

1.  Any response to a HEAD request and any response with a 1xx
    (Informational), 204 (No Content), or 304 (Not Modified) status
    code is always terminated by the first empty line after the
    header fields, regardless of the header fields present in the
    message, and thus cannot contain a message body.

2.  Any 2xx (Successful) response to a CONNECT request implies that
    the connection will become a tunnel immediately after the empty
    line that concludes the header fields.  A client MUST ignore any
    Content-Length or Transfer-Encoding header fields received in
    such a message.

3.  If a Transfer-Encoding header field is present and the chunked
    transfer coding (Section 4.1) is the final encoding, the message
    body length is determined by reading and decoding the chunked
    data until the transfer coding indicates the data is complete.

    If a Transfer-Encoding header field is present in a response and
    the chunked transfer coding is not the final encoding, the
    message body length is determined by reading the connection until
    it is closed by the server.  If a Transfer-Encoding header field
    is present in a request and the chunked transfer coding is not
    the final encoding, the message body length cannot be determined
    reliably; the server MUST respond with the 400 (Bad Request)
    status code and then close the connection.

    If a message is received with both a Transfer-Encoding and a
    Content-Length header field, the Transfer-Encoding overrides the
    Content-Length.  Such a message might indicate an attempt to
    perform request smuggling (Section 9.5) or response splitting
    (Section 9.4) and ought to be handled as an error.  A sender MUST
    remove the received Content-Length field prior to forwarding such
    a message downstream.

4.  If a message is received without Transfer-Encoding and with
    either multiple Content-Length header fields having differing
    field-values or a single Content-Length header field having an
    invalid value, then the message framing is invalid and the
    recipient MUST treat it as an unrecoverable error.  If this is a
    request message, the server MUST respond with a 400 (Bad Request)
    status code and then close the connection.  If this is a response
    message received by a proxy, the proxy MUST close the connection
    to the server, discard the received response, and send a 502 (Bad
    Gateway) response to the client.  If this is a response message
    received by a user agent, the user agent MUST close the
    connection to the server and discard the received response.

5.  If a valid Content-Length header field is present without
    Transfer-Encoding, its decimal value defines the expected message
    body length in octets.  If the sender closes the connection or
    the recipient times out before the indicated number of octets are
    received, the recipient MUST consider the message to be
    incomplete and close the connection.

6.  If this is a request message and none of the above are true, then
    the message body length is zero (no message body is present).

7.  Otherwise, this is a response message without a declared message
    body length, so the message body length is determined by the
    number of octets received prior to the server closing the
    connection.

Since there is no way to distinguish a successfully completed,
close-delimited message from a partially received message interrupted
by network failure, a server SHOULD generate encoding or
length-delimited messages whenever possible.  The close-delimiting
feature exists primarily for backwards compatibility with HTTP/1.0.

A server MAY reject a request that contains a message body but not a
Content-Length by responding with 411 (Length Required).

Unless a transfer coding other than chunked has been applied, a
client that sends a request containing a message body SHOULD use a
valid Content-Length header field if the message body length is known
in advance, rather than the chunked transfer coding, since some
existing services respond to chunked with a 411 (Length Required)
status code even though they understand the chunked transfer coding.
This is typically because such services are implemented via a gateway
that requires a content-length in advance of being called and the
server is unable or unwilling to buffer the entire request before
processing.

A user agent that sends a request containing a message body MUST send
a valid Content-Length header field if it does not know the server
will handle HTTP/1.1 (or later) requests; such knowledge can be in
the form of specific user configuration or by remembering the version
of a prior received response.

If the final response to the last request on a connection has been
completely received and there remains additional data to read, a user
agent MAY discard the remaining data or attempt to determine if that
data belongs as part of the prior response body, which might be the
case if the prior message's Content-Length value is incorrect.  A
client MUST NOT process, cache, or forward such extra data as a
separate response, since such behavior would be vulnerable to cache
poisoning.
\subsubsection{Handling Incomplete Messages}
A server that receives an incomplete request message, usually due to
   a canceled request or a triggered timeout exception, MAY send an
   error response prior to closing the connection.

   A client that receives an incomplete response message, which can
   occur when a connection is closed prematurely or when decoding a
   supposedly chunked transfer coding fails, MUST record the message as
   incomplete.  Cache requirements for incomplete responses are defined
   in Section 3 of [RFC7234].

   If a response terminates in the middle of the header section (before
   the empty line is received) and the status code might rely on header
   fields to convey the full meaning of the response, then the client
   cannot assume that meaning has been conveyed; the client might need
   to repeat the request in order to determine what action to take next.

   A message body that uses the chunked transfer coding is incomplete if
   the zero-sized chunk that terminates the encoding has not been
   received.  A message that uses a valid Content-Length is incomplete
   if the size of the message body received (in octets) is less than the
   value given by Content-Length.  A response that has neither chunked
   transfer coding nor Content-Length is terminated by closure of the
   connection and, thus, is considered complete regardless of the number
   of message body octets received, provided that the header section was
   received intact.
\subsubsection{Message Parsing Robustness}
Older HTTP/1.0 user agent implementations might send an extra CRLF
after a POST request as a workaround for some early server
applications that failed to read message body content that was not
terminated by a line-ending.  An HTTP/1.1 user agent MUST NOT preface
or follow a request with an extra CRLF.  If terminating the request
message body with a line-ending is desired, then the user agent MUST
count the terminating CRLF octets as part of the message body length.

In the interest of robustness, a server that is expecting to receive
and parse a request-line SHOULD ignore at least one empty line (CRLF)
received prior to the request-line.

Although the line terminator for the start-line and header fields is
the sequence CRLF, a recipient MAY recognize a single LF as a line
terminator and ignore any preceding CR.

Although the request-line and status-line grammar rules require that
each of the component elements be separated by a single SP octet,
recipients MAY instead parse on whitespace-delimited word boundaries
and, aside from the CRLF terminator, treat any form of whitespace as
the SP separator while ignoring preceding or trailing whitespace;
such whitespace includes one or more of the following octets: SP,
HTAB, VT (%x0B), FF (%x0C), or bare CR.  However, lenient parsing can
result in security vulnerabilities if there are multiple recipients
of the message and each has its own unique interpretation of
robustness (see Section 9.5).

When a server listening only for HTTP request messages, or processing
what appears from the start-line to be an HTTP request message,
receives a sequence of octets that does not match the HTTP-message
grammar aside from the robustness exceptions listed above, the server
SHOULD respond with a 400 (Bad Request) response.
\subsection{Métodos del protocolo HTTP}
\subsection{HTTPS con SSL} 